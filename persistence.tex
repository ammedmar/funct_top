
\section{Sublevel filtrations and persistence theory} \label{s:persistence}

\todo{start with the standard vector space setting, before generalizing?}
Let $R$ be a ring. The category of \emph{persistence $R$-modules} is the functor category  $(\Mod{R})^{\mathbf{R}}$, where $\mathbf{R}$ is the category corresponding to the real numbers $\mathbb{R}$ considered as a poset. We will simply say \emph{persistence module} if the coefficient ring is clear from the context. 

A persistence module $M$ is called \emph{pointwise finitely generated (p.f.g.)} if $M_{t}$ is a finitely generated module for every $t\in\mathbb{R}$. $M$ is called \emph{q-tame} if $\im (M_{s}\to M_{t})$ is finitely generated for all $s<t\in\mathbb{R}$.
\todo{note that for Noetherian rings this implies finitely presented.}

The theory of persistence modules is particularly nice if $R=\mathbb{F}$ is a field, which we will assume for the rest of this section. In this case, we will say that $M$ is \emph{pointwise finite-dimensional (p.f.d.)} instead of p.f.g.

What makes the field coefficient case nice is that it allows for a combinatorial description of many persistence modules in terms of basic summands. Concretely, we say that $M$ has a \emph{barcode} if there is a family of intervals $(I_{\alpha})_{\alpha}$, $I_{\alpha}\subseteq\mathbb{R}$ such that
\[
M\cong\bigoplus_{\alpha} C(I_{\alpha}),
\]
where the persistence module $C(I)$ for an interval $I$ is defined via
\[
C(I)_t=
\begin{cases}
    \mathbb{F} & \text{if } t\in I,\\
    0              & \text{otherwise,}
\end{cases}
\qquad
\text{with structure maps}
\qquad
C(I)_{t,u}=
\begin{cases}
    \operatorname{id}_{\mathbb{F}} & \text{if } t,u\in I,\\
    0              & \text{otherwise.}
\end{cases}
\]

Interval modules are indecomposable in the category of persistence modules and by the Krull--Remak--Schmidt--Azumaya Theorem \cite{MR37832}, two direct sums of interval modules can only be isomorphic if the corresponding families of intervals already agree. Hence, we can speak of \emph{the} barcode of a persistence module and use it as a complete isomorphism invariant.

Unfortunately, not every persistence module has a barcode, so general existence results are important to have. The most widely used of these is Crawley-Boevey's Theorem \cite{Crawley-Boevey.2015}, which asserts that every p.f.d.\ persistence module has a barcode. 

As is well-known, Crawley-Boevey's Theorem does not extend to q-tame persistence modules. However, there are still ways to get a notion of barcode for q-tame persistence modules, for example by phrasing the definition of barcode in terms of so-called rectangle measures \cite{Chazal.2016a}. Other methods include using a slightly weaker notion of isomorphism for the direct sum decomposition into interval modules \cite{Chazal.2016b} or imposing additional algebraic constraints and allowing for the decompositions to be in terms of products instead of sums \cite{schmahl2020structure}. For each of these alternative methods, restricting to the p.f.d.\ setting recovers the basic notion of barcode discussed before.

The usefulness of barcodes in applications often rests on their stability. To make precise what stability means, we define the \emph{interleaving distance} between two persistence modules $M$ and $N$ as 
\[
d_{I}(M,N)=\inf\{\delta>0\mid\text{there exists a }\delta\text{-interleaving between }M\text{ and }N\},
\]
where a $\delta$-interleaving between $M$ and $N$ is given by a collection of maps $M_{t}\to N_{t+\delta}$ and $N_{t}\to M_{t+\delta}$ for each $t\in\mathbb{R}$ such that the diagrams
\[
\begin{tikzcd}
M_s\arrow[r]\arrow[d] & M_t\arrow[d] \\
N_{s+\delta}\arrow[r] & N_{t+\delta}
\end{tikzcd}
\qquad
\begin{tikzcd}
N_s\arrow[r]\arrow[d] & N_t\arrow[d] \\
M_{s+\delta}\arrow[r] & M_{t+\delta}
\end{tikzcd}
\qquad
\begin{tikzcd}
M_{t-\delta}\arrow[rd]\arrow[r]& M_{t}\arrow[rd]\arrow[r] & M_{t+\delta} \\
N_{t-\delta}\arrow[ru]\arrow[r]& N_{t}\arrow[ru]\arrow[r] & N_{t+\delta}
\end{tikzcd}
\]
commute for all $s<t\in\mathbb{R}$.

We also define the \emph{Bottleneck distance} between two barcodes $B=(I_{\alpha})_{\alpha\in A}$ and $B'=(I'_{\alpha'})_{\alpha'\in A'})$ as
\[
d_{b}(B,B')=\inf\{\delta>0\mid\text{there exists a }\delta\text{-matching between }M\text{ and }N\},
\]
where a $\delta$-matching between $B$ and $B'$ is given by subsets $X\subseteq A$, $X'\subseteq A'$ and a bijection $\sigma\colon X\to X'$ such that
\[
\lvert \sup I - \inf I \rvert < 2\delta
\]
whenever there is $\alpha\in A\setminus X$ with $I=I_{\alpha}$ or $\alpha'\in A'\setminus X'$ with $I=I'_{\alpha'}$, and
\begin{align*}
\lvert \sup I'_{\sigma(\alpha)} - \sup I_{\alpha} \rvert &< \delta \\
\lvert \inf I'_{\sigma(\alpha)} - \inf I_{\alpha} \rvert &< \delta
\end{align*}
for all $\alpha\in X$.

\begin{thm}[\cite{}]\todo{what do we want to cite here?}
Let $M$ and $M'$ be q-tame persistence modules with barcodes $B$ and $B'$. Then
\[
d_{b}(B,B')= d_{I}(M,M').
\]
\end{thm}

The most commonly studied example of a persistence module is \emph{persistent homology}, i.e., the persistence module obtained by applying homology to an $\mathbb{R}$-indexed diagram of spaces. In what follows, we will study the persistent homology of certain classes of such diagrams, namely filtrations.

A \textit{sublevel filtration} is a pair $(X, f)$ with $X$ a Hausdorff space and $f$ a real-valued function on $X$.
For any $t \in \R$, the \textit{$t$-sublevel set} is defined by 
\begin{equation*}
X_{\leq t} = f^{-1}((-\infty, t]).
\end{equation*}

Recall that a subset $N$ of a space $X$ is said to be a \textit{neighborhood} of $x$ in $X$ if there is an open set $U$ with $x \in U \subseteq N$, and that a Hausdorff space is \textit{locally compact} if for every $x$ in $X$, every neighborhood of $x$ contains a compact neighborhood of $x$.

A sublevel filtration $(X, f)$ is said to be a \textit{Morse filtration} if $X$ is a locally compact space and $X_{\leq t}$ is compact for every $t \in \R$.
Notice that this condition implies, since $X$ is Hausdorff, that $f$ is \textit{lower semi-continuous}, i.e, the preimage of open sets of the form $(t, +\infty)$ is open.