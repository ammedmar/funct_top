
\subsection{Sublevel filtrations}

A \textit{sublevel filtration} is a pair $(X, f)$ with $X$ a Hausdorff space and $f$ a real-valued function on $X$.
For any $t \in \R$, the \textit{$t$-sublevel set} is defined by 
\begin{equation*}
X_{\leq t} = f^{-1}((-\infty, t]).
\end{equation*}

Recall that a subset $N$ of a space $X$ is said to be a \textit{neighborhood} of $x \in X$ if there is an open set $U$ with $x \in U \subseteq N$, and that a Hausdorff space is \textit{locally compact} if for every open neighborhood $U$ of $x$ there exists a neighborhood $N$ of $x$ such that $N \subseteq U$.

A sublevel filtration $(X, f)$ is said to be a \textit{Morse filtration} if $X$ is a locally compact space and $X_{\leq t}$ is compact for every $t \in \R$.
Notice that this condition implies, since $X$ is Hausdorff, that $f$ is \textit{lower semi-continuous}, i.e, the preimage of open sets of the form $(t, +\infty)$ is open.