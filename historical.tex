
\section{Historical remarks} \label{s:historical}

The approach of Morse and Tompkins in \cite{Morse.1939} was to give a topological definition of critical points in the general setting, prove existence of these via Morse inequalities for the so-called Douglas functional, and then to prove that these topological critical points correspond to minimal surfaces. For a more detailed review of their work we refer to \cite{Struwe.1988}. In the minimal surface community, these methods do not seem to be deemed very helpful, as is reflected e.gchoosing.\@ in \cite[p.472]{MR2566897}, where the authors state that Morse's "...general Morse-theoretic statements are more or less useless as they are based on topological assumptions which cannot be verified in a concrete situation." More comments on the perceived unsuitability of the theory of Functional Topology for minimal surfaces are made by Struwe in \cite{MR850612}. 

Morse assumes all spaces in his filtrations to be compact and uses Vietoris homology, which agrees with \v{C}ech homology by Dowker's Theorem \cite{Dowker.1952}. This implies semi-continuity by a general result stating that, for field coefficients, \v{C}ech homology commutes with inverse limits for compact Hausdorff spaces \cite[Theorem VIII.3.6 and Theorem X.3.1]{MR0050886}. To achieve q-tameness, Morse wants to use the following condition, which he refers to as \textit{local $F$-connectivity}. 
\begin{displaycquote}[p.431]{Morse.1940}
	The space $M$ is said to be \textit{locally $F$-connected} of order $r$ at $p$ if corresponding to each positive constant $e$ there exists a positive constant $\delta$ such that each singular $r$-sphere on the $\delta$-neighborhood of $p$ on $F_{c+\delta}$ bounds an $(r+1)$-cell of norm $e$ on $F_{c+e}$.
\end{displaycquote}
Morse then states that as consequence of bounded compactness and local $F$-connectivity the rank of the persistence Vietoris homology module associated to $(M, F)$ is q-tame.
In the original it reads:
\begin{displaycquote}[Theorem 6.3, p.432]{Morse.1940}
	Let $a$ and $c$ be positive constants such that $a < c < 1$.
	The $k^{\mathrm{th}}$ connectivity $R^k(a,c)$ of $F_a$ on $F_c$ is finite.
\end{displaycquote}
The proof Morse cites, found in \cite[Theorem 6.1]{Morse.1938}, is not correct; neither is the statement.
A stronger notion of local connectivity for the pair $(M, F)$ is needed to prove q-tameness.
Such condition had already appeared in print, introduced by Morse himself, three years earlier.
\begin{displaycquote}[p.421-422]{Morse.1937}
	The space $M$ will be said to be locally $F$-connected for the order $n$ if corresponding to $n$, an arbitrary point $p$ on $M$, and an arbitrary positive constant $e$, there exists a positive constant $\delta$ with the following property. For $c \geq F(p)$ any singular $n$-sphere on $F \leq c$ (the continuous image on $F \leq c$ of an ordinary $n$-sphere) on the $\delta$-neighborhood $p_{\delta}$ of $p$ is the boundary of a singular $(n + 1)$-cell on $F \leq c + e$ and on $p_e$.
\end{displaycquote}
This notion of local-$F$-connectedness is stronger than the one first quoted, with the key difference being the freedom granted by the constant $c \geq F(p)$.
Following the introduction of this stronger definition Morse states as \cite[Theorem~9.2, p.422]{Morse.1937} that q-tameness follows from this form of local $F$-connectivity and bounded compactness, adding that the proof ``while not difficult involves too great detail to be presented here".

%From our current viewpoint it is simple to identify an important problem with both of these definitions of local $F$-connectivity, since they seem to assume that all singular homology classes can be represented by spheres, a fact that the fundamental class of a torus disproves.
%Nonetheless, the distinction between the two teach us valuable lessons: q-tameness follows from bounded compactness and a modern interpretation of the second but not the first.\footnote{These are presented as Definitions \ref{defi:local_connectedness} and \ref{defi:strong_local_connectedness}}


%By now, however, Morse's ideas from Functional Topology, and in particular from \cite{Morse.1940}, have been rediscovered in a different context, namely topological data analysis. More specifically, Morse studies how the homology of the sublevel sets of a function change as the level changes. He defines spaces that encode the same information as the modern persistent homology groups, as well as quantities that correspond to the modern notions of birth and death. Even though he did not have a structure theory of persistence modules via barcodes, he emphasized many points that are now known to be connected to this theory: The use of field coefficients, as well as ensuring that the occuring persistence modules are upper semi-continuous and q-tame (which implies the existence of interval decompositions \cite{schmahl2020structure}).