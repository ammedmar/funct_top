
\begin{abstract}
	During the 1930s, Marston Morse developed a vast generalization of what is commonly known as Morse theory relating the critical points of a semi-continuous functional with the topology of its sublevel sets.
	Morse and Tompkins applied this body of work, referred to as functional topology, to prove the Unstable Minimal Surface Theorem in the setting defined by Douglas' solution to Plateau's Problem.
	Several concepts introduced by Morse in this context can be seen as early precursors to the theory of persistent homology, which by now has established itself as a popular tool in applied and theoretical mathematics.
	In this article, we provide a modern redevelopment of the homological aspects of Morse's functional topology from the perspective of persistence theory.
	We adjust several key definitions and prove stronger statements, including a generalized version of the Morse inequalities, in order to allow for novel uses of persistence techniques in functional analysis and symplectic geometry.
	As an application, we identify and correct a mistake in the proof of the Unstable Minimal Surface Theorem by Morse and Tompkins.
\end{abstract}