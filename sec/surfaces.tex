%!TEX root = ../func_top.tex

\section{Persistent homology and functional topology} \label{s:surfaces}

Having established topological conditions for the existence of persistence diagrams associated to filtrations and corresponding generalized Morse inequalities, we now describe how these results relate to Morse's general theory of functional topology \cite{Morse.1937, Morse.1938, Morse.1940}, and to its application in Morse and Tompkins' work on unstable minimal surfaces from \cite{Morse.1939}.
We will focus on the homological aspects of this approach referring to \cite[Sections 4.3--5]{Bott.1980} for a general exposition.
For a more thorough presentation of the unstable minimal surface problem, including the analytical details, please consult \cite[Section II.6]{Struwe.1988}, and for a historical account and an overview of subsequent results see \cite[Section 6]{Dierkes.2010} and \cite[Section 6.8.1]{Dierkes.2010b}.

\subsection{The unstable minimal surface problem}

Morse and Tompkins considered the following setting introduced by Douglas.
Let $g \colon \R \to \R^n$ be a $2\pi$-periodic function representing a simple closed curve such that $g$ is differentiable with Lipschitz derivative.
Let $\widetilde{\Omega}$ be the space of continuous non-decreasing functions $\varphi \colon \R \to \R$ with $\varphi(t+2\pi) = \varphi(t) + 2\pi$ for all $t$ and $\varphi(\alpha_i)=\alpha_i$ for three fixed distinct points $\alpha_i \in [0,2\pi)$.
The \emph{Douglas functional} on $\widetilde \Omega$ associated to the curve $g$ is defined as
\begin{equation*}
A_g(\varphi) = \frac{1}{16 \pi} \int_0^{2\pi} \int_0^{2\pi}  \left\| \frac{g(\varphi(\alpha)) - g(\varphi(\beta))}{\sin \frac{\alpha-\beta}{2}} \right\|_2^2 \ \mathrm{d}\alpha \ \mathrm{d}\beta.
\end{equation*}
It coincides with the Dirichlet energy of the unique harmonic extension of the reparametrized curve $g \circ \varphi$ to a parametrized surface.
The Dirichlet energy is an upper bound for the area, with equality if the parametrization is conformal.
Let $\Omega_g = \{\varphi \in \widetilde\Omega \mid A_g(\varphi) < \infty\}$, equipped with the $C^0$ metric.
Douglas proved that, for a large class of curves~$g$, the set $\Omega_g$ is non-empty and the sublevel sets of $A_g$ are compact.
Since $A_g$ is bounded below by $0$, this implies that $A_g$ attains a global minimum.
The corresponding surface is then a solution of \emph{Plateau's Problem}, which asks for a surface homeomorphic to a disk with boundary $g$ and minimum area.

As alluded to in \cref{r:homotopically critial points}, Morse and Tompkins consider a homotopical notion of critical point for a general function $F \colon M \to \R$ on a metric space \cite[p.~445]{Morse.1939}, see also \cite{Morse.1943}.
Roughly, a point $p$ is \emph{homotopically critical} if it has no neighborhood in $M_{\leq F(p)}$ that can be mapped by a homotopy into $M_{\leq t}$ for some $t<F(p)$.

Morse and Tompkins prove \cite[p.~464]{Morse.1939} that each homotopically critical point of the functional $A_g$ indeed corresponds to
%
%a critical point of the area functional,
%called
a \emph{minimal surface} -- a surface with vanishing mean curvature -- and use this correspondence to prove the following result, also reviewed in \cite[Theorem II.6.10]{Struwe.1988}.

\begin{thm}[{Unstable Minimal Surface Theorem \cite[p.~472]{Morse.1939}}]
	If the space $\Omega_g$ contains two distinct solutions of Plateau's Problem contained in disjoint critical sets of the functional $A_g$, then it also contains a critical set not of minimum type.
	%, more precisely, a homotopically critical point of $A_g$ of non-minimum type.
\end{thm}
Here, a \emph{critical set} $S$ is a closed and open subspace of the set of all critical points with a given function value.
It is said to be of \emph{minimum type} if for every neighborhood $N$ of $S$, the function value on the complement $N \setminus S$ exceeds the function value on $S$.
%it corresponds to a homotopically critical point of $A_g$ of non-minimum type?
In addition to the general theorem above, Morse and Tompkins also give an explicit example of a curve $g$ satisfying the assumptions, thus proving the existence of an unstable minimal surface in the above sense.

%sci-hub.se/10.1215/S0012-7094-41-00828-1
%
%A point q of M will be termed non-minimizing if in every neighborhood of q there is a point p such thatF(p)<F(q). A homotopic critical set which is not a minimizing set contains at least one non-minimizing point of F.

While more efficient and more general proofs for the existence of an unstable minimal surface (with respect to more natural topologies than $C^0$) have subsequently been established \cite{Struwe.1988,Dierkes.2010}, including less restrictive assumptions on the boundary curve, the original approach of Morse and Tompkins is notable for its close connection to persistent homology.


\subsection{Morse's local connectivity conditions}

Parts of the framework of functional topology that Morse and Tompkins use is developed by Morse in \cite{Morse.1940} in a very general setting.
In particular, he proves inequalities for cap numbers associated to the persistent \v{C}ech homology of the sublevel set filtration, as also considered in \cref{s:inequalities}.
From these generalized Morse inequalities, the existence of an unstable minimal surface can easily be deduced in the presence of two distinct critical sets of minimum type.

We have seen that these inequalities require q-tameness, so in order to apply them in the minimal surface setting % and deduce their theorem,
one needs to prove the \mbox{q-tameness} of the sublevel set filtration of the Douglas functional $A_g$.
Throughout his work on functional topology, in order to obtain \mbox{q-tameness}, Morse assumed slightly varying forms of local connectivity on the resulting sublevel set filtrations.
In particular, Morse and Tompkins used the following condition in their applications to minimal surface theory (see also \cite[p.~ 25]{Morse.1938} and \cite[p.~464]{Morse.1939}, but note that the definitions given there contain typographical errors):
\begin{displaycquote}[p.~431]{Morse.1940}
%\cite[p.~ 25]{Morse.1938}
	Let $p$ be a point of $M$ at which $F(p)=c$.
	The space $M$ is said to be \emph{locally $F$-connected} of order $r$ at~$p$ if corresponding to each positive constant $e$ there exists a positive constant $\delta$ such that each singular $r$-sphere on the $\delta$-neighborhood of $p$ and on $F_{c+\delta}$ bounds an $(r+1)$-cell of norm $e$ on $F_{c+e}$.
\end{displaycquote}
Using similar language to the one used in \cref{s:connectivity}, the property of local $F$-connectedness of all orders is equivalent to the following notion applicable to general topological spaces.

\begin{defi}
	The sublevel set filtration of a function $f \colon X \to \R$ is said to be \emph{weakly homotopically locally connected}, or \emph{weakly $\piLC$}, if for any $x \in X$, $V$ a neighborhood of $x$, and any index $t > f(x)$, there is an index $s$ with $f(x) < s < t$ and a neighborhood $U$ of $x$ with $U \subseteq V$ such that the inclusion $f_{\leq s} \cap U \to f_{\leq t} \cap V$ induces trivial maps on homotopy groups.
\end{defi}

Morse then goes on to claim that the persistent \v{C}ech homology of this sublevel set filtration is q-tame, provided that $F$ is bounded from below and satisfies the assumptions of local $F$-connectivity and compactness of sublevel sets.
%, and a further condition called $F$-regularity, which is trivially satisfied if the domain of $F$ is compact.
In the original (where the function is assumed to take values in $[0,1)$) the claim reads:
\begin{displaycquote}[Theorem 6.3, p.~432]{Morse.1940}
	Let $a$ and $c$ be positive constants such that $a < c < 1$.
	The $k^{\mathrm{th}}$ connectivity $R^k(a,c)$ of $F_a$ on $F_c$ is finite.
\end{displaycquote}
Morse does not prove this statement in the given reference, but rather refers to \cite[Theorem~6.1]{Morse.1938}.
Unfortunately, the above claim does not hold in general, as exemplified by the sublevel set filtration from \cref{e:counterexample}.
To elaborate on this, we consider a stronger version of weak local connectedness.

\begin{defi}
	The sublevel set filtration of a function $f \colon X \to \R$ is said to be \emph{weakly locally connected} or \emph{weakly $\LC$} if for any $x \in X$, $V$ a neighborhood of $x$, and any index $t > f(x)$, there is an index $s$ with $f(x) < s < t$ and a neighborhood $U$ of $x$ with $U \subseteq V$ such that the inclusion $f_{\leq s} \cap U \to f_{\leq t} \cap V$ is homotopic to a constant map.
\end{defi}

Clearly, being weakly $\LC$ implies being weakly $\piLC$ and, if the homology $\H$ takes finite dimensional values on one-point spaces, also weakly $\HLC$.
Observe that \cref{e:counterexample} actually establishes that the filtration given there is weakly $\LC$, so not even the weak $\LC$ condition is sufficient to ensure the q-tameness of compact sublevel set filtrations that are induced by non-continuous functions in general.
In particular, our construction invalidates Morse's claim quoted above because \v{C}ech homology satisfies the assumptions on the homology theory made in \cref{e:counterexample}.

Specifically, using the the fact that \v{C}ech homology of compact Hausdorff spaces commutes with inverse limits, it is straightforward to verify that the \v{C}ech homology in degree $d$ of the $d$-dimensional Hawaiian earring is isomorphic to $\prod_{n\in\N}\F$, which is infinite dimensional over $\F$.
%To see that his is the case, one can use the fact that \v{C}ech homology commutes with totally ordered limits for compact Hausdorff spaces \cite[Theorems VIII.3.6 and X.3.1]{Eilenberg.1952} as follows.
%Define
%\begin{align*}
%\HE_k &=
%\left\{ (x_0, \dots, x_d) \in \R^{d+1} \ \middle| \ \left( x_0 - \frac{1}{k} \right)^2 + x_1^2 + \dots + x_d^2 \leq \left( \frac{1}{k} \right)^2 \right\} \\ &\, \cup
%\bigcup_{n=1}^{k-1} \left\{ (x_0, \dots, x_d) \in \R^{d+1} \ \middle |\ \left( x_0 - \frac{1}{n} \right)^2 + x_1^2 + \dots + x_d^2 = \left( \frac{1}{n} \right)^2 \right\},
%\end{align*}
%i.e., the $d$-dimensional Hawaiian earring but with the $k$-th largest $d$-sphere filled.
%We have $\lim_{k} \HE_{k} = \bigcap_{k} \HE_{k} = \HE$, and hence $\CH_{d}(\HE; \F) = \lim_{k} \CH_{d}(\HE_{k}; \F)$, where $\CH$ denotes \v{C}ech homology.
%Clearly, each $\HE_{k}$ is a CW-complex, so we can simply use cellular homology to compute
%\begin{equation*}
%\lim_{k}\CH_{d}(\mathbb{H}^{d}_{k}; \F)=\lim\left(\dots\to \prod_{n=1}^2\F\to \prod_{n=1}^1\F\to \prod_{n=1}^0\F\right)=\prod_{n\in\N}\F,
%\end{equation*}
%which is infinite-dimensional over $\F$.
%We will consider the \emph{$d$-dimensional Hawaiian earring}
%\begin{equation*}
%\HE = \bigcup_{n \in \N} \left\{ (x_0, \dots, x_d) \in \R^{d+1} \ \middle | \ \left( x_0 - \frac{1}{n} \right)^2 \!\! + x_1^2 + \dots + x_d^2 = \left( \frac{1}{n} \right)^2 \right\},
%\end{equation*}
%which is a compact subspace of $\R^{d+1}$.
%
%\begin{figure}[t]
%	\centering
%	\begin{tikzpicture}[scale = 60]
%	\draw[thick] (-.1,-.001) rectangle (.1,.05);
%	\clip (-.1,0) rectangle (.1,.05);
%	\foreach \i in {1,...,100}{
%		\draw[line width=0.4/\i^0.25 pt] (0, 1/\i^2) circle (1/\i^2);
%	}
%	\end{tikzpicture}
%	\caption{A closeup of the Hawaiian earring $\mathbb{H}^1$.}
%\end{figure}
%
%\begin{thm} \label{t:counterexample}
%	The function $f \colon \HE \to \R$ whose value at the origin is $0$ and is $1$ everywhere else defines a compact and weakly $\LC$ sublevel set filtration that is not q-tame with respect to $\H$ if $\H_{n}(\HE)$ is infinite dimensional for some $n$.
%\end{thm}
%
%\begin{proof}
%	To verify that $f$ has compact sublevel sets we notice that all sublevel sets are either the empty set, the singleton containing the origin, or $\HE$ itself, all compact Hausdorff spaces.
%
%	In order to verify that the sublevel set filtration of $f$ is weakly $\LC$, we
%	consider $x \in \HE$, $V$ a neighborhood of $x$ in $\HE$ and $t > f(x)$. We need to find a neighborhood $U \subseteq V$ of $x$ and $s \in (f(x), t)$ such that the inclusion $f_{\leq s} \cap U \to f_{\leq t} \cap V$ is homotopic to a constant map.
%
%	If $x$ is the origin, we have $f(x) = 0$ and choose $s \in (0, \min\{t, 1\})$.
%	Then $f_{\leq s} = \{x\}$, so with $U = V$ the inclusion $f_{\leq s} \cap U \to f_{\leq t} \cap V$ is the inclusion of $\{x\}$ into $f_{\leq t} \cap V$, which is a constant map, so the weak $\LC$ condition is trivially satisfied.
%
%	For $x$ not the origin we have $f(x) = 1$ and choose $s \in (1,t)$ arbitrarily, so that $f_{\leq s} = f_{\leq t} = \HE$.
%	Note that since $x$ is not the origin, there is a unique $d$-sphere in $\HE$ that contains $x$.
%	Clearly, we may choose $\delta > 0$ so small that $B_{\delta}(x) = \{y \in \R^{d+1} \mid \Vert x - y \Vert < \delta\} \cap \HE$ is a topological ball contained in this sphere and contained in $V$.
%	The ball $B_\delta(x)$ can be contracted to $\{x\}$ in $V$, so choosing $U = B_{\delta}(x)$, we obtain that the inclusion $f_{\leq s} \cap U \to f_{\leq t} \cap V$ is homotopic to the constant map with value $x$.
%
%	It remains to be shown that $f_{\leq \bullet}$ is not q-tame for $\H$.
%	This follows directly from our assumption that $\H_{n}(\HE)$ is not finite dimensional for some $n$, as $f_{\leq t}$ is constant with value $\HE$ for $t \geq 1$.
%\end{proof}
Moreover, the singular homology of the $d$-dimensional Hawaiian earring is also infinite dimensional, as proven in \cite{Barratt.1962}. In summary, we have the following.

\begin{cor} \label{c:counterexample}
	The function $f \colon \HE \to \R$ with value~$0$ at the origin and $1$ elsewhere defines a weakly $\LC$ compact sublevel set filtration that is not q-tame with respect to singular and \v{C}ech homology.
\end{cor}

The gap we have thus highlighted in the argument of Morse and Tompkins, that the sublevel set filtration of $A_g$ is not necessarily q-tame, can be fixed by applying \cref{t:local connectedness implies q-tameness}.
This is because the proof given in \cite[p.464]{Morse.1939} for the local connectivity of the sublevel set filtration induced by $A_g$ can actually be seen to establish a stronger property described next.

\begin{defi}
	The sublevel set filtration of a function $f \colon X \to \R$ is said to be \emph{locally connected} $\LC$ if for any $x \in X$, any neighborhood $V$ of $x$ and any pair of indices $f(x) < s < t$ there is a neighborhood $U \subseteq V$ of $x$ such that the map $f_{\leq s} \cap U \to f_{\leq t} \cap V$ is homotopic to a constant map.
\end{defi}

Clearly, the filtration being $\LC$ implies that the filtration is also $\HLC$ for \v{C}ech homology with field coefficients, which is the homology theory Morse and Tompkins use.
Therefore, from \cref{t:local connectedness implies q-tameness} we can conclude that the sublevel set filtration of $A_g$ is indeed \mbox{q-tame} as needed.
This implies that the generalized Morse inequalities hold, and hence so does the Unstable Minimal Surfaces Theorem.

We have also mentioned that Morse introduced another condition that he also called local $F$-connectivity three years earlier.
It roughly corresponds to being $\piLC$ with a certain added uniformity property.
In the original it reads:
\begin{displaycquote}[p.421--422]{Morse.1937}
	The space $M$ will be said to be locally $F$-connected for the order $n$ if corresponding to $n$, an arbitrary point $p$ on $M$, and an arbitrary positive constant $e$, there exists a positive constant $\delta$ with the following property.
	For $c \geq F(p)$ any singular $n$-sphere on $F \leq c$ (the continuous image on $F \leq c$ of an ordinary $n$-sphere) on the $\delta$-neighborhood $p_{\delta}$ of $p$ is the boundary of a singular $(n + 1)$-cell on $F \leq c + e$ and on $p_e$.
\end{displaycquote}
Morse also claims in the given reference that this condition is sufficient for q-tameness, but without providing a proof.
Whether this statement is true or not is not covered by our analysis, because the $\piLC$ and $\HLC$ conditions generally do not imply each other.
We expect the quoted claim to be true, but do not investigate it further.
