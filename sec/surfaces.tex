%!TEX root = ../func_top.tex

\section{Persistent homology and functional topology} \label{s:surfaces}

Having established topological conditions for the existence of persistence diagrams associated to filtrations and corresponding generalized Morse inequalities, we now describe how these results relate to Morse's general theory of functional topology \cite{Morse.1937, Morse.1938, Morse.1940}, and to its application in Morse and Tompkins' work on unstable minimal surfaces from \cite{Morse.1939}.
We will focus on the homological aspects of this approach referring to \cite[Sections 4.3--5]{Bott.1980} for a general exposition.
For a more thorough presentation of the unstable minimal surface problem, including the analytical details, please consult \cite[Section II.6]{Struwe.1988}, and for a historical account and an overview of subsequent results see \cite[Section 6]{Dierkes.2010} and \cite[Section 6.8.1]{Dierkes.2010b}.

\subsection{The unstable minimal surface problem}

Morse and Tompkins considered the following setting introduced by Douglas.
Let $g \colon \R \to \R^n$ be a $2\pi$-periodic function representing a simple closed curve such that $g$ is differentiable with Lipschitz derivative.
Let $\widetilde{\Omega}$ be the space of continuous non-decreasing functions $\varphi \colon \R \to \R$ with $\varphi(t+2\pi) = \varphi(t) + 2\pi$ for all $t$ and $\varphi(\alpha_i)=\alpha_i$ for three fixed distinct points $\alpha_i \in [0,2\pi)$.
The \emph{Douglas functional} on $\widetilde \Omega$ associated to the curve $g$ is defined as
\begin{equation*}
A_g(\varphi) = \frac{1}{16 \pi} \int_0^{2\pi} \int_0^{2\pi} \left\| \frac{g(\varphi(\alpha)) - g(\varphi(\beta))}{\sin \frac{\alpha-\beta}{2}} \right\|_2^2 \ \mathrm{d}\alpha \ \mathrm{d}\beta.
\end{equation*}
It coincides with the Dirichlet energy of the unique harmonic extension of the reparametrized curve $g \circ \varphi$ to a parametrized surface.
The Dirichlet energy is an upper bound for the area, with equality if the parametrization is conformal.
Let $\Omega_g = \{\varphi \in \widetilde\Omega \mid A_g(\varphi) < \infty\}$, equipped with the $C^0$ metric.
The set $\Omega_g$ is non-empty and the sublevel sets of $A_g$ are compact \cite[p.~448]{Morse.1939}.
Since $A_g$ is bounded below by $0$, this implies that $A_g$ attains a global minimum.
The corresponding surface is then a solution of \emph{Plateau's Problem}, which asks for a surface homeomorphic to a disk with boundary $g$ and minimum area.

As alluded to in \cref{r:homotopically critial points}, Morse and Tompkins consider a homotopical notion of critical point for a general function $F \colon M \to \R$ on a metric space \cite[p.~445]{Morse.1939}, see also \cite{Morse.1943}.
%Roughly, a point $p$ is \emph{homotopically critical} if it has no neighborhood in $F_{\leq F(p)}$ that can be mapped by a homotopy into $F_{\leq t}$ for some $t<F(p)$.

\begin{defi}[{{\cite[Definition II.6.1-II.6.2]{Struwe.1988},\cite[p.~445, 472]{Morse.1939}}}]
	Consider a real-valued function $F$ on a metric space $(M,d)$.
	A point $p \in M$ is called \emph{homotopically regular} if there exists a neighborhood $U$ of $p$ in $F_{\leq F(p)}$ and a continuous map $\varphi \colon U \times [0,1] \to M$ satisfying $\varphi(p,1) \neq p$ such that for every compact subset $V \subseteq U$ there exists a function $\delta \colon \R_{\geq 0} \to \R_{\geq 0}$ with
	$\delta(e) = 0$ if and only if $e = 0$, and
	\[
	F(\varphi(x,s)) - F(\varphi(x,t)) > \delta(d(\varphi(x,s),\varphi(x,t)))
	\]
	for all $x \in V$ and $0 \leq s \leq t \leq 1$.
% 	\begin{itemize}
% 		\item $\delta(e) = 0$ if and only if $e = 0$, and
% 		\item $F(\varphi(x,s)) - F(\varphi(x,t)) > \delta(d(\varphi(x,s),\varphi(x,t)))$ for all $x \in V$ and $0 \leq s \leq t \leq 1$.
% 	\end{itemize}
	A point that is not homotopically regular is called \emph{homotopically critical}.
	Function values of homotopically critical points will be called \emph{critical values} and all other values will be called \emph{regular values}.
	A \emph{critical set} $S$ is a closed and open subset of the subspace of all homotopically critical points with a given function value.
	It is said to be of \emph{minimum type} if there exists a neighborhood $N$ of the closure $\overline{S}$ of $S$, taken in $M$, such that the function values on $N \setminus S$ strictly exceed the function value on $S$.
\end{defi}

Note that, in particular, an isolated local minimum constitutes a critical set of minimum type.
Morse and Tompkins state the following result with slightly different assumptions, as a general Mountain Pass Lemma for critical sets.

% \begin{thm}[{\cite[Corollary 7.1, p.~472]{Morse.1939}}]
% \label{thm:mountain_pass}
% 	Let $M$ be a metric space and $F \colon M \to \R$ a function.
% 	Assume that $F$ is bounded below, that the sublevel set filtration of $F$ is compact, and let $\H$ be an additive homology theory taking non-zero values on non-empty sets in dimension 0, and such that $\H(F_{\leq \bullet})$ is q-tame, continuous from above and has essential dimension $p_{0} = 1$.
% 	If $M$ contains two distinct critical sets of $F$ of minimum type, then it also contains a critical set not of minimum type.\todo{sort of difficult to parse, assumptions too long.
%	maybe only formulate for \v{C}ech homology to get rid of some text?}
% \end{thm}

\begin{thm}[{\cite[Corollary 7.1, p.~472]{Morse.1939}}]
\label{thm:mountain_pass}
	Let $M$ be a metric space and $F \colon M \to \R$ a function.
	Assume that $F$ is bounded below, that the sublevel set filtration of $F$ is compact, and that $\CH(F_{\leq \bullet})$ is q-tame with essential dimension $\cp_{0} = 1$, where $\CH$ denotes \v{C}ech homology.
	If $M$ contains two distinct critical sets of $F$ of minimum type, then it also contains a critical set not of minimum type.
\end{thm}

\begin{rem}
	We mention that more general homology theories can be considered in this theorem.
	The precise hypothesis on the homology theory $H$ are that it is additive, taking non-zero values on non-empty sets in dimension $0$, and such that $\H(F_{\leq \bullet})$ is continuous from above.
	\todo{Where is continuous from above introduced? Maybe put a reference here.}
	% Recall that $\H$ is called \emph{additive} if $\H (A \sqcup B) \cong \H(A) \oplus \H(B)$ for all $A$ and $B$.
\end{rem}

An important case where the hypotheses of \cref{thm:mountain_pass} are satisfied is when $M$ is contractible (which ensures that $\check{p}_0 = 1$ by \cref{lem:essential_cech_dim}), and the sublevel set filtration is $\HLC$ with respect to $\CH$ (which ensures the q-tameness hypothesis by \cref{t:local connectedness implies q-tameness}).

Morse and Tompkins aim to prove that the Douglas functional satisfies the assumptions of the \cref{thm:mountain_pass} (we will elaborate on this at the end of this section and in \cref{subsec:historic_hlc}) and they show \cite[Theorem 6.2]{Morse.1939} that each homotopically critical point of the Douglas functional $A_g$ indeed corresponds to
%
%a critical point of the area functional,
%called
a \emph{minimal surface} -- a surface with vanishing mean curvature -- and use this correspondence to deduce the following result, also reviewed in \cite[Theorem II.6.10]{Struwe.1988}.

\begin{cor}[{Unstable Minimal Surface Theorem \cite[Corollary 7.1, p.~472]{Morse.1939}}]
\label{thm:unstable_minimial_surface}
	If the space~$\Omega_g$ contains two minimal surfaces contained in distinct critical sets of minimum type of the functional $A_g$, then it also contains an \emph{unstable} minimal surface, i.e., a minimal surface contained in a critical set that is not of minimum type.
\end{cor}

In \cite[Section 8]{Morse.1939}, Morse and Tompkins provide an example of a curve $g$ for which $\Omega_{g}$ indeed contains two distinct critical sets of minimum type, so that $g$ then also spans an unstable minimal surface.

\begin{rem}
	In their definition of a critical set $S$ of \emph{minimum type} \cite[p.~472]{Morse.1939}, Morse and Tompkins do not require the neighborhood $N$ of $S$ on which the function values exceed those on $S$ to contain the closure of $S$.
	However, without this additional assumption, \cref{thm:mountain_pass} does not hold, as the example given by $f \colon [0,1] \to \R$ with $f(0) = f(1) = 0$ and $f(t) = 1$ for $0 < t < 1$ shows:
	$f$ has the four critical sets $\{0\}$, $\{1\}$, $\{0\} \cup \{1\}$ and $(0,1)$, which all satisfy the minimum type condition if the neighborhood $N$ need not contain their closure, but then there is no critical set that is not of minimum type.
\end{rem}

While more efficient and more general proofs for the existence of an unstable minimal surface (with respect to more natural topologies than $C^0$) have subsequently been established \cite{Struwe.1988,Dierkes.2010}, including less restrictive assumptions on the boundary curve, the original approach of Morse and Tompkins is notable
% for its close connection to persistent homology.
for its connections to other areas of mathematics.
As an illustration, we will now sketch a proof of \cref{thm:mountain_pass} using the previously developed machinery, starting with some intermediate results.

\begin{lem}
\label{lem:endpoint_implies_crit_pt}
	Let $F \colon M \to \R$ be a function on a metric space with compact sublevel set filtration.
	Assume that $\H(F_{\leq \bullet})$ is q-tame and continuous from above and consider $t \in \R$ and its cap numbers $c^{\epsilon}(t)$.
	If there exists $\epsilon > 0$ such that $c^{\epsilon}(t) > 0$, then $t$ is a critical value.
\end{lem}
\begin{proof}
	Following \cite[Remark II.6.3]{Struwe.1988}, we know that if $t$ is a regular value, there exists $\epsilon > 0$ such that the inclusion $F_{\leq s} \to F_{\leq t}$ is a homotopy equivalence for all $s \in [t - \epsilon, t]$.
	\todo{not quite what struwe says, check whether this is actually true}
	Thus, $\H(F_{\leq \bullet})$ is continuous from below at every regular value.
	However, we assume $\H(F_{\leq \bullet})$ to be also continuous from above at every value, and in particular at regular values.
	Hence, \cref{cor:regular_value_no_endpoint} implies that $\alpha(t) = \omega(t) = 0$ whenever $t$ is regular, which proves the claim.
\end{proof}

%\begin{lem}
%\label{lem:minimum_type_implies_summand}
%	Let $F \colon M \to \R$ be a function on a metric space and let $S$ be a critical set with value $t$.
%	If $S$ is of minimum type, then it is a topological summand of $F_{\leq t}$ in the sense that $F_{\leq t}$ is homeomorphic to the disjoint union $S \sqcup (F_{\leq t} \setminus S)$.
%\end{lem}
%\begin{proof}
%	By definition, there exists a neighborhood $N$ of $\overline{S}$ in $M$ such that the function values of $F$ on $N \setminus S$ exceed $t$.
%	In particular, we have $F_{\leq t} \cap N = S$, showing that $S$ is open in $F_{\leq t}$.
%	Because $N$ contains $\overline{S}$, we also obtain $F_{\leq t} \cap \overline{S} = S$, showing that $S$ is closed in $F_{\leq t}$.
%	This proves the claim.
%\end{proof}

%In particular, \cref{lem:minimum_type_implies_summand} implies that a critical set of minimum type can only give birth to new features in the persistence diagram and not kill existing ones, provided that the homology theory being used is additive.

\begin{lem}
\label{lem:minimum_type_implies_some_birth_and_no_death}
	Let $F \colon M \to \R$ be a function on a metric space with compact sublevel set filtration and let $S$ be a critical set of minimum type with value $t$.
	Assume that $\H$ is additive and that $\H(F_{\leq \bullet})$ is q-tame and continuous from above.
	\begin{enumerate}
		\item We have $\alpha(t) \geq \dim \H(S)$ for the number of births at $t$.
		\item If there are not homotopically critical points with value $t$ outside $S$, then we have $\omega(t) = 0$ for the number of deaths at $t$.
	\end{enumerate}
\end{lem}
\begin{proof}
	We start by showing that $S$ is a topological summand of $F_{\leq t}$ in the sense that $F_{\leq t}$ is homeomorphic to the disjoint union $S \sqcup (F_{\leq t} \setminus S)$.
	It suffices to show that $S$ is open and closed in $F_{\leq t}$.
	By definition, there exists a neighborhood $N$ of $\overline{S}$ in $M$ such that the function values of $F$ on $N \setminus S$ exceed $t$.
	In particular, we have $F_{\leq t} \cap N = S$, showing that $S$ is open in $F_{\leq t}$.
	Because $N$ contains $\overline{S}$, we also obtain $F_{\leq t} \cap \overline{S} = S$, showing that $S$ is closed in $F_{\leq t}$.

	Using additivity of $\H$ and \cref{lem:birth_death_formulas}, we now obtain
	\begin{align*}
		\alpha(t) &= \dim \coker \big( \colim_{s < t} \H(F_{\leq s}) \to \lim_{u > t} \H(F_{\leq u}) \big) \\
			&= \dim \coker \big( \colim_{s < t} \H(F_{\leq s}) \to \H(F_{\leq t}) \big) \\
			&= \dim \coker \big( \colim_{s < t} \H(F_{\leq s}) \to \H(F_{\leq t} \setminus S) \oplus \H(S) \big) \\
			&\geq \dim \H(S),
	\end{align*}
	where we have used the assumption that $\H(F_{\leq \bullet})$ is continuous from above for the second equality and the fact that $F_{\leq s} \subseteq (F_{\leq t} \setminus S)$ for all $s < t$ for the final inequality.

	Now, assuming that $S$ is the set of all homotopically critical points with value $t$ implies that the inclusion $F_{\leq s} \to (F_{\leq t} \setminus S)$ is in fact a homotopy equivalence for $s < t$, again following \cite[Remark II.6.3]{Struwe.1988}.\todo{again, needs to be reviewed whether this actually follows from what struwe says}
	Hence, again using additivity of $\H$, continuity from above, and \cref{lem:birth_death_formulas}, we obtain that
	\begin{align*}
		\omega(t) &= \dim \ker \big( \colim_{s < t} \H(F_{\leq s}) \to \lim_{u > t} \H(F_{\leq u}) \big) \\
				&= \dim \ker \big( \colim_{s < t} \H(F_{\leq s}) \to \H(F_{\leq t}) \big) \\
				&= \dim \ker \big( \colim_{s < t} \H(F_{\leq s}) \to \colim_{s < t} H(F_{\leq s}) \oplus \H(S) \big) \\
				&= 0
	\end{align*}
	as claimed.
\end{proof}

We are now ready to give a proof of the Mountain Pass Lemma.
\begin{proof}[Proof of \cref{thm:mountain_pass}]
	We write $\cc_d$, $\calpha_d$, $\comega_d$, and $\cp_d$ for the cap numbers, births, deaths, and essential dimensions of $\CH_d(F_{\leq \bullet})$, respectively.
	Assume that $F$ has two distinct critical sets $S_{1}$ and $S_{2}$ of minimum type with values $t_{1}$ and $t_{2}$, respectively.
	Since both critical sets are non-empty, we have $\CH_{0}(S_{i}) \neq 0$, and thus the first assertion of \cref{lem:minimum_type_implies_some_birth_and_no_death} implies $\calpha_{0}(t_{i}) \geq \dim \CH_0(S_{i}) \geq 1$ for $i = 1,2$.
	This implies that there exists $\epsilon > 0$ with $\cc_{0}^{\epsilon} \geq \calpha_{0}(t_{1}) + \calpha_{0}(t_{2}) \geq 1 + 1 = 2$.
	However, we assume $\cp_{0} = 1$, so with notation as in the proof of \cref{t:inequalities}, we obtain $\comega_{0}^{\epsilon} = \cc_{0}^{\epsilon} - \cp_{0} \geq 2 - 1 = 1$.
	Thus, there must be some $t \in \R$ with $\comega_{0}(t) > 0$, so that in particular $\cc_{0}^{\epsilon}(t) > 0$.
	Next, we apply \cref{lem:endpoint_implies_crit_pt} and obtain that the set $S$ of homotopically critical points at value $t$ is non-empty.
	If $S$ were of minimum type, then we would have $\comega_{0}(t) = 0$ by the second assertion of \cref{lem:minimum_type_implies_some_birth_and_no_death}, contradicting the choice of $t$.
	Hence, $S$ cannot be of minimum type, which finishes the proof.
\end{proof}

\todo{should we remark on the fact that the morse inequalities only yield $c_{1}^{\epsilon} \geq 1$, but that this is not enough to conclude existence of something that is not of minimum type? $\omega(t) \geq 1$ is needed}

What remains to be discussed in order to obtain the Unstable Minimal Surface Theorem of Morse and Tompkins (\cref{thm:unstable_minimial_surface}) as a consequence of their Mountain Pass Lemma, is why the Douglas functional $A_{g}$ satisfies the hypothesis of \cref{thm:mountain_pass}.
We have already mentioned that $A_{g} \colon \Omega_{g} \to \R$ is bounded below and has compact sublevel sets.
% One can use \v{C}ech homology
% , which satisfies the assumptions on the homology theory in \cref{thm:mountain_pass} (see \cite{Milnor.1962} for additivity),
% to obtain a sublevel set persistence module that is continuous from above.
We can also check that the essential dimension $\check{p}_{0}$ is $1$ using the following lemma, because $\Omega_{g}$ is contractible by \cite[Theorem 4.3]{Morse.1939}.

\begin{lem}
\label{lem:essential_cech_dim}
	Let $F \colon M \to \R$ be a function on a non-empty metric space with compact sublevel set filtration and let $\cp_{d}$ be the essential dimensions of its q-tame persistent \v{C}ech homology.
	If $M$ is contractible, then $\cp_{0} = 1$.
\end{lem}
\begin{proof}
	Let $\H$ denote singular homology, let $p_{d}$ denote the essential dimensions of the persistent singular homology of $F_{\leq \bullet}$, and let $\CH$ denote \v{C}ech homology.
	Since $M$ is contractible, we have $\dim H_{0}(M) = 1$.
	We also have $p_{d} = \dim \colim_{t \to \infty} H_{d}(F_{\leq t}) = \dim H_{d}(M)$, where the second equality holds because all the sublevel sets of $F$ are compact \cite[Proposition 3.33]{Hatcher.2002}.
	Now, the natural map from singular to \v{C}ech homology is always surjective for compact metric spaces in dimension 0 \cite{Eda.2000}, so $\cp_{0} \leq p_{0}$.
	Moreover, we clearly have $\cp_{0} \geq 1$ because $M$ is non-empty.
	%Map from some non-empty sublevel set to $M$ has to have at least rank 1 and has ti factor through colimit, so $\cp_{0} = \dim \colim \geq 1$.
	In total, we obtain $1 \leq \cp_{0} \leq p_{0} = \dim H_{0}(M) = 1$, which proves the claim.
\end{proof}

The only assumption left to verify is q-tameness.

\subsection{Morse's local connectivity conditions}\label{subsec:historic_hlc}
%Parts of the framework of functional topology that Morse and Tompkins use is developed by Morse in \cite{Morse.1940} in a very general setting.
%In particular, he proves inequalities for cap numbers associated to the persistent \v{C}ech homology of the sublevel set filtration, as also considered in \cref{s:inequalities}.
%From these generalized Morse inequalities, the existence of an unstable minimal surface can easily be deduced in the presence of two distinct critical sets of minimum type.

%We have seen that these inequalities require q-tameness, so in order to apply them in the minimal surface setting % and deduce their theorem,
%one needs to prove the \mbox{q-tameness} of the sublevel set filtration of the Douglas functional $A_g$.


% What remains to be discussed is why the Douglas functional $A_{g} \colon \Omega_{g} \to \R$ satisfies the assumptions of \cref{thm:mountain_pass} so that the Unstable Minimal Surface Theorem follows as a corollary.

Throughout his work on functional topology, in order to obtain \mbox{q-tameness}, Morse assumed slightly varying forms of local connectivity on the resulting sublevel set filtrations.
In particular, Morse and Tompkins used the following condition in their applications to minimal surface theory:
\begin{displaycquote}[p.~431]{Morse.1940}
%\cite[p.~ 25]{Morse.1938}
	Let $p$ be a point of $M$ at which $F(p)=c$.
	The space $M$ is said to be \emph{locally $F$-connected} of order $r$ at~$p$ if corresponding to each positive constant $e$ there exists a positive constant $\delta$ such that each singular $r$-sphere on the $\delta$-neighborhood of $p$ and on $F_{c+\delta}$ bounds an $(r+1)$-cell of norm $e$ on $F_{c+e}$.
\end{displaycquote}
See also \cite[p.~ 25]{Morse.1938} and \cite[p.~464]{Morse.1939}, but note that the definitions given there contain typographical errors.
Using similar language to the one used in \cref{s:connectivity}, the property of local $F$-connectedness of all orders is equivalent to the following notion applicable to general topological spaces.

\begin{defi}
	The sublevel set filtration of a function $f \colon X \to \R$ is said to be \emph{weakly locally connected of all orders}, or \emph{weakly $\piLC$}\todo{check whether $\piLC$ is still appropriate}, if for any $x \in X$, $V$ a neighborhood of $x$, and any index $t > f(x)$, there is an index $s$ with $f(x) < s < t$ and a neighborhood $U$ of $x$ with $U \subseteq V$ such that the inclusion $f_{\leq s} \cap U \to f_{\leq t} \cap V$ induces trivial maps on homotopy groups.
\end{defi}

Morse then goes on to claim that the persistent \v{C}ech homology of this sublevel set filtration is q-tame, provided that $F$ is bounded from below and satisfies the assumptions of local $F$-connectivity and compactness of sublevel sets.
%, and a further condition called $F$-regularity, which is trivially satisfied if the domain of $F$ is compact.
In the original (where the function is assumed to take values in $[0,1)$) the claim reads:
\begin{displaycquote}[Theorem 6.3, p.~432]{Morse.1940}
	Let $a$ and $c$ be positive constants such that $a < c < 1$.
	The $k^{\mathrm{th}}$ connectivity $R^k(a,c)$ of $F_a$ on $F_c$ is finite.
\end{displaycquote}
Morse does not prove this statement in the given reference, but rather refers to \cite[Theorem~6.1]{Morse.1938}.
Unfortunately, the above claim does not hold in general, as exemplified by the sublevel set filtration from \cref{e:counterexample}.
To elaborate on this, we consider a stronger version of weak local connectedness.

\begin{defi}
	The sublevel set filtration of a function $f \colon X \to \R$ is said to be \emph{weakly locally contractible}, or \emph{weakly $\LC$}\todo{check whether $\LC$ is still appropriate}, if for any $x \in X$, $V$ a neighborhood of $x$, and any index $t > f(x)$, there is an index $s$ with $f(x) < s < t$ and a neighborhood $U$ of $x$ with $U \subseteq V$ such that the inclusion $f_{\leq s} \cap U \to f_{\leq t} \cap V$ is homotopic to a constant map.
\end{defi}

Clearly, being weakly $\LC$ implies being weakly $\piLC$ and, if the homology $\H$ takes finite dimensional values on one-point spaces, also weakly $\HLC$.
Observe that \cref{e:counterexample} actually establishes that the filtration given there is weakly $\LC$, so not even the weak $\LC$ condition is sufficient to ensure the q-tameness of compact sublevel set filtrations that are induced by non-continuous functions in general.
In particular, our construction invalidates Morse's claim quoted above because \v{C}ech homology satisfies the assumptions on the homology theory made in \cref{e:counterexample}.

Specifically, using the fact that \v{C}ech homology of compact Hausdorff spaces commutes with inverse limits, it is straightforward to verify that the \v{C}ech homology in degree $d$ of the $d$-dimensional Hawaiian earring is isomorphic to $\prod_{n\in\N}\F$, which is infinite dimensional over $\F$.
%To see that his is the case, one can use the fact that \v{C}ech homology commutes with totally ordered limits for compact Hausdorff spaces \cite[Theorems VIII.3.6 and X.3.1]{Eilenberg.1952} as follows.
%Define
%\begin{align*}
%\HE_k &=
%\left\{ (x_0, \dots, x_d) \in \R^{d+1} \ \middle| \ \left( x_0 - \frac{1}{k} \right)^2 + x_1^2 + \dots + x_d^2 \leq \left( \frac{1}{k} \right)^2 \right\} \\ &\, \cup
%\bigcup_{n=1}^{k-1} \left\{ (x_0, \dots, x_d) \in \R^{d+1} \ \middle |\ \left( x_0 - \frac{1}{n} \right)^2 + x_1^2 + \dots + x_d^2 = \left( \frac{1}{n} \right)^2 \right\},
%\end{align*}
%i.e., the $d$-dimensional Hawaiian earring but with the $k$-th largest $d$-sphere filled.
%We have $\lim_{k} \HE_{k} = \bigcap_{k} \HE_{k} = \HE$, and hence $\CH_{d}(\HE; \F) = \lim_{k} \CH_{d}(\HE_{k}; \F)$, where $\CH$ denotes \v{C}ech homology.
%Clearly, each $\HE_{k}$ is a CW-complex, so we can simply use cellular homology to compute
%\begin{equation*}
%\lim_{k}\CH_{d}(\mathbb{H}^{d}_{k}; \F)=\lim\left(\dots\to \prod_{n=1}^2\F\to \prod_{n=1}^1\F\to \prod_{n=1}^0\F\right)=\prod_{n\in\N}\F,
%\end{equation*}
%which is infinite-dimensional over $\F$.
%We will consider the \emph{$d$-dimensional Hawaiian earring}
%\begin{equation*}
%\HE = \bigcup_{n \in \N} \left\{ (x_0, \dots, x_d) \in \R^{d+1} \ \middle | \ \left( x_0 - \frac{1}{n} \right)^2 \!\! + x_1^2 + \dots + x_d^2 = \left( \frac{1}{n} \right)^2 \right\},
%\end{equation*}
%which is a compact subspace of $\R^{d+1}$.
%
%\begin{figure}[t]
%	\centering
%	\begin{tikzpicture}[scale = 60]
%	\draw[thick] (-.1,-.001) rectangle (.1,.05);
%	\clip (-.1,0) rectangle (.1,.05);
%	\foreach \i in {1,...,100}{
%		\draw[line width=0.4/\i^0.25 pt] (0, 1/\i^2) circle (1/\i^2);
%	}
%	\end{tikzpicture}
%	\caption{A closeup of the Hawaiian earring $\mathbb{H}^1$.}
%\end{figure}
%
%\begin{thm} \label{t:counterexample}
%	The function $f \colon \HE \to \R$ whose value at the origin is $0$ and is $1$ everywhere else defines a compact and weakly $\LC$ sublevel set filtration that is not q-tame with respect to $\H$ if $\H_{n}(\HE)$ is infinite dimensional for some $n$.
%\end{thm}
%
%\begin{proof}
%	To verify that $f$ has compact sublevel sets we notice that all sublevel sets are either the empty set, the singleton containing the origin, or $\HE$ itself, all compact Hausdorff spaces.
%
%	In order to verify that the sublevel set filtration of $f$ is weakly $\LC$, we
%	consider $x \in \HE$, $V$ a neighborhood of $x$ in $\HE$ and $t > f(x)$.
%We need to find a neighborhood $U \subseteq V$ of $x$ and $s \in (f(x), t)$ such that the inclusion $f_{\leq s} \cap U \to f_{\leq t} \cap V$ is homotopic to a constant map.
%
%	If $x$ is the origin, we have $f(x) = 0$ and choose $s \in (0, \min\{t, 1\})$.
%	Then $f_{\leq s} = \{x\}$, so with $U = V$ the inclusion $f_{\leq s} \cap U \to f_{\leq t} \cap V$ is the inclusion of $\{x\}$ into $f_{\leq t} \cap V$, which is a constant map, so the weak $\LC$ condition is trivially satisfied.
%
%	For $x$ not the origin we have $f(x) = 1$ and choose $s \in (1,t)$ arbitrarily, so that $f_{\leq s} = f_{\leq t} = \HE$.
%	Note that since $x$ is not the origin, there is a unique $d$-sphere in $\HE$ that contains $x$.
%	Clearly, we may choose $\delta > 0$ so small that $B_{\delta}(x) = \{y \in \R^{d+1} \mid \Vert x - y \Vert < \delta\} \cap \HE$ is a topological ball contained in this sphere and contained in $V$.
%	The ball $B_\delta(x)$ can be contracted to $\{x\}$ in $V$, so choosing $U = B_{\delta}(x)$, we obtain that the inclusion $f_{\leq s} \cap U \to f_{\leq t} \cap V$ is homotopic to the constant map with value $x$.
%
%	It remains to be shown that $f_{\leq \bullet}$ is not q-tame for $\H$.
%	This follows directly from our assumption that $\H_{n}(\HE)$ is not finite dimensional for some $n$, as $f_{\leq t}$ is constant with value $\HE$ for $t \geq 1$.
%\end{proof}
Moreover, the singular homology of the $d$-dimensional Hawaiian earring is also infinite dimensional, as proven in \cite{Barratt.1962}.
In summary, we have the following.

\begin{cor} \label{c:counterexample}
	The function $f \colon \HE \to \R$ with value~$0$ at the origin and $1$ elsewhere defines a weakly $\LC$ compact sublevel set filtration that is not q-tame with respect to singular and \v{C}ech homology.
\end{cor}

The gap we have thus highlighted in the argument of Morse and Tompkins, that the sublevel set filtration of $A_g$ is not necessarily q-tame, can be fixed by applying \cref{t:local connectedness implies q-tameness}.
This is because the proof given in \cite[p.464]{Morse.1939} for the local connectivity of the sublevel set filtration induced by $A_g$ can actually be seen to establish a stronger property described next.

\begin{defi}
	The sublevel set filtration of a function $f \colon X \to \R$ is said to be \emph{locally contractible} or \emph{$\LC$} if for any $x \in X$, any neighborhood $V$ of $x$ and any pair of indices $f(x) < s < t$ there is a neighborhood $U \subseteq V$ of $x$ such that the map $f_{\leq s} \cap U \to f_{\leq t} \cap V$ is homotopic to a constant map.
\end{defi}

Clearly, the filtration being $\LC$ implies that the filtration is also $\HLC$ for \v{C}ech homology with field coefficients, which is the homology theory Morse and Tompkins use.
Therefore, from \cref{t:local connectedness implies q-tameness} we can conclude that the sublevel set filtration of $A_g$ is indeed \mbox{q-tame} as needed.
This implies that \cref{thm:mountain_pass} indeed applies to the Douglas functional, and hence the Unstable Minimal Surface Theorem holds.

We have also mentioned that Morse introduced another condition that he also called local $F$-connectivity three years earlier.
It roughly corresponds to being $\piLC$ with a certain added uniformity property.
In the original it reads:
\begin{displaycquote}[p.421--422]{Morse.1937}
	The space $M$ will be said to be locally $F$-connected for the order $n$ if corresponding to $n$, an arbitrary point $p$ on $M$, and an arbitrary positive constant $e$, there exists a positive constant $\delta$ with the following property.
	For $c \geq F(p)$ any singular $n$-sphere on $F \leq c$ (the continuous image on $F \leq c$ of an ordinary $n$-sphere) on the $\delta$-neighborhood $p_{\delta}$ of $p$ is the boundary of a singular $(n + 1)$-cell on $F \leq c + e$ and on $p_e$.
\end{displaycquote}
Morse also claims in the given reference that this condition is sufficient for q-tameness, but without providing a proof.
Whether this statement is true or not is not covered by our analysis, because the $\piLC$ and $\HLC$ conditions generally do not imply each other.
We expect the quoted claim to be true, but do not investigate it further.
