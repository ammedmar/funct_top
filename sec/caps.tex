%!TEX root = ../func_top.tex

\section{Regular values and cap numbers} \label{s:caps}
We will now discuss the equivalence of our definition of cap numbers and the one used by Morse.
Futhermore, we provide a proof of \cref{thm:sublevel_set_homotopy} establishing a homotopy equivalence of sublevel sets at a regular value.
As a starting point, we observe that the basic use of \cref{thm:sublevel_set_homotopy} is to show that if $t$ is an endpoint of a feature in the persistence diagram associated to our function $F$, i.e., if $c(t) > 0$, then there must be a homotopically critical point with value $t$.
To achieve this, we assume that $F$ is weakly upper-reducible and its sublevel set filtration is compact.
A very similar statement appears in \cite[Theorem 8.1]{Morse.1938}, which says that if $t$ is a \emph{cap limit}, then there must be a homotopically critical point with value $t$.
There are two slight differences to our approach:
First, in \cite[Theorem 8.1]{Morse.1938} it is assumed that $F$ is not only weakly upper-reducible but upper-reducible.
Second, $t$ is not required to satisfy $c(t) > 0$, but required to be a \emph{cap limit}, meaning that $\colim_{s: s < t} \CH(F_{\leq t}, F_{\leq s}) \neq 0$ \cite[p.~12]{Morse.1938}.
Regarding the second point, it turns out that the two requirements are equivalent.

\begin{prop}\label{prop:cap_limits}
	Let $F \colon M \to \R$ be a function on a metric space with compact sublevel set filtration and fix $t \in \R$.
	Assume that $\H$ has long exact sequences for pairs of compact spaces, and that $\H_d(F_{\leq \bullet})$ is q-tame and continuous from above.
	Then $\dim \colim_{s: s < t} \H_d(F_{\leq t}, F_{\leq s}) = c_d(t)$ or both quantities are infinite.
\end{prop}
\begin{proof}
	First, note that we have a splitting
	\begin{align*}
	\colim_{s: s < t} \H_d(F_{\leq t}, F_{\leq s})
	&\cong \im \left( \H_d(F_{\leq t}) \to \colim_{s: s < t} \H_d(F_{\leq t}, F_{\leq s}) \right) \\
	&\oplus \coker \left( \H_d(F_{\leq t}) \to \colim_{s: s < t} \H_d(F_{\leq t}, F_{\leq s}) \right).
	\end{align*}
	Computing with the long exact sequences of pairs in homology, we have
	\begin{align*}
		\alpha_d(t)
		&= \dim \coker \left( \colim_{s: s < t} \H_d(F_{\leq s} \hookrightarrow F_{\leq t}) \right) \\
		&= \dim \colim_{s: s < t} \coker \left( \H_d(F_{\leq s} \hookrightarrow F_{\leq t}) \right) \\
		&= \dim \colim_{s: s < t} \im \left( \H_d(F_{\leq t}) \to \H_d(F_{\leq t}, F_{\leq s}) \right) \\
		&= \dim \im \left( \H_d(F_{\leq t}) \to \colim_{s: s < t} \H_d(F_{\leq t}, F_{\leq s}) \right), \\
	\end{align*}
	and
	\begin{align*}
		\omega_{d-1}(t)
		&= \dim \ker \left( \colim_{s: s < t} \H_{d-1}(F_{\leq s} \hookrightarrow F_{\leq t}) \right) \\
		&= \dim \colim_{s: s < t} \ker \left( \H_{d-1}(F_{\leq s} \hookrightarrow F_{\leq t}) \right) \\
		&= \dim \colim_{s: s < t} \coker \left( \H_d(F_{\leq t}) \to \H_d(F_{\leq t}, F_{\leq s}) \right) \\
		&= \dim \coker \left( \H_d(F_{\leq t}) \to \colim_{s: s < t} \H_d(F_{\leq t}, F_{\leq s}) \right), \\
	\end{align*}
	where we have in both cases used the fact that taking filtered colimits of vector spaces is exact, and where the first equalities hold in case both sides of the equation are finite because of \cref{lem:birth_death_formulas} and the fact that $\H(F_{\leq \bullet})$ is continuous from above.
	Since $c_d(t) = \alpha_d(t) + \omega_{d-1}(t)$, this finishes the proof.
\end{proof}

\begin{rem}
	Note that the assumptions of \cref{prop:cap_limits} are in particular satisfied for \v{C}ech homology with field coefficients and a function $F$ whose sublevel set filtration is $\HLC$ with respect to this homology theory.

	\cref{prop:cap_limits} only references cap limits, i.e., levels $t$ for which the \emph{cap space} $\colim_{s: s < t} \H_{d}(F_{\leq t}, F_{\leq s})$ is non-zero.
	However, Morse not only considers these cap spaces on their own, but he also defines the \emph{span} of a cap, and considers the space of caps with span greater than some $\epsilon > 0$, see for example \cite[Section 11]{Morse.1940}.
	Using slight generalizations of the arguments proving \cref{prop:cap_limits,lem:birth_death_formulas}, one can show that the dimensions of these spaces agree with $c_{d}^{\epsilon}(t)$.
	We already mentioned this when talking about Morse inequalities, which Morse originally formulated for dimensions of such cap spaces.
\end{rem}

We now discuss the difference between our approach and the one from \cite[Theorem 8.1]{Morse.1938} concerning the assumption of upper-reducibility.
We will show that our weakened assumptions are sufficient for the conclusion, giving an outline of the arguments leading up to \cite[Theorem 8.1]{Morse.1938} invoking only weak upper-reducibility.
As mentioned previously, similar considerations can be found in \cite[Remark II.6.3]{Struwe.1988} (although the role of weak upper-reducibility is not addressed there).
\begin{proof}[Proof of \cref{thm:sublevel_set_homotopy}]
	Let $t \in \R$ be a regular value of $F$.
	For each $p \in F_{\leq t}$ we can find a number $\delta_{p} > 0$ and a continuous map $\varphi_{p} \colon B_{\delta_{p}}(p) \times [0,1] \to M$, where $B_{\delta_{p}}(p)$ is the open metric ball around $p$ in $F_{\leq t}$ with radius $\delta_{p}$, such that
	\begin{enumerate}[label={(\arabic*)}]
		\item \label{item:id} $\varphi_{p}(\cdot,0) = \operatorname{id}$,
		\item \label{item:epsilon} for all compact $V \subseteq B_{\delta_{p}}(p)$ there exists $\epsilon(p,V) > 0$ with
		\begin{enumerate}[label={(\alph*)}]
			%\item $F(\varphi_{p}(x,r))$ is non-increasing in $r$ for all $x \in B_{\delta_{p}}(p)$, and
			\item \label{item:s_below_s} $\varphi_{p}(V \cap F_{\leq s}\times [0,1]) \subseteq F_{\leq s}$ for all $s \in [t - \epsilon(p,V), t]$,
			\item \label{item:contracting} $\varphi_{p}(V,1) \subseteq F_{\leq t - \epsilon(p,V)}$, and
			\item \label{item:displacement} there exists a displacement function for $\varphi_{p}$ and $F$ on $V \cap F_{\geq t - \epsilon(p,V)}$.
		\end{enumerate}
	\end{enumerate}
	For $p$ with $F(p) < t$ the existence of such $\delta_{p}$ and $\varphi_{p}$ is guaranteed by the assumption that $F$ is weakly upper-reducible.
	For $p$ with $F(p) = t$ it follows from the assumption that $t$ is a regular value and hence $p$ is homotopically regular, by the following argument.

	If $p$ is homotopically regular, there exists a neighborhood $U$ of $p$ in $F_{\leq t}$ and a continuous map $\varphi \colon U \times [0,1] \to M$ such that $\varphi(\cdot,0) = \operatorname{id}_{U}$, $\varphi(p,1) \neq p$, and such that on every compact set $V \subseteq U$ there exists a displacement function for $\varphi$ and~$F$.
	It is clear that $\varphi(V \cap F_{\leq s} \times [0,1]) \subseteq F_{\leq s}$ for all $s \leq t$ and $V \subseteq U$ compact since there exists a displacement function and hence the values of $F$ can not increase along trajectories of $\varphi$.
	It remains to be checked that we can shrink $U$ enough such that for all compact $V \subseteq U$ there is $\epsilon(p,V)$ with $\varphi(V,1) \subseteq F_{\leq t - \epsilon(p,V)}$.
	Assume for a contradiction that for all neighborhoods $W \subseteq U$ there exists a compact $V \subseteq W$ such that for all $\epsilon > 0$ there is some $q \in V$ with $t \geq F(q) \geq F(\varphi(q,1)) > t - \epsilon$, where we note that $F(q) \geq F(\varphi(q,1))$ is a consequence of the existence of a displacement function on~$V$.
	It follows that there exists a sequence $(q_{n})_{n}$ in $F_{\leq t}$ with $F(q_{n}) \to t$, $F(\varphi(q_{n},1)) \to t$, and $q_{n} \to p$ for $n \to \infty$.
	Since $\varphi$ is continuous we have that $\varphi(q_{n},1)$ converges to $\varphi(p,1)$, which is different from $p$ by the choice of $\varphi$, so there can be no infinite subsequence of $(q_{n})_{n}$ consisting of fixed points for $\varphi(\cdot,1)$.
	Without loss of generality, we can thus assume that $\varphi(q_{n},1) \neq q_{n}$ for all $n$.
	Now consider the compact set $K = \{q_{n} \mid n \in \N \} \cup \{p\}$.
	By assumption, we can choose a continuous displacement function $\delta$ on $K$.
	Since $K$ is compact, $\varphi$ is continuous, %$\delta$ is continuous,
	and $\varphi(\cdot,1)$ has no fixed points in $K$, the function $x \mapsto %\delta(
	d(x,\varphi(x,1))
	%)
	$ attains a minimum $\epsilon > 0$ on $K$.
	We get $t - F(\varphi(x,1)) \geq F(x) - F(\varphi(x,1)) > \delta(\frac{\epsilon}{2}) > 0$ for all $x \in K$, which contradicts the fact that $F(\varphi(q_{n},1)) \to t$ for $n \to \infty$.
	This proves the claim.


	Now, for any $p \in F_{\leq t}$, we extend the homotopy $\varphi_{p}$ from $B_{\delta_{p}}(p)$ to all of $F_{\leq t}$, not changing them on $B_{\frac{\delta_{p}}{3}}(p)$, and being the identity on $F_{\leq t} \setminus B_{\delta_{p}}(p)$:
	Let $\psi \colon [0,1] \to [0,1]$ be the continuous map given by
	\[
		\psi(x) =
		\begin{cases}
			1 & \text{if } x \in [0,\frac{1}{3}), \\
			-3x + 2 & \text{if } x \in [\frac{1}{3}, \frac{2}{3}), \\
			0 & \text{if } x \in [\frac{2}{3},1].
		\end{cases}
	\]
	We define $\tilde{\varphi}_{p} \colon F_{\leq t} \times [0,1] \to F_{\leq t}$ by
	\[
		\tilde{\varphi}_{p}(x,r) =
		\begin{cases}
			\varphi_{p} \left( x, \psi \left( \frac{d(x,p)}{\delta_{p}} \right) \cdot r \right)  & \text{if } x \in B_{\delta_{p}}(p), \\
			x & \text{otherwise},
		\end{cases}
	\]
	which is continuous since $\psi(1) = 0$.

	Because $F_{\leq t}$ is compact, we can now choose finitely many points $p_{1}, \dots, p_{n}$ such that $F_{\leq t}$ is covered by $( B_{\nicefrac{\delta_{p_{i}}}{6}}(p_{i}))_{i}$.
	For these points $p_{i}$, we modify the homotopies $\tilde{\varphi}_{i} \defeq \tilde{\varphi}_{p_{i}}$ to obtain homotopies $\hat{\varphi}_{i} \colon F_{\leq t} \times [0,1] \to F_{\leq t}$ such that
	\begin{enumerate}[label={(\arabic*')}]
		\item \label{item:id'} $\hat{\varphi}_{i}(\cdot,0) = \operatorname{id}$,
		\item \label{item:epsilon'} there exists $\epsilon_{i} > 0$ such that
		\begin{enumerate}[label={(\alph*)}, ref={(\arabic{enumi}'\alph*)}]
			\item \label{item:s_below_s'} $\hat{\varphi}_{i}(F_{\leq s} \times [0,1]) \subseteq F_{\leq s}$ for all $s \in [t - \epsilon_{i}, t]$,
			\item \label{item:contracting'} $\hat{\varphi}_{i}(B_{\nicefrac{\delta_{p_{i}}}{3}}(p_{i}), 1) \subseteq F_{\leq t - \epsilon_{i}}$, and
		\end{enumerate}
		\item \label{item:small_change} $d(\hat{\varphi}_{i}(x, 0), \hat{\varphi}_{i}(x,1)) < \hat{\delta}$,
			where we define
			\[
				\hat{\delta} \defeq \frac{1}{6n} \min_{i} \delta_{p_{i}}.
			\]
	\end{enumerate}
	Note that $\tilde{\varphi}_{i}$ already has \cref{item:id',item:epsilon'} as a consequence of $\varphi_{p_{i}}$ having \cref{item:id,item:epsilon}, but $\tilde{\varphi}_{i}$ need not have \cref{item:small_change}.
	To construct $\hat{\varphi}_{i}$, we use the theory of $\mu$-curves from \cite{Morse.1936}:
	By \cite{Morse.1936}, for any continuous path $\eta \colon [0,1] \to F_{\leq t}$ there exists $\mu(\eta) > 0$ and a continuous reparametrization of $\eta$, denoted by $q(\eta) = \eta \circ \chi(\eta) \colon [0, \mu(\eta)] \to F_{\leq t}$ for some order-preserving homeomorphism $\chi(\eta) \colon [0, \mu(\eta)] \to [0,1]$, such that the resulting maps $(\eta,s) \mapsto q(\eta)(s)$ and $\eta \mapsto \mu(\eta)$ are continuous with respect to the Fr\'echet distance on the space of paths and $s$ varying in $[0, \mu(\eta)]$.
	Moreover, for any $\eta$ and any $s \in [0, \mu(\eta)]$, it holds that
	\begin{equation}\label{eq:mu_param}
		\frac{1}{2}\diam\{q(\eta)(r) \mid 0 \leq r \leq s\} \leq s \leq \diam\{q(\eta)(r) \mid 0 \leq r \leq s\}.
	\end{equation}
	Now, for $p \in F_{\leq t}$, let $\eta_{i}(p)$ be the path given by $\tilde{\varphi}_{i}(p,\cdot)$.
	Since $\tilde{\varphi}_{i}$ is continuous and $F_{\leq t}$ is a compact metric space and in particular locally compact Hausdorff, the map $\eta_{i}$ is continuous with respect to the compact open topology on the space of paths, which in this case is equal to the metric topology induced by the supremum distance.
	The supremum distance is an upper bound for the Fr\'echet distance, so $\eta_{i}$ is also continuous with respect to this distance.
	In total, this allows us to define a continuous map $\hat{\varphi}_{i} \colon F_{\leq t} \times [0,1] \to F_{\leq t}$ given by
	\[
	\hat{\varphi}_{i}(p,s) = q(\eta_{i}(p))(\overline{\mu}_{i}(p)\cdot s),
	\]
	where
	\[
	\overline{\mu}_{i}(p) \defeq \min \left\{ \mu(\eta_{i}(p)), \frac{\hat{\delta}}{2} \right\}.
	\]
	It is clear that $\hat{\varphi}_{i}$ has \cref{item:id',item:s_below_s'} because, by construction, for all $p$ the curve $\hat{\varphi}_{i}(p,\cdot)$ is a reparametrization of the  curve $\tilde{\varphi}_{i}(p,\cdot)$ restricted to some interval $[0,a_p]$ with $a_p >0$.

	To see that $\hat{\varphi}_{i}$ has \cref{item:contracting'}, we start by partitioning $B_{\nicefrac{\delta_{p_{i}}}{3}}(p_{i})$ into two disjoint subsets $B_{i}$ and $B'_{i}$,
	 where $B_{i}$ consists of those points $p$ for which $\overline{\mu}_{i}(p) = \mu(\eta_{i}(p))$.
	 This implies that $\hat{\varphi}_{i}(p,1) = \tilde{\varphi}_{i}(p,1)$ for all $p \in B_{i}$.
	Thus, because $\tilde{\varphi}_{i}$ has \cref{item:contracting'}, this implies
	$\hat{\varphi}_{i}(B_{i}, 1) \subseteq F_{\leq t - \epsilon_i}$ for some sufficiently small $\epsilon_i > 0$.
	It remains to verify that there exists $\epsilon_{i} > 0$ for which $\hat{\varphi}_{i}(B'_{i},1) \subseteq F_{\leq t-\epsilon_{i}}$.
	%Now, for $p \in B'_{i}$, we have $\overline{\mu}_{i}(p) < \mu(\eta_{i}(p))$.
%	Assume that
%
%	If $p \in F_{\leq t-\epsilon_{i}}$.
%
%
	If $\epsilon_{i}$ is chosen according to \cref{item:s_below_s'} for $\hat{\varphi}_{i}$, then we get \[\hat{\varphi}_{i}(B'_{i} \cap F_{\leq t-\epsilon_{i}},1) \subseteq F_{\leq t-\epsilon_{i}}.\]
	Thus, it remains to check that there exists $\epsilon_{i} > 0$ for which \[\hat{\varphi}_{i}(B'_{i} \cap F_{> t-\epsilon_{i}},1) \subseteq F_{\leq t-\epsilon_{i}}.\]
	The closure $A_{i}$ of $B'_{i} \cap F_{> t-\epsilon_{i}}$ in $F_{\leq t}$ is a compact subset of $B_{\delta_{p_{i}}}(p_{i})$, so by \cref{item:displacement} we get a displacement function $\rho_{i}$ on $A_{i}$ for $F$ and $\varphi_{p_{i}}$.
	Note that $\rho_{i}$ is also a displacement function for $\tilde{\varphi}_{i}$ on $A_{i} \subset \overline{B}_{\nicefrac{\delta_{p}}{3}}(p_i)$, where $\tilde{\varphi}_{i} = \varphi_{p_{i}}$.

	By definition of $A_i$ and $B'_{i}$, any point $p \in A_i$ satisfies
	$\mu(\eta_{i}(p)) \geq \frac{\hat{\delta}}{2}$.
	Invoking
	\cref{eq:mu_param} with $s = \frac{\hat{\delta}}{2}$ yields
	\[
	\diam\{q(\eta_{i}(p))(r) \mid 0 \leq r \leq \frac{\hat{\delta}}{2}\} \geq \frac{\hat{\delta}}{2}.
	\]
	Unraveling the definitions $q(\eta) = \eta \circ \chi(\eta)$
	and letting $a_p = \chi(\eta_{i}(p))(\frac{\hat{\delta}}{2})$
	yields
	\[
	\diam\{\eta_{i}(p)(s) \mid 0 \leq s \leq a_p\} \geq \frac{\hat{\delta}}{2}.
	\]
%
%	By compactness, the diameter is attained by two parameters $r_1<r_2$.
%	Let $s_1= \chi(\eta_{i}(p))(r_1)$, $s_2=\chi(\eta_{i}(p))(r_2)$ denote the corresponding parameters for the curve $\eta_{i}(p)$, so that
%	$q(\eta_{i}(p))(r_1) = \eta_{i}(p)(s_1)$ and $q(\eta_{i}(p))(r_2) = \eta_{i}(p)(s_2)$.
%
	By compactness, the diameter is attained by two parameters $s_1<s_2$.
%	We thus have
%	$d(\eta_{i}(p)(s_1), \eta_{i}(p)(s_2)) \geq \frac{\hat{\delta}}{2}$.
	Unraveling the definition $\eta_{i}(p)=\tilde{\varphi}_{i}(p,\cdot)$,
	we thus have
	\[
	d(\tilde{\varphi}_{i}(p,s_1), \tilde{\varphi}_{i}(p,s_2)) \geq \frac{\hat{\delta}}{2}.\]
	By definition of displacement function,
	\[F(\tilde{\varphi}_{i}(p,s_1)) - F(\tilde{\varphi}_{i}(p,s_2)) > \rho_{i}(\nicefrac{\hat{\delta}}{2}).\]
	Moreover, by \cref{item:s_below_s'} we have
	\[
	t \geq F(\tilde{\varphi}_{i}(p,s_1)).
	\]
	Together, this yields
	\[F(\tilde{\varphi}_{i}(p,s_2)) < t - \rho_{i}(\nicefrac{\hat{\delta}}{2}).\]
	Choosing $r_2$ such that $\chi(\eta_{i}(p))(r_2)= s_2$,
	we have
	\[\tilde{\varphi}_{i}(p,s_2) =
	\hat{\varphi}_{i}(p,r_2\frac{2}{\hat{\delta}}).
	\]
	The existence of the displacement function $\rho_i$ implies that $F \circ \hat{\varphi}_i(p,\cdot) \colon [0,1] \to \R$ is monotonically decreasing for $p \in A_i$, and thus
	\[F(\hat{\varphi}_{i}(p,1)) < t - \rho_{i}(\nicefrac{\hat{\delta}}{2}).\]
	Choosing $\epsilon_i < \rho_{i}(\nicefrac{\hat{\delta}}{2})$ thus ensures that $\hat{\varphi}_{i}(B'_{i} \cap F_{> t-\epsilon_{i}},1) \subseteq F_{\leq t-\epsilon_{i}}$.
	We conclude that $\hat{\varphi}_{i}$ has \cref{item:contracting'}.


%	Want:
%	$d(\hat{\varphi}_{i}(x, 0), \hat{\varphi}_{i}(x,1)) < \hat{\delta}$
%	Have:
%		\begin{equation}\label{eq:mu_param}
%		\frac{1}{2}\diam\{q(\eta)(r) \mid 0 \leq r \leq s\} \leq s \leq \diam\{q(\eta)(r) \mid 0 \leq r \leq s\}.
%	\end{equation}

	To show \cref{item:small_change}, we observe that
	\begin{align*}
	d(\hat{\varphi}_{i}(x, 0), \hat{\varphi}_{i}(x,1)) &\leq
	\diam\{q(\eta_{i}(p))(r) \mid 0 \leq r \leq \overline{\mu}_{i}(p)\} \\
	&
	\leq 2 \overline{\mu}_{i}(p)%\\
	%&
	\leq\hat{\delta},
	\end{align*}
	where the second inequality follows from \cref{eq:mu_param} and the third inequality from the definition of $\overline{\mu}_{i}$.

	%\hat{\varphi}_{i}(p,s) = q(\eta_{i}(p))(\overline{\mu}_{i}(p)\cdot s)

	Finally, we define $\varphi \colon F_{\leq t} \times [0,1] \to F_{\leq t}$ as the concatenation of the maps $\hat{\varphi}_{i}$, i.e., $\varphi(x,0) = x$ and
	\[\varphi(x,r) = \hat{\varphi}_{i} \left( \hat{\varphi}_{i-1} \left( \dots \hat{\varphi}_{2} \left( \hat{\varphi}_{1} \left( x, 1 \right), 1 \right) \dots, 1 \right), n \cdot r - i + 1\right)\] for $r \in \left( \frac{i-1}{n}, \frac{i}{n} \right]$.
	From $\hat{\varphi}_{i}$ having \cref{item:id',item:epsilon',item:small_change}, setting $\epsilon = \min_{i} \epsilon_{i}$, we obtain that $\varphi$ satisfies
	\begin{enumerate}[label={(\arabic*'')}]
		\item $\varphi(\cdot, 0) = \operatorname{id}$,
		\item $\varphi(F_{\leq s} \times [0,1]) \subseteq F_{\leq s}$ for all $s \in (t - \epsilon, t]$, and
		\item $\varphi(F_{\leq t}, 1) \subseteq F_{\leq t - \epsilon}$.\label{item:small_change_v2}
	\end{enumerate}
	The first two properties are immediate from the corresponding properties of $\hat{\varphi}_{i}$.
	To show \cref{item:small_change_v2},
	let $p \in F_{\leq t}$ and let $i$ be such that
	$p \in B_{\nicefrac{\delta_{p_{i}}}{6}}(p_{i})$.

	We first show that $\varphi(p, \nicefrac i n) \in F_{\leq t - \epsilon}$.
	We have
	\[
	\varphi(p, \nicefrac i n) = \hat{\varphi}_{i} \left( \hat{\varphi}_{i-1} \left( \dots \hat{\varphi}_{2} \left( \hat{\varphi}_{1} \left( p, 1 \right), 1 \right) \dots, 1 \right), 1\right).\]
	Since $\hat{\varphi}_{i}$ has \cref{item:contracting'}, it suffices to show that \[x:=\hat{\varphi}_{i-1} \left( \dots \hat{\varphi}_{2} \left( \hat{\varphi}_{1} \left( p, 1 \right), 1 \right) \dots, 1 \right) \in B_{\nicefrac{\delta_{p_{i}}}{3}}(p_{i}).\]
	To see this, note that $d(p_i, p) < \nicefrac{\delta_{p_{i}}}{6}$ and $d(x,p) < (i-1)\hat{\delta} \leq (i-1) \frac 1 {6n} \delta_{p_{i}} < \nicefrac{\delta_{p_{i}}}{6}$.
	By the triangle inequality, we now have $d(x, p_i) < \nicefrac{\delta_{p_{i}}}{3}$ as claimed.

	It remains to show that the point $p$ does not leave the sublevel set at the end of the homotopy, $\varphi(p, 1) \in F_{\leq t - \epsilon}$.
	To see this, note that
		\[
	\varphi(p, 1) =
	%
	\hat{\varphi}_{n} \left( \hat{\varphi}_{n-1} \left( \dots
	\hat{\varphi}_{i+1} \left(
	%
	\varphi(p, \nicefrac i n) , 1 \right)\dots, 1\right), 1 \right).\]
	Recall that the point $\varphi(p, \nicefrac i n)$ lies in the sublevel set $F_{\leq t - \epsilon}$, and each of the homotopies $\hat{\varphi}_{i+1}, \dots, \hat{\varphi}_n$ preserves the sublevel set according to \cref{item:s_below_s'} and our choice of $\epsilon = \min_{i} \epsilon_{i}$.
	This shows that, indeed, for $s \in (t - \epsilon, t]$ the inclusion $F_{\leq s} \hookrightarrow F_{\leq t}$ is a homotopy equivalence, with homotopy inverse given by $\varphi$.
\end{proof}
