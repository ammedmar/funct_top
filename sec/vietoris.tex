%!TEX root = ../func_top.tex

\section{Vietoris and \texorpdfstring{\v{C}}{}ech homology} \label{s:vietoris}

For greater generality, in this paper we considered \v{C}ech homology instead of the homology construction used by Morse in functional topology: metric Vietoris homology.
We now justify this choice by reminding the reader that these two constructions agree on compact metric spaces following \cite{Dowker.1952}. For an earlier approach, the reader may also wish to consult \cite[Section VII.6]{Lefschetz.1942}.

First, let us recall the definition of \v{C}ech homology as presented for example by \cite[Section~IX--X]{Eilenberg.1952}.
Let $X$ be a topological space and let $\Cov(X)$ be the set of all open covers of $X$ ordered by the refinement relation, which we implicitly regard as a poset category.
Recall that for an open cover $\alpha \in \Cov(X)$ its \emph{nerve} $\Nrv(\alpha)$ is defined as the simplicial complex
\begin{equation*}
\Nrv(\alpha) =
\big\{ \beta \subseteq \alpha \mid \beta \text{ is finite and } \textstyle{\bigcap_{U \in \beta}} \, U \neq \emptyset \big\}.
\end{equation*}
For any field $\F$, the nerve construction composed with the functor of simplicial homology with coefficients in $\F$ defines a functor from $\Cov(X)$ to the category of graded $\F$-vector spaces.
The \emph{\v{C}ech homology with coefficients in $\F$} of $X$ is defined as
\begin{equation*}
\CH(X; \F) \ =
\lim_{\alpha \in \Cov(X)} H(\Nrv(\alpha); \F).
\end{equation*}
As an alternative to the nerve construction, for a cover $\alpha \in \Cov(X)$ one can define $\Vietoris(\alpha)$ as the simplicial complex
\begin{equation*}
\Vietoris (\alpha) = \left\{ \sigma \subseteq X \mid \sigma \text{ is finite and } \sigma \in U \text{ for some } U \in \alpha \right\}.
\end{equation*}
This construction is dual to the nerve construction in the sense of Dowker's Theorem \cite{Dowker.1952}, which asserts that the two complexes $\Nrv (\alpha)$ and $\Vietoris (\alpha)$ are homotopy equivalent after geometric realization.
As a consequence, we have that $H (\Nrv (\alpha); \F) \cong H (\Vietoris (\alpha); \F)$.
This isomorphism is natural with respect to refinement of covers, so we get an alternative description of \v{C}ech homology as
\begin{equation*}
\CH (X; \F) \ \cong
\lim_{\alpha \in \Cov (X)} H (\Vietoris (\alpha); \F).
\end{equation*}

If $X$ is an arbitrary metric space, this is still not exactly the same as the construction of \emph{metric} Vietoris homology, as originally defined by Vietoris \cite{Vietoris.1927} and used by Morse, which in modern notation is the limit over all covers of $X$ by metric $\delta$-balls,
\begin{equation*}
\lim_{\alpha \in \Balls(X)} H (\Vietoris (\alpha); \F),
\end{equation*}
where
\begin{equation*}
\Balls (X) = \left\{ ( B_{\delta} (x) )_{x \in X} \mid \delta > 0 \right\}
\subseteq \Cov (X).
\end{equation*}
If the metric space $X$ is compact, then for each cover $\alpha$ there exists $\lambda > 0$ by Lebesgue's number lemma such that $(B_{\lambda}(x))_{x \in X}$ refines $\alpha$.
In other words, if the metric space $X$ is compact, then $\Balls (X)$ is coinitial in $\Cov (X)$, that is to say, they yield the same limit.
Thus, in this case we have a natural isomorphism
\begin{equation*}
\CH (X; \F) \ \cong \,
\lim_{\alpha \in \Balls(X)} H (\Vietoris (\alpha); \F)
\end{equation*}
In other words, for compact metric spaces the metric Vietoris homology theory employed in Morse's setting is isomorphic to the \v{C}ech homology we consider in this paper.
Note that the relevant isomorphisms above can all be seen to be natural with respect to continuous maps (see in particular \cite[Lemma 7a]{Dowker.1952}), so that in particular both homology constructions yield the same persistence modules when applied to sublevel set filtrations.
