
\section{Preliminaries}

\subsection*{Notation and Conventions}
For the rest of the paper, we let $\mathbb{F}$ be a field and whenever we write homology without explicit mention of coefficients, we mean homology with coefficients in $\mathbb{F}$.

\subsection{Sublevel set filtrations}

Treat in parallel the case of spaces $X$ and the case of sublevel filtrations $(X, f)$. Leads nicely to different connectivities.
\begin{itemize}
	\item metric space
	\item upper semi-continuous function
	\item sublevel set filtrations
\end{itemize}

\begin{defi}
	Let $M$ be a metric space. For any $p \in M$ and $\epsilon \geq 0$ denote
	\begin{equation*}
	B_\epsilon(p) = \{q \in M\ |\ d(p,q) < \epsilon\}, \qquad
	\overline B_\epsilon(p) = \{q \in M\ |\ d(p,q) \leq \epsilon\}.
	\end{equation*}
\end{defi}

\begin{defi}
	Let $X$ be a space and $f \colon X \to \R$ a function. For any $t \in \R$, the \textit{$t$-sublevel set of $f$} is defined by 
	\begin{equation*}
	X_{\leq t} = f^{-1}((-\infty, t]).
	\end{equation*}
\end{defi}

\subsection{Persistence homology}

\begin{itemize}
	\item Persistence module
	\item q-tameness
	\item Barcode
	\item bottleneck
	\item stability
\end{itemize}