\documentclass{article}
\usepackage{csquotes}
\usepackage{hyperref}
\usepackage[margin=1in]{geometry}

\title{Author's reply to the referee report of \\ \textsc{
    ``PERSISTENCE IN FUNCTIONAL TOPOLOGY: EXISTENCE OF PERSISTENCE DIAGRAMS AND MORSE INEQUALITIES''
    }
}
\author{Bauer \and Medina-Mardones \and Schmahl}
\begin{document}
	\maketitle
	\section{First point.} We have added a complete and self-contained formulation of our main result (Theorem 4.7 in the reviewers copy) to the introduction.

	A: What should we do about THM 5.1?

	\section{Second point.} Thank you for the suggestions.
	We have implemented them all as requested.

	\section{Third point.} TBW

	\section{Additional suggestion.} TBW

	\section{List of typos, minor remarks and suggestions.}
	\begin{enumerate}
		\item Replaced as suggested (RAS), thank you.
		\item RAS.
		\item We did mean ``functional analysis''. Thank you for the question.
		\item RAS.
		\item RAS.
		\item Before discussing more details of the stability result, including the concepts related to the metric theory of persistence modules we added the sentence:
		\begin{quote}
			We now provide more details regarding this result although they are not used in the present work.
		\end{quote}
		\item Definition moved as suggested.
		\item RAS.
		\item RAS.
		\item RAS.
		\item RAS, thank you. (ULI, PLEASE CHECK THIS CHANGE)
		\item RAS.
		\item Removed ``finite'' from the informal description and changed the definition of $c_d^{\epsilon}(t)$ as suggested.
		\item RAS.
		\item Replaced the preceding paragraph with:
		\begin{quote}
			In analogy to Morse's definitions, we may define, for an integer $d$ and real numbers $t$ and $\epsilon > 0$, the $(d, t, \epsilon)$-\emph{cap number} of our graded q-tame persistence module $M$ in terms of its persistence diagram as:
		\end{quote}
		\item Replaced $N$ by $M$.
		\item RAS.
		\item RAS.
		\item RAS.
		\item TBW
		\item RAS.
		\item (I THINK WE ARE ASSUMING HAUSDORFF, RIGHT? IS THIS NOT MENTIONED?)
		Max: I guess not in general. Here, we want the version 3' of local compactness from \url{https://en.wikipedia.org/wiki/Locally_compact_space}, which is only equivalent to weaker versions for Hausdorff spaces. We apply the lemma only to Hausdorff spaces, so it's fine to add the assumption.

		Added Hausdorff assumption as suggested, also added the assumption for the domain $Y$ to be compact in the main lemma.
		\item TBW
		\item RAS.
		\item RAS.
		\item Assumption that $\varphi$ should be non-decreasing was indeed missing and has been added, thank you.
		\item RAS.
		\item RAS.
		\item (I DO NOT KNOW)
		\item TO BE ADDRESS AFTER DISCUSSING WHAT ``homotopically critical point'' ARE.
		\item (I DO NOT KNOW)
		Max: Not making a change, yet, but i think the referee is correct. As stated right now "minimum type" would only refer to global minima, but it should probably also refer to local minima. Please check this, Uli.
		\item RAS.
		\item RAS.
		\item RAS.
		\item RAS.
		\item RAS.
		\item Made it clearer that the results in the appendix are not original and added suggested references in the first paragraph.
		\item Added a remark as suggested at the end.
		\item RAS.
		\item Explained that this indeed follows from Lebesgue's number lemma.
	\end{enumerate}
\end{document}
