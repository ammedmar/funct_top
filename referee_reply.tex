\documentclass{article}
\usepackage{csquotes}
\usepackage{hyperref}
\usepackage[margin=1in]{geometry}

\title{Author's reply to the referee report of \\ \textsc{
    ``PERSISTENCE IN FUNCTIONAL TOPOLOGY: EXISTENCE OF PERSISTENCE DIAGRAMS AND MORSE INEQUALITIES''
    }
}
\author{Bauer \and Medina-Mardones \and Schmahl}
\begin{document}
	\maketitle
	\section{First point.} We have added a complete and self-contained formulation of our main result (Theorem 4.7 in the reviewers copy) to the introduction.
    We have also introduced in section 5 the necessary definitions in order to state the Unstable Minimal Surface Theorem (5.1 in the reviewer's copy), now referenced in the introduction.
    (Please see point 3 below.)

	\section{Second point.} Thank you for the suggestions.
	We have implemented them all as requested.

	\section{Third point.} Thank you for the suggestions. 
	We have included a proof of the Unstable Minimal Surface Theorem and a formal definition of the notion of homotopically critical point in Section 5.1, mostly following the general outline that was suggested. 
	The proof is broken down into some intermediate lemmas and cites Morse--Tompkins and Struwe as necessary.
	We have also added the notions of \emph{number of births} and \emph{number of deaths} and gave some basic formulas to compute them, which was necessary to make the later proofs work.
	Notably, another slight correction to a definition used by Morse--Tompkins was necessary: the notion of a \emph{critical set of minimum type} is not accurately captured by the original definition, as discussed in Remark 5.3.

	\section{Additional suggestion.} We have implemented the conceptually easier proof of the Morse inequalities as suggested, thank you.

	\section{List of typos, minor remarks and suggestions.}
	\begin{enumerate}
		\item Replaced as suggested (RAS), thank you.
		\item RAS.
		\item We did mean ``functional analysis''. To make clearer what exactly we were referring to, we changed the wording to ``calculus of variations". Thank you for the question.
		\item RAS.
		\item RAS.
		\item Before discussing more details of the stability result, including the concepts related to the metric theory of persistence modules we added the sentence:
		\begin{quote}
			We now provide more details regarding this result although they are not used in the present work.
		\end{quote}
		\item Definition moved as suggested.
		\item RAS.
		\item RAS.
		\item RAS.
		\item RAS.
		\item RAS.
		\item Removed ``finite'' from the informal description and changed the definition of $c_d^{\epsilon}(t)$ as suggested.
		\item RAS.
		\item Replaced the preceding paragraph with:
		\begin{quote}
			In analogy to Morse's definitions, we may define, for an integer $d$ and real numbers $t$ and $\epsilon > 0$, the $(d, t, \epsilon)$-\emph{cap number} of our graded q-tame persistence module $M$ in terms of its persistence diagram as:
		\end{quote}
		\item Replaced $N$ by $M$.
		\item RAS.
		\item RAS.
		\item RAS.
		\item Paragraph has been reformulated.
		\item RAS.
		\item Clarified as suggested by adding the Hausdorff assumption explicitly.
		\item RAS.
		\item RAS.
		\item RAS.
		\item Assumption that $\varphi$ should be non-decreasing was indeed missing and has been added, thank you.
		\item RAS.
		\item RAS.
		\item The sentence ``for a large class of curves'' seems to have been a remnant of a previous version where we did not make the assumptions on the curve explicit. We have removed this part and included a precise reference to the relevant part in Morse--Tompkins' paper.
		\item RAS.
		\item It should have indeed said ``there exists a neighborhood'' instead of ``for every neighborhood''. We also realised there was another assumption missing: The neighborhood should not only contain the critical set but in fact its closure -- in general a critical set will only be closed in its level set, but not necessarily in the whole space.
		\item RAS.
		\item RAS.
		\item RAS.
		\item RAS.
		\item RAS.
		\item Made it clearer that the results in the appendix are not original and added suggested references in the first paragraph.
		\item Added a remark as suggested at the end.
		\item RAS.
		\item Explained that this indeed follows from Lebesgue's number lemma.
	\end{enumerate}
	
	\section{Other changes}
	The notion that was formerly called 'homotopically locally connected' is now simply called 'locally connected of all orders' and the notion that was formerly called 'locally connected' is now called 'locally contractible', which seems to be more in line with the existing literature.
	
	We have also added a reference to a paper by Cagliari and Landi where a multiparameter version of Corollary 4.12 in the current version has been established for slightly stronger assumptions on the domain of the function.
	
	Considering the changes suggested by the reviewer and feedback gathered from other mathematicians while presenting this work, we have settled on a new title: ``Persistent homology for functionals.''
	
	We have also made several small changes to the expository parts of the paper, expanded the discussion on the historical assumptions for the Mountain Pass Theorem by Morse and Tompkins, and fixed typos throughout.
\end{document}
