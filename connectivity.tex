
\section{Persistence diagrams from local connectivity} \label{s:connectivity}

We have seen that q-tame functions admit persistence diagrams, which can be used to formulate Morse inequalities.
With q-tameness being a rather abstract condition, we now establish more concrete topological conditions that ensure the q-tameness of a function.
Our definitions are motivated by similar conditions that Morse considered in his work on functional topology.
We will present a historical account in \cref{s:surfaces}.

Whether a function is q-tame or not depends on the functor that is used to pass from the sublevel set filtration of the function to a persistence module.
The general idea that we want to employ is to deduce global finiteness properties like q-tameness from local ones.
Thus, the functors we consider should have some property that allows us to do so.

Concretely, let $\H = (\H_d)_{d \in \Z} \colon \Top \to \Vect$ be a fixed graded homotopy invariant functor, which we call a \emph{homology theory}.
A triple of spaces $X_{1}, X_{2} \subseteq X$ is said to have a \emph{Mayer-Vietoris sequence} for $\H$ if the inclusion-induced maps fit in a long exact sequence
\begin{equation*}
\begin{tikzcd}[column sep = small]
\cdots \arrow[r] & \H_{n}(X_{1} \cap X_{2}) \arrow[r] & \H_{n}(X_{1}) \oplus \H_{n}(X_{2}) \arrow[r] & \H_{n}(X) \arrow[r] & \H_{n-1}(X_{1} \cap X_{2}) \arrow[r] & \cdots \ .
\end{tikzcd}
\end{equation*}
We say that $\H$ has the \emph{open} (resp.\@\ \emph{compact}) \emph{Mayer-Vietories property} if there are natural Mayer-Vietoris sequences for all triples $X_{1}, X_{2} \subseteq X$ with $X = X_1 \cup X_2$ and $X_i \subseteq X$ open (resp.\@\ compact Hausdorff).

For the rest of this section, we will assume that $\H$ is a homology theory that has either the open or the compact Mayer-Vietoris property and for which there is $n_0$ such that $\H_{n}$ is 0 for all $n \leq n_0$.
Note that this includes singular homology with field coefficients, which, like any homology theory in the sense of the Eilenberg-Steenrod axioms \cite[Section I]{Eilenberg.1952}, has the open Mayer-Vietoris property, and it also includes \v{C}ech homology with field coefficients, which has the compact Mayer-Vietoris property.

\begin{defi}
	For $n \in \Z$ a continuous map is said to be \emph{$n$-homologically small} or $\HS_n$ if the image of the map induced by $\H_{n}$ is finite dimensional.
	We omit references to $n$ if the condition holds for all integers.
\end{defi}

\begin{defi}
	The sublevel set filtration of a function $f \colon X \to \R$ is said to be \emph{locally homologically small} or $\HLC$ if for any $x \in X$, any neighborhood $V$ of $x$, and any index $t > f(x)$, there is an index $s$ with $f(x) < s < t$ and a neighborhood $U$ of $x$ with $U \subseteq V$ such that the inclusion $f_{\leq s} \cap U \to f_{\leq t} \cap V$ is $\HS$.
\end{defi}

Similarly, we have the following stronger version of this property.

\begin{defi}
	The sublevel set filtration of a function $f \colon X \to \R$ is called \emph{strongly locally homologically small} or \emph{strongly $\HLC$} if for any $x \in X$, any neighborhood $V$ of $x$, and any pair of indices $s,t$ with $f(x) < s < t$ there is a neighborhood $U$ of $x$ with $U \subseteq V$ such that the inclusion $f_{\leq s} \cap U \to f_{\leq t} \cap V$ is $\HS$.
\end{defi}

Clearly, any strongly $\HLC$ sublevel set filtration is also $\HLC$:
while the $\HLC$ property merely requires the existence of an index $s \in (f(x),t)$ satisfying the $\HS$ condition, the \emph{strong} $\HLC$ property requires the $\HS$ condition to hold for any $s \in (f(x),t)$.
If the filtration is induced by a continuous function, the converse also holds as the following theorem shows.

\begin{thm} \label{t:hlc to strong hlc}
	If the sublevel set filtration of a continuous function $f \colon X \to \R$ is $\HLC$, then it is also strongly $\HLC$.
\end{thm}

\begin{proof}
	Fix $x \in X$, a neighborhood $V$ of $x$ and indices $f(x) < s < t$.
	We need to show that there is a neighborhood $U \subseteq V$ of $x$ such that the inclusion $f_{\leq s} \cap U \to f_{\leq t} \cap V$ is $\HS$.
	
	To do so, we start by using the $\HLC$ property to choose a neighborhood $U' \subseteq V$ of $x$ and an index $s' \in (f(x),\, t)$ such that the inclusion $f_{\leq s'} \cap U' \to f_{\leq t} \cap V$ is $\HS$.
	Now, we choose $U = f_{< s'} \cap U'$, where $f_{< s'} = f^{-1} (-\infty, s')$.
	Note that this choice of $U$ still defines a neighborhood of $x$ because $f$ is assumed to be continuous, so that $f_{< s'}$ is an open subset of $X$.
	
	We obtain that $f_{\leq s} \cap U \subseteq f_{\leq s'} \cap U'$, so that the inclusion $f_{\leq s} \cap U \to f_{\leq t} \cap V$ factors through the inclusion $f_{\leq s'} \cap U' \to f_{\leq t} \cap V$.
	This second map is $\HS$ by construction.
	Any map that factors through an $\HS$ map is also $\HS$, so the proof is finished.
\end{proof}

As our main result, we will now show that for compact sublevel set filtrations the strong $\HLC$ condition implies \mbox{q-tameness}, and consequently also implies the existence of a persistence diagram which yields persistent Morse inequalities.

\begin{thm} \label{t:strong local connectedness implies q-tameness}
	If the sublevel set filtration of a function $f \colon X \to \R$ is compact and strongly $\HLC$, then it is also q-tame.
\end{thm}

The general proof strategy is inspired by the proof of a result attributed to Wilder as presented by Bredon \cite[Section II.17]{Bredon.1997}.
We collect the main ideas in several lemmas.

\begin{lem} \label{l:commutative algebra}
	Given a commutative diagram of modules over a principal ideal domain
	\begin{equation*}
	\begin{tikzcd}
	A_{1,1} \arrow[r] & A_{1,2} & \\
	A_{2,1} \arrow[r] \arrow[u] & A_{2,2} \arrow[r] \arrow[u] & A_{2,3} \\
	& A_{3,2} \arrow[r] \arrow[u] & A_{3,3} \arrow[u]
	\end{tikzcd}
	\end{equation*}
	where the middle row is exact and both $A_{2,1} \to A_{1,1}$ and $A_{3,3} \to A_{2,3}$ have finitely generated images, then so does $A_{3,2} \to A_{1,2}$.
\end{lem}

\begin{proof}
	This is proven via a straightforward diagram chase.
For more details see \cite[Lemma II.17.3]{Bredon.1997}.
\end{proof}

\begin{lem} \label{l:neighborhood third}
	Let $X$ be a locally compact space.
	For any compact subset $K$ and open set $U$ with $K \subseteq U$ there exists a compact set $K^\prime$ such that
	\begin{equation*}
	K \subseteq \interior(K^\prime) \subseteq K^\prime \subseteq U.
	\end{equation*}
\end{lem}

\begin{proof}
	For any $x \in K$ choose a compact neighborhood $C(x) \subseteq U$.
	We have
	\begin{equation*}
	K \subseteq \bigcup_{x \in K} \interior(C(x)) \subseteq \interior\left(\bigcup_{x \in K} C(x)\right) \subseteq \bigcup_{x \in K} C(x) \subseteq U
	\end{equation*}
	Since $K$ is compact, the first inclusion above is achieved over a finite subset $\{x_1, \dots, x_m\}$ of elements in $K$.
	Defining $K^\prime = \bigcup_{i=1}^m C(x_i)$ finishes the proof.
\end{proof}

We want to use \cref{l:neighborhood third} on the domain of the function whose sublevel set filtration we consider. However, we do not want to assume the domain to be locally compact for \cref{t:strong local connectedness implies q-tameness}. To circumvent this, we will work in one of the sublevel sets, which are assumed to be compact Hausdorff and hence locally compact. This requires the use of a slight weakening of the strong $\HLC$ condition.

\begin{defi}
	For $u \in \R$, the sublevel set filtration of a function $f \colon X \to \R$ is called \emph{strongly $\HLC$ below $u$} if for any $x \in X$, any neighborhood $V$ of $x$, and any pair of indices $s,t$ with $f(x) < s < t < u$ there is a neighborhood $U$ of $x$ with $U \subseteq V$ such that the inclusion $f_{\leq s} \cap U \to f_{\leq t} \cap V$ is $\HS$.
\end{defi}

\begin{lem} \label{l:restriction of LHS filtration}
	Let $f \colon X \to \R$ be a function whose sublevel set filtration is strongly $\HLC$.
	Fix $u \in \R$ and let $g \colon Y \to \R$ be the restriction of $f$ to the sublevel set $Y = f_{\leq u}$.
	Then, the sublevel set filtration defined by $g$ is strongly $\HLC$ below $u$.
\end{lem}

\begin{proof}
	Let $x \in Y$, $V$ a neighborhood of $x$ in $Y$ and consider indices $s,t$ with $g(x) < s < t < u$. 
	We need to find a neighborhood $U \subseteq V$ of $x$ such that the inclusion $g_{\leq s} \cap U \to g_{\leq t} \cap V$ is $\HS$.
	
	Since $Y \subseteq X$ carries the subspace topology, we may choose a neighborhood $V'$ of $x$ in $X$ such that $V = V' \cap Y$. 
	The sublevel set filtration of $f$ is assumed to be strongly $\HLC$, so there is a neighborhood $U' \subseteq V'$ of $x$ in $X$ such that the inclusion $f_{\leq s} \cap U' \to f_{\leq t} \cap V'$ is $\HS$.
	We set $U = U' \cap Y$, which defines a neighborhood of $x$ in $Y$.
	
	Now, $s < t < u$ implies that we have $g_{\leq s} = f_{\leq s}$ and $g_{\leq t} = f_{\leq t}$. 
	Moreover, we have $f_{\leq s} \cap Y = f_{\leq s} \cap f_{\leq u} = f_{\leq s}$ and $f_{\leq t} \cap Y = f_{\leq t} \cap f_{\leq u} = f_{\leq t}$. 
	Thus, we obtain $g_{\leq s} \cap U = f_{\leq s} \cap Y \cap U' = f_{\leq s} \cap U'$ and $g_{\leq t} \cap V = f_{\leq t} \cap Y \cap V' = f_{\leq t} \cap V'$.
	This implies that the inclusion $g_{\leq s} \cap U \to g_{\leq t} \cap V$ is $\HS$ because it agrees with the inclusion $f_{\leq s} \cap U' \to f_{\leq t} \cap V'$, which is $\HS$ by construction.
	This finishes the proof.
\end{proof}

\begin{lem} \label{l:key lemma for q-tameness}
	Let $f \colon X \to \R$ be a function on a locally compact space $X$ whose sublevel set filtration is compact and strongly $\HLC$ below $u \in \R$, and consider subsets $C \subseteq L \subseteq X$ with $C$ compact and $L$ open.
	For any $s < t < u$ the inclusion $C \cap f_{\leq s} \to L \cap f_{\leq t}$ is $\HS$.
\end{lem}

\begin{proof}
	Recall that we are assuming that the underlying homology theory $\H$ has either the open or the compact Mayer-Vietoris property and that there is some $n_0$ such that $\H_{n}$ is 0 for all $n \leq n_0$.
	The statement of the lemma holds for $\HS_{n}$ in place of $\HS$ for any $n \leq n_0$ since $\H_{n}$ induces the zero map.
	We will proceed by induction on $n \geq n_0$ assuming the statement for $\HS_{n-1}$.
	
	We define $\Sigma_{s, t}$ to be the collection of all open subsets $V \subseteq X$ whose closure $\overline{V}$ is compact, contained in $L$, and has an open neighborhood $U$ with $\overline{V} \subseteq U \subseteq L$	for which there exists $s' \in (s,\, t)$ such that the inclusion $U \cap f_{\leq s'} \to L \cap f_{\leq t}$ is $\HS_n$.
	We will show that $\Sigma_{s, t}$ has the following three properties:
	\begin{enumerate}
		 \item Any point $x \in L \cap f_{\leq s}$ has a neighborhood $V_x \in \Sigma_{s,t}$.
		 \item If $V_1,\, V_2 \in \Sigma_{s,t}$ then $V_1 \cup V_2 \in \Sigma_{s,t}$.
		 \item For each $V \in \Sigma_{s,t}$ the inclusion 
		 $V \cap f_{\leq s} \to L \cap f_{\leq t}$ is $\HS_n$.
	\end{enumerate}
	
	Assuming them for the moment, the first property allows us to cover $C$ by sets $V_x \in \Sigma_{s,t}$, $x \in C$.
	Because $C$ is compact, this cover can be chosen finite, represented by say $x_1,\dots, x_m$.
	By the second property, we have $V = \bigcup_{i = 1}^m V_{x_i} \in \Sigma_{s,t}$.
	Using the third property, the inclusion 
	$V \cap f_{\leq s} \to L \cap f_{\leq t}$ 
	is $\HS_n$, so the inclusion 
	$C \cap f_{\leq s} \to L \cap f_{\leq t}$ 
	is also $\HS_n$ because this map factors through the aforementioned one.
	What is left to do is to show that $\Sigma_{s,t}$ has the properties we want.
	
	For the third property, let $V \in \Sigma_{s,t}$ with $U$ an open neighborhood of $\overline{V}$ in $L$ and $s' \in (s,\, t)$ such that 
	$U \cap f_{\leq s'} \to L \cap f_{\leq t}$
	is $\HS_n$.
	The inclusion
	$V \cap f_{\leq s} \to L \cap f_{\leq t}$
	factors through the previous one so it is $\HS_n$ as well.
	Thus, $\Sigma_{s, t}$ has the third property we want.
	
	Next, we will show using the strong $\HLC$ property that $\Sigma_{s, t}$ has the first required property, i.e., that any point $x \in L \cap f_{\leq s}$ has a neighborhood in $\Sigma_{s, t}$.
	Choose an arbitrary $s' \in (s,\, t)$.
	Since the sublevel set filtration of $f$ is strongly $\HLC$ below $u$ and we have $s' < t < u$, there is an open neighborhood $U_x \subseteq L$ such that the inclusion
	$U_x \cap f_{\leq s'} \to L \cap f_{\leq t}$
	is $\HS$, so in particular $\HS_n$.
	By local compactness of $X$ we can choose a compact neighborhood $K_x$ of $x$ contained in $U_x$.
	Now $V_x = \interior (K_x)$ is a neighborhood of $x$ with $V_x \in \Sigma_{s,t}$.
	
	Finally, using Mayer-Vietoris and the induction hypothesis we will show that $\Sigma_{s,t}$ has the second required property, i.e., that it is closed under finite unions.
	So for $i \in \{1, 2\}$ let $V_i \in \Sigma_{s,t}$ with $U_i$ and $s'_i \in (s,\, t)$ such that 
	$\overline{V_i} \subseteq U_i \subseteq L$ 
	and
	$U_{i} \cap f_{\leq s'_i} \to L \cap f_{\leq t}$
	is $\HS_n$.
	Writing $K_i = \overline{V_i}$, we use \cref{l:neighborhood third} to construct compact sets $K'_i$ such that
	\begin{equation*}
	V_i \subseteq K_i \subseteq V'_i \subseteq K'_i \subseteq U_i \subseteq L
	\end{equation*}
	where $V'_i = \interior(K'_i)$.
	We have that the union $V_1 \cup V_2 \subseteq L$ is open, its closure $\overline{V_1 \cup V_2}$ is compact, and we have $\overline{V_1 \cup V_2} \subseteq V'_1 \cup V'_2 \subseteq L$.
	Thus, we obtain $V_1 \cup V_2 \in \Sigma_{s,t}$ if we can show that there is an $s' \in (s,\, t)$ such that the inclusion 
	$\left(V'_1 \cup V'_2 \right) \cap f_{\leq s'} \to L \cap f_{\leq t}$
	is $\HS_n$.
	To do so, we set $s'' = \min_i s'_i$ and choose $s' \in (s,\, s'')$.
	For proving that $\left(V'_1 \cup V'_2 \right) \cap f_{\leq s'} \to L \cap f_{\leq t}$ is $\HS_n$ we now distinguish the two cases where $\H$ has either the open or the compact Mayer-Vietoris property.
	
	In the open case, notice that for both choices of $i$ the inclusions
	$U_i \cap f_{\leq s''} \to L \cap f_{\leq t}$
	are $\HS_n$.
	Additionally, the inclusion
	$V'_1 \cap V'_2 \cap f_{\leq s'} \to U_1 \cap U_2 \cap f_{\leq s''}$
	is $\HS_{n-1}$ because it factors through the inclusion
	$K'_1 \cap K'_2 \cap f_{\leq s'} \to U_1 \cap U_2 \cap f_{\leq s''}$,
	which is $\HS_{n-1}$ by the induction hypothesis.
	Because the $V_i$ and $V'_i$ are open and because $\H$ has the open Mayer-Vietoris property, we obtain the following commutative diagram satisfying the assumptions of \cref{l:commutative algebra}:
	\begin{equation*}
	\begin{tikzcd}[column sep=small]
	\H_n(L \cap f_{\leq t}) \oplus \H_n(L \cap f_{\leq t}) \arrow[r] &
	\H_n(L \cap f_{\leq t}) & \\
	\H_{n}(U_1 \cap f_{\leq s''}) \oplus \H_n(U_2 \cap f_{\leq s''}) \arrow[r] \arrow[u] & 
	\H_{n}((U_1 \cup U_2) \cap f_{\leq s''}) \arrow[r] \arrow[u] &
	\H_{n-1}(U_1 \cap U_2 \cap f_{\leq s''}) \\ & 
	\H_{n}((V'_1 \cup V'_2) \cap f_{\leq s'}) \arrow[r] \arrow[u] &
	\H_{n-1}(V'_1 \cap V'_2 \cap f_{\leq s'}). \arrow[u]
	\end{tikzcd}
	\end{equation*}
	We conclude that the inclusion 
	$\left(V'_1 \cup V'_2 \right) \cap f_{\leq s'} \to L \cap f_{\leq t}$ 
	is $\HS_n$, which finishes this part of the proof.
	
	In the compact case, we apply \cref{l:neighborhood third} again to obtain compact sets $K''_i$ such that
	\begin{equation*}
	V_i \subseteq K_i \subseteq V'_i \subseteq K'_i \subseteq V''_i \subseteq K''_i \subseteq U_i \subseteq L
	\end{equation*}
	where $V''_i = \interior(K''_i)$.
	The rest of the proof is then analogous to the previous case:
	We have that for both choices of~$i$ the inclusions
	$K''_i \cap f_{\leq s''} \to L \cap f_{\leq t}$
	are $\HS_n$ because this is true for the corresponding inclusions with $K''_i$ replaced by $U_i$.
	Moreover, the inclusion
	$K'_1 \cap K'_2 \cap f_{\leq s'} \to K''_1 \cap K''_2 \cap f_{\leq s''}$
	is $\HS_{n-1}$ because it factors through the inclusion
	$K'_1 \cap K'_2 \cap f_{\leq s'} \to V''_1 \cap V''_2 \cap f_{\leq s''}$,
	which is $\HS_{n-1}$ by the induction hypothesis.
	Because the $K'_i$ and $K''_i$ as well as the sublevel sets of $f$ are all compact and because $\H$ has the compact Mayer-Vietoris property, we obtain the following commutative diagram satisfying the assumptions of \cref{l:commutative algebra}:
	\begin{equation*}
	\begin{tikzcd}[column sep=small]
	\H_n(L \cap f_{\leq t}) \oplus \H_n(L \cap f_{\leq t}) \arrow[r] &
	\H_n(L \cap f_{\leq t}) & \\
	\H_{n}(K''_1 \cap f_{\leq s''}) \oplus \H_n(K''_2 \cap f_{\leq s''}) \arrow[r] \arrow[u] & 
	\H_{n}((K''_1 \cup K''_2) \cap f_{\leq s''}) \arrow[r] \arrow[u] &
	\H_{n-1}(K''_1 \cap K''_2 \cap f_{\leq s''}) \\ & 
	\H_{n}((K'_1 \cup K'_2) \cap f_{\leq s'}) \arrow[r] \arrow[u] &
	\H_{n-1}(K'_1 \cap K'_2 \cap f_{\leq s'}). \arrow[u]
	\end{tikzcd}
	\end{equation*}
	We conclude that the inclusion 
	$\left(K'_1 \cup K'_2 \right) \cap f_{\leq s'} \to L \cap f_{\leq t}$
	is $\HS_n$, so the same is true for the inclusion
	$\left(V'_1 \cup V'_2 \right) \cap f_{\leq s'} \to L \cap f_{\leq t}$ 
	because it factors through the previous one.
\end{proof}

We can now complete the proof of the claim stating that for compact sublevel set filtrations, strong $\HLC$ implies q-tameness.

\begin{proof}[Proof of \cref{t:strong local connectedness implies q-tameness}]
	By definition, the sublevel set filtration of $f$ is q-tame if and only if the inclusion $f_{\leq s} \to f_{\leq t}$ is $\HS$ for all pairs $s < t$. 
	Choose $u \in \R$ with $u > t$ and let $g \colon Y \to \R$ be the restriction of $f$ to the sublevel set $Y=f_{\leq u}$. 
	Since we assume $f$ to induce a strongly $\HLC$ sublevel set filtration, by \cref{l:restriction of LHS filtration} the sublevel set filtration of $g$ is strongly $\HLC$ below $u$. 
	Clearly, the sublevel set filtration of $g$ is also compact, and its domain $Y$ is locally compact being a compact Hausdorff space by assumption.
	Thus, we can apply \cref{l:key lemma for q-tameness} to the filtration $g_{\leq \bullet}$ with $C = f_{\leq s}$ and $L = Y$ to obtain that the inclusion $f_{\leq s} = C \cap g_{\leq s} \to L \cap g_{\leq t} = f_{\leq t}$ is $\HS$.
\end{proof}

Combining \cref{t:hlc to strong hlc} with \cref{t:strong local connectedness implies q-tameness} also yields the following result for continuous functions.

\begin{cor} \label{c:q-tameness for continuous functions}
	If the sublevel set filtration of a continuous function $f \colon X \to \R$ is compact and $\HLC$, then it is also q-tame.
\end{cor}


