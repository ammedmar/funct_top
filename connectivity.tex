
\section{Local connectivity}

\begin{defi} \label{defi:local_connectedness}
	For $n \in \Z$ a continuous map is said to be \textit{$n$-homologically small} or \textit{trivial}, denoted $\HS_n$ and $\HT_n$ respectively, if the image of the map induced by $\tilde H_n(-)$ is finitely generated or 0, and we say it is \textit{homologically small} or \textit{trivial}, denoted respectively $\HS$ and $\HT$, if it is $\HS_n$ or $\HT_n$ for every~$n$.
	
	A space $X$ is said to be \emph{homologically locally connected}, denoted $\HLC$, if for each $x \in X$ any neighborhood $V$ of $x$ contains a neighborhood $U$ of $x$ such that the inclusion $U \to V$ is $\HT$.
\end{defi}

\anibal{Use a different example, $\H$ not yet defined.}
\begin{ex}
	For any $d$, the $d$-dimensional Hawaiian earring is not $\HLC$. However, if we consider $\H$ as a subspace of $\mathbb{H}^{d'}$ via the obvious embedding for $d < d'$, then $\H$ is $\HLC$ in $\mathbb{H}^{d'}$.
\end{ex}

\begin{defi}
	Let $M$ be a metric space and $f \colon M \to \R$ a function.
	The space $M$ is said to be \textit{locally-$f$-connected} if for each $p \in M$, $e > 0$, and $n \in \N$ there exists $\delta \in (0, e)$ such that for every $c \geq 0$ the inclusion $B_\delta(p) \cap M_{\leq f(p)+\delta+c} \to B_e(p) \cap M_{\leq f(p)+e+c}$ is $\HT$.
\end{defi}

Local connectedness assumptions have a long history of being used to ensure that different homology theories agree and to ensure that homology is finite dimensional, see \cite{MR0007094} for early examples. For us, the following is relevant.

\begin{prop}[{\cite{MR105677, MR1481706}}]\label{prop:cech_sing_hom_hlc}
	On the category of paracompact Hausdorff $\HLC$ spaces, \v{C}ech and singular homology with arbitrary coefficients coincide.
\end{prop}

We also get a persistent version by applying the above pointwise.

\begin{cor}\label{cor:cech_sing_persistent_iso}
	Let $M$ be a metric space and $f\colon M\to\mathbb{R}$ a function such that $f_{\leq c}$ is a paracompact $HLC^\infty$-space for each $c\in\mathbb{R}$. Then $\CH_d(f_{\leq\bullet})\cong H_d(f_{\leq\bullet})$ for any $d$.
\end{cor}

For more results of this type, we refer to \cite{MR1481706}. There, the above result \cite[Corollary VI.12.6]{MR1481706} and many similar ones are shown using sheaf theoretic methods. For example, Bredon also discusses cohomology local connectedness \cite[Section II.17]{MR1481706} and various other comparison results, e.g.\@ between singular and Borel-Moore homology \cite[Corollary V.12.15]{MR1481706} in the presence of homology local connectedness, as well as examples such as spaces that are sheaf cohomology locally connected but not homology locally connected \cite[Example II.17.12]{MR1481706}. Using Bredon's results, we also get the following.

\begin{prop}\label{prop:fin_dim_sing_hom}
	Let $X$ be compact Hausdorff and $HLC^{\infty}$. Then $H_d(X;\mathbb{F})$ is finite-dimensional for all $d$ and any field $\mathbb{F}$.
\end{prop}
 
\begin{proof}
	By the universal coefficient theorem for homology, trivial singular homology with integer coefficients also implies trivial singular homology for any other coefficient group. In particular, this means that $X$ being $HLC^{\infty}$ also implies that $X$ is $HLC^{\infty}_{\mathbb{F}}$. By the universal coefficient theorem for cohomology, the singular cohomology with coefficients in $\mathbb{F}$ is naturally isomorphic to the dual space of the singular homology with coefficients in $\mathbb{F}$. This means that being $HLC^{\infty}_{\mathbb{F}}$ implies being locally connected with respect to singular cohomology in all degrees with coefficients in $\mathbb{F}$. Since we assume $X$ to be $HLC^{\infty}$, \cite[Theorem III.1.1]{MR1481706} now implies that $X$ is locally connected with respect to sheaf cohomology with coefficients in the locally constant sheaf with value $\mathbb{F}$. \cite[Corollary II.17.7]{MR1481706} implies that the sheaf cohomology of the whole space with coefficients in this sheaf is finite-dimensional, so again by \cite[Theorem III.1.1]{MR1481706} the same holds for singular cohomology with coefficients in $\mathbb{F}$. Applying the universal coefficient theorem once more proves the claim.
\end{proof}

Again, we get a persistent version.

\begin{cor}
	Let $M$ be a topological space and $f\colon M\to\mathbb{R}$ a function such that $f_{\leq c}$ is a compact Hausdorff $HLC^\infty$-space for each $c\in\mathbb{R}$. Then $H_d(f_{\leq\bullet})$ is p.f.d. for any $d$.
\end{cor}