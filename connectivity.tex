
\section{Local connectivity}

Let $\H$ be a fixed homology theory.

\begin{defi} \label{defi:local_connectedness}
	For $n \in \Z$ a continuous map is said to be \textit{$n$-homologically small} or \textit{trivial}, denoted $\HS_n$ and $\HT_n$ respectively, if the image of the map induced by $\H_n$ is finitely generated or 0, and we say it is \textit{homologically small} or \textit{trivial}, denoted respectively $\HS$ and $\HT$, if it is $\HS_n$ or $\HT_n$ for every~$n$.
\end{defi}

\begin{defi}
	A space $X$ is said to be \emph{homologically locally connected}, denoted $\HLC$, if for each $x \in X$ any neighborhood $V$ of $x$ contains a neighborhood $U$ of $x$ such that the inclusion $U \to V$ is $\HT$.
	
	A sublevel filtration $(X,f)$ is said to be $\emph{homologically locally connected}$, denoted $\HLC$, if for any $\epsilon > 0$, $x \in X$, and $V$ an open neighborhood of $x$, there is a $\delta \in (0, \epsilon)$ and a neighborhood $U$ of $x$ such that $U \subseteq V$ and
	\begin{equation*}
	X_{\leq f(x) + \delta} \cap U \to X_{f(x) + \epsilon} \cap V
	\end{equation*}
	is $\HT$.
\end{defi}

%\anibal{Use a different example, $\HE$ not yet defined.}
%\begin{ex}
%	The $d$-dimensional Hawaiian earring $\HE$ is not $\HLC$. However, if we consider $\HE$ as a subspace of $\mathbb{H}^{d'}$ via the obvious embedding for $d < d'$, then $\HE$ is $\HLC$ in $\mathbb{H}^{d'}$.
%\end{ex}

Local connectivity assumptions have a long history of being used to ensure that different homology theories agree and to ensure that homology is finite dimensional, see \cite{MR0007094} for early examples.
For us, the following is relevant.
Recall that a space is said to be \textit{paracompact} if every open cover has a locally finite open refinement, i.e., every point has a neighborhood intersecting finitely many sets in the refinement.

\begin{prop}[{\cite{MR105677, MR1481706}}]\label{prop:cech_sing_hom_hlc}
	On the category of paracompact Hausdorff $\HLC$ spaces, \v{C}ech and singular homology with arbitrary coefficients coincide.
\end{prop}

We also get a persistent version by applying the above pointwise.

\begin{cor}\label{cor:cech_sing_persistent_iso}
	Let $(X, f)$ be a sublevel filtration. If each sublevel set is paracompact and $\HLC$, then its singular and \v{C}ech persistence modules coincide.
\end{cor}

For more results of this type, we refer to \cite{MR1481706}. There, the above proposition \cite[Corollary VI.12.6]{MR1481706} and many similar ones are proven using sheaf theoretic methods.
For example, Bredon also discusses cohomology local connectedness \cite[Section II.17]{MR1481706} and various other comparison results, e.g.\@ between singular and Borel-Moore homology \cite[Corollary V.12.15]{MR1481706} in the presence of homology local connectedness, as well as examples such as spaces that are sheaf cohomology locally connected but not homology locally connected \cite[Example II.17.12]{MR1481706}.
Using Bredon's results, we also get the following.

\begin{prop} \label{prop:fin_dim_sing_hom}
	Let $X$ be a compact Hausdorff $\HLC$ space.
	Then the singular homology of $X$ with field coefficients is finite-dimensional in every degree.
\end{prop}
\anibal{Does this really needs field coefficients?} 
\begin{proof}
	By the universal coefficient theorem for homology, trivial singular homology with integer coefficients also implies trivial singular homology for any other coefficient group. In particular, this means that $X$ being $HLC^{\infty}$ also implies that $X$ is $HLC^{\infty}_{\mathbb{F}}$. By the universal coefficient theorem for cohomology, the singular cohomology with coefficients in $\mathbb{F}$ is naturally isomorphic to the dual space of the singular homology with coefficients in $\mathbb{F}$. This means that being $HLC^{\infty}_{\mathbb{F}}$ implies being locally connected with respect to singular cohomology in all degrees with coefficients in $\mathbb{F}$. Since we assume $X$ to be $HLC^{\infty}$, \cite[Theorem III.1.1]{MR1481706} now implies that $X$ is locally connected with respect to sheaf cohomology with coefficients in the locally constant sheaf with value $\mathbb{F}$. \cite[Corollary II.17.7]{MR1481706} implies that the sheaf cohomology of the whole space with coefficients in this sheaf is finite-dimensional, so again by \cite[Theorem III.1.1]{MR1481706} the same holds for singular cohomology with coefficients in $\mathbb{F}$. Applying the universal coefficient theorem once more proves the claim.
\end{proof}

Again, we get a persistent version.

\begin{cor}
	Let $(X, f)$ be a sublevel filtration. If each sublevel set is compact and $\HLC$
	Then its singular persistence module is p.f.d. in every degree.
\end{cor}
