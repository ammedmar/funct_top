\immediate\write18{bibtex \jobname}
\documentclass{amsart}
\usepackage{amsmath, amsthm, amssymb}
\usepackage[colorlinks=true,linkcolor=blue,urlcolor=blue]{hyperref}
\usepackage[capitalize,noabbrev]{cleveref}
\usepackage{tikz-cd}
%\usepackage{newtxtext,newtxmath}
\usepackage[numbers,sort,compress]{natbib}
\usepackage{graphicx}
\usepackage{csquotes}

%To-do
\usepackage[marginpar]{todo}
\makeatletter
\renewcommand{\@tododisplay}[1]{%
%\marginpar{#1}%
\textsuperscript{#1}%
}
%
\renewcommand\@displaytodo[2][\todomark]{%
\@tododisplay{{\todoformat [#1~\ref{todolbl:\thetodo}]}}%
\footnotetext[\thetodo]{\todoformat #1:~#2}%
\global\@todotoks\expandafter{\the\@todotoks\todoitem{#1}{#2}}%
\@todotrue%
}%
\renewcommand\todomark{todo}
\makeatother

% theorems
\theoremstyle{plain}
\newtheorem{thm}{Theorem}[section]
\Crefname{thm}{Theorem}{Theorems}
\newtheorem{cor}[thm]{Corollary}
\newtheorem{lem}[thm]{Lemma}
\newtheorem{prop}[thm]{Proposition}
\Crefname{prop}{Proposition}{Propositions}
\newtheorem{claim}[thm]{Claim}
\newtheorem{conj}[thm]{Conjecture}
\Crefname{conj}{Conjecture}{Conjectures}
\theoremstyle{definition}
\newtheorem{defi}[thm]{Definition}
\newtheorem{ex}[thm]{Example}
\newtheorem{rem}[thm]{Remark}

% commands
\newcommand{\N}{\mathbb{N}}
\newcommand{\Z}{\mathbb{Z}}
\newcommand{\R}{\mathbb{R}}
\newcommand{\HS}{\mathrm{HS}}
\newcommand{\HT}{\mathrm{HT}}
\newcommand{\HLC}{\mathrm{HLC}}
\newcommand{\HE}{\mathbb{H}^d}
\renewcommand{\H}{\widetilde{H}}
\DeclareMathOperator{\interior}{int}
\DeclareMathOperator{\im}{im}
\newcommand\CH{\check{H}}
\newcommand{\Mod}[1]{#1-\mathbf{Mod}}


% margins
\setlength{\textwidth}{\paperwidth}
\addtolength{\textwidth}{-1.6in}
\setlength{\textheight}{\paperheight}
\addtolength{\textheight}{-1.8in}
\calclayout

\newcommand{\anibal}[1]{\textcolor{blue}{[Anibal: #1]}}

\begin{document}

\title[Persistence in functional topology]{Persistence in functional topology: local-connectivity, q-tameness and a correction to a theorem of Morse}
\author{}
\date{\today}

\begin{abstract}
	During the 1930's Morse developed a general theory, which he named functional topology, relating the critical set of a semi-continuous functional and the topology of its sublevel sets; and, in joint work with Tompkins, used this body of work to study minimal surfaces.
	Many of the key insights of functional topology have been rediscovered in the study of persistence homology.
	In particular, conditions enforcing the existence of an interval decomposition are as central to functional topology as they are to persistence theory.
	With modern applications to geometry in mind, in this article we present a comprehensive and up-to-date study of local-connectivity for Morse filtrations, the most delicate of these conditions, and provide in passing a correction to the Critical Surface Theorem of Morse and Tompkins.
\end{abstract}


\maketitle
\tableofcontents

\section{Introduction}

The interplay between the critical set of a function and the topology of its domain is a cornerstone of modern mathematics.
Nowadays, when thinking about the pioneering work of Marston Morse, our first thought probably involves a differentiable function on a closed smooth manifold, but more general settings should also be considered.
Morse theory in the smooth context was masterfully presented in Milnor's famous book on the subject \cite{Milnor.1963}, where he also gave a new proof of Bott's periodicity by applying Morse theory to the energy functional of paths in a Riemannian manifold, which notably goes beyond the compact setting.
Another important example of the use of Morse's insights in an infinite context is Floer's work on the Arnold conjecture and its many ramifications in symplectic topology, as surveyed for example in \cite{Salamon.1999}.
Morse himself worked in a very general setting, publishing in the 1930s a pair of papers \cite{Morse.1937, Morse.1940} and a monograph \cite{Morse.1938} in which he established the key results of Morse theory in the broad context defined by semi-continuous functionals on metric spaces.
He called the theory set forth in this body of work \emph{functional topology} and used it to study questions about minimal surfaces motivated by Douglas' solution to Plateau’s Problem \cite{Douglas.1931}.
In particular, Morse and Tompkins \cite{Morse.1939, Morse.1941} used these techniques to prove a general form of \emph{Mountain Pass Theorem} --~an existence result for saddle points~-- applying to functions that are not necessarily continuous.
From this, they deduce their \emph{Unstable Minimal Surface Theorem}, showing the existence of critical points of the Douglas functional that are not local minima.

Morse's work on functional topology did not have a long lasting impact on minimal surface theory or the calculus of variations in general; possibly in part because, as expressed by Struwe:
\begin{displaycquote}[p.~82]{Struwe.1988}
	The technical complexity and the use of a sophisticated topological machinery [...] tend to make Morse--Tompkins' original paper unreadable and inaccessible for the non-specialist.
\end{displaycquote}
A similar assessment was given by Raoul Bott, who writes in \cite[p.~934]{Bott.1980} that the papers \cite{Morse.1937, Morse.1940} ``are not easy reading'' and constitute a ``tour de force'' by Morse.

The intricacies of Morse's development notwithstanding, many of his ideas have resurfaced in the intervening years and flourished in other domains.
In particular, in applied topology and topological data analysis, several key insights of Morse have been independently rediscovered as part of the development of \emph{persistent homology}, a technique that provides robust and efficiently computable invariants of filtered spaces using the functorial properties of homology.
Its success in applied topology has motivated a refined abstract theory of persistence that lies in the intersection of geometry, topology, and representation theory.

The homology of a filtered space is an example of what is called a \emph{persistence module}, a functor to vector spaces from the real numbers considered as a poset category.
In many important cases, a persistence module $M$ admits an essentially unique decomposition into indecomposable direct summands, and the structure of this decomposition yields a complete discrete invariant of $M$ known as its \emph{persistence diagram}.
The set of all persistence diagrams can be organized into a metric space.
This often allows to recast geometric questions about general filtered spaces in a combinatorial metric model, since the passage via the homology construction to this metric space is Lipschitz, a statement commonly known as the \emph{stability} of persistence diagrams.

The most remarkable connections between functional topology and persistence theory come from Morse's paper \cite{Morse.1940}, where he developed the theory of \emph{caps} and their \emph{spans}.
They capture much of the same information as the modern notion of persistence diagram, including concepts such as the persistence or birth and death of a homology class, although Morse's results still fall short of yielding global decompositions of persistence modules.
Morse used his theory of caps to study functionals on a metric space by analyzing the evolution of the topology of their sublevel sets.
A key tool for this end is a version of his eponymous inequalities for cap numbers, which expands their usual version in the compact and smooth setting.
In this work, using persistence diagrams, we generalize the definition of these cap numbers to persistence modules and prove the existence of Morse inequalities for a large class of them (\cref{t:inequalities}).
Our approach makes these inequalities accessible in new contexts beyond those originally covered by functional topology, including, for example, symplectic geometry.

Given the importance of persistence diagrams for stating and proving Morse inequalities, our focus will then be on the study of topological properties ensuring their existence for a broad class of filtered spaces and homology constructions.
For general persistence modules, a well studied condition for the existence of persistence diagrams is \emph{q-tameness} \cite{Chazal.2016a,Chazal.2016b}, which simply states that all linear maps between different real values in the persistence module have finite dimensional rank.
The motivating question can then be reformulated as asking for topological conditions on a filtered space that ensure its persistent homology to be q-tame.
In fact, the problem of finding such conditions was already considered by Morse, who studied certain local connectivity properties of a filtered metric space and claimed them to suffice for the q-tameness of the associated persistent \v{C}ech homology.
Contrary to these claims, we show that the local connectivity condition used by Morse and Tompkins in the proof of the Unstable Minimal Surface Theorem is insufficient (\cref{c:counterexample}).
We introduce two similar but more widely applicable conditions, and show that the first suffices for the \mbox{q-tameness} of filtrations defined by continuous functionals (\cref{c:q-tameness for continuous functions}), and the second for those defined by functionals with compact sublevel sets (\cref{t:local connectedness implies q-tameness}).
Since Douglas' functional has compact sublevel sets but is not continuous in the $C^0$ topology considered by Morse and Tompkins, we use this second condition to correct their proof.

\subsection*{Summary}

The primary contribution of this work consists in the introduction of topological conditions on a broad class of filtered spaces ensuring their associated persistent homology modules satisfy Morse inequalities.
As an application of our results, we identify and fix a mistake in the proof the Unstable Minimal Surface Theorem of Morse and Tompkins.

\subsection*{Outline}

In \cref{s:persistence} we recall the foundations of persistence theory where, for us, the persistent homology of a sublevel set filtration is the key example.
We present a persistence-theoretic point of view on Morse inequalities in \cref{s:inequalities}.
It generalizes both their versions in the smooth and compact setting as well as the one used in functional topology.
The core of this work is presented in \cref{s:connectivity}, where we define two natural notions of local-connectivity for a sublevel set filtration and show under what circumstances they imply q-tameness for its associated persistence module.
We close in \cref{s:surfaces} with a historical overview of Morse--Tompkins' use of functional topology in minimal surface theory, and explore its relation to our results.
\cref{s:vietoris} contains a brief discussion on the definitions of Vietoris and \v{C}ech homology and a comparison between them based on Dowker's Theorem.

\section{Homology theories} \label{s:homology}

\begin{itemize}
\item Definition (generalized homology theory, without dimension axiom, but with Milnor's additivity axiom)
\item Examples, main: singular with integer coeff and cech with field \cite{Kelly.1961}
\item mention Dowker duality
\item Mayer-Vietoris
\end{itemize}
As special instances of homology theories, we will consider the well-known singular theory, as well as the \v{C}ech and Vietoris theories that we will recall next.

Let $(X_{1},X_{2})$ be a pair of topological spaces. We denote by $\Cov(X_{1},X_{2})$ the set of covers of $(X_{1},X_{2})$ directed by the refinement relation. 

If $\alpha=(\alpha_{1},\alpha_{2})$ is a cover of $(X_{1},X_{2})$, we write $\Nrv(\alpha)$ for the simplicial pair $(\Nrv(\alpha_{1}),\Nrv(\alpha_{2}))$, where 
\[
\Nrv(\alpha_{i})=\left\{\beta\subseteq\alpha_{i}\mid\beta\text{ is finite and }\bigcap_{U\in\beta} U\neq\emptyset\right\}.
\]
denotes the nerve of a cover. The nerve construction defines a functor from $\Cov(X_{1},X_{2})$ regarded as a category to the category of simplicial pairs. 

If $D\subseteq\Cov(X_{1},X_{2})$ is a directed set with respect to refinement and $G$ is an abelian group, we define
\[
\CH_{*}^{D}(X_{1},X_{2};G)=\lim_{\alpha\in D}H_{*}(\Nrv(\alpha);G).
\]
The \emph{\v{C}ech homology with coefficients in $G$} of $(X_{1},X_{2})$ is defined as
\[
\CH_{*}(X_{1},X_{2};G)=\CH_{*}^{\Cov(X_{1},X_{2})}(X_{1},X_{2};G).%=\lim_{\alpha\in\Cov(X_{1},X_{2})}H_{*}(\Nrv(\alpha);G)
\]
If $(X_{1},X_{2})$ is a pair of metric spaces, its \emph{metric \v{C}ech homology with coefficients in $G$} is defined as
\[
\MCH_{*}(X_{1},X_{2};G)=\CH_{*}^{\Balls(X_{1},X_{2})}(X_{1},X_{2};G),%=\lim_{\alpha\in\Balls(X_{1},X_{2})}H_{*}(\Nrv(\alpha);G).
\]
where $\Balls(X_{1},X_{2})=\{((B_{\delta}(x))_{x\in X_{1}},(B_{\delta}(x))_{x\in X_{2}})\mid\delta>0\}\subseteq\Cov(X_{1},X_{2})$.

If $\alpha=(\alpha_{1},\alpha_{2})$ is a cover of $(X_{1},X_{2})$, we write $\Vietoris(\alpha)$ for the simplicial pair $(\Vietoris(\alpha_{1}),\Vietoris(\alpha_{2}))$, where 
\[
\Vietoris(\alpha_{i})=\left\{\sigma\subseteq X_{i}\mid\sigma\text{ is finite and there is}U\in\alpha_{i}\text{ with }\sigma\in U\right\}.
\]
denotes the Vietoris complex of a cover. The Vietoris complex construction defines a functor from $\Cov(X_{1},X_{2})$ regarded as a category to the category of simplicial pairs. 

If $D\subseteq\Cov(X_{1},X_{2})$ is a directed set with respect to refinement and $G$ is an abelian group, we define
\[
\VH_{*}^{D}(X_{1},X_{2};G)=\lim_{\alpha\in D}H_{*}(\Vietoris(\alpha);G).
\]
The \emph{Vietoris homology with coefficients in $G$} of $(X_{1},X_{2})$ is defined as
\[
\VH_{*}(X_{1},X_{2};G)=\VH_{*}^{\Cov(X_{1},X_{2})}(X_{1},X_{2};G).%=\lim_{\alpha\in\Cov(X_{1},X_{2})}H_{*}(\Vietoris(\alpha);G)
\]
If $(X_{1},X_{2})$ is a pair of metric spaces, its \emph{metric Vietoris homology with coefficients in $G$} is defined as
\[
\MVH_{*}(X_{1},X_{2};G)=\VH_{*}^{\Balls(X_{1},X_{2})}(X_{1},X_{2};G).%=\lim_{\alpha\in\Balls(X_{1},X_{2})}H_{*}(\Vietoris(\alpha);G).
\]

\v{C}ech and Vietoris homology, as well as their metric versions, can be made into functors and have boundary operators; we omit the straightforward constructions. The metric Vietoris theory is what Morse uses in Functional Topology. As we will see, all of the aforementioned theories agree for the spaces we are interested in. To see this, we use a special case of Dowker's Theorem.

\begin{thm}[{{\cite{Dowker.1952}}}]
Let $(X,A)$ be a pair of topological spaces, $\alpha\in\Cov(X_{1},X_{2})$ and $G$ an abelian group. There exists an isomorphism 
\[
H_{*}(\Nrv(\alpha);G)\cong H_{*}(\Vietoris(\alpha);G),
\]
which is natural with respect to refinement and compatible with the boundary operators.
\end{thm}

\begin{cor}
\v{C}ech and Vietoris homology with arbitrary coefficients agree for pairs of topological spaces. Metric \v{C}ech and metric Vietoris homology with arbitrary coefficients agree for metric pairs.
\end{cor}

Note that when $D_{1},D_{2}\subseteq\Cov(X_{1},X_{2})$ are directed subsets such that one of them is coinital in the other, we have natural isomorphisms $\CH_{*}^{D_{1}}(X_{1},X_{2};G)\cong\CH_{*}^{D_{2}}(X_{1},X_{2};G)$ and $\VH_{*}^{D_{1}}(X_{1},X_{2};G)\cong\VH_{*}^{D_{2}}(X_{1},X_{2};G)$. If $X_{1}$ and $X_{2}$ are compact metric spaces, $\Balls(X_{1},X_{2})$ is coinitial in $\Cov(X_{1},X_{2})$. This yields the following.

\begin{cor}
\v{C}ech and metric \v{C}ech homology with arbitrary coefficients, as well as Vietoris and metric Vietoris homology with arbitrary coefficients, agree for compact metric pairs.
\end{cor}

The previous comparison results justify that from now on, we will use \v{C}ech and Vietoris homology interchangeably.

In full generality, none of of the previously defined functors is an actual homology theory in the sense of the Eilenberg-Steenrod axioms because their long sequences associated to a pair are generally only chain complexes and not necessarily exact. We refer to \cite[Section IX--X]{MR0050886} for verifications of the other axioms. For vector space coefficients and compact pairs, the long sequences are indeed exact (\cite{Kelly.1961}), so we get the following.

\begin{thm}
\v{C}ech homology with coefficients in a vector space $V$ over some field $\mathbb{F}$ defines a homology theory for compact pairs.
\end{thm}

Another important property of \v{C}ech homology, and ultimately the reason why Morse uses it, is its compatibility with inverse limits.

\begin{thm}[{\cite[Chapter VIII, Theorem 3.6.~and Chapter X, Theorem 3.1.]{MR0050886}}]
\v{C}ech homology commutes with inverse limits of compact Hausdorff pairs, i.e., if $(X_{1,i},X_{2,i})_{i}$ is an inverse system of compact Hausdorff pairs, the inverse limit $\lim_{i}(X_{1,i},X_{2,i})$ taken in the category of pairs of topological spaces is again a compact Hausdorff pair and for any coefficient group $G$ the natural map
\[
\CH_{*}(\lim_{i}(X_{1,i},X_{2,i});G)\to\lim_{i}\CH_{*}(X_{1,i},X_{2,i};G)
\]
is an isomorphism.
\end{thm}

\todo{Uli: I saw a discussion about the topological invariance of Vietoris homology for non-compact metric spaces.
I thought that it suggested that invariance holds only for compact spaces,
but I can't find it anymore. 

What I found was this:
\url{http://pldml.icm.edu.pl/pldml/element/bwmeta1.element.zamlynska-8f369e96-b1eb-4ec9-99e2-8c366367a5a4} claims invariance also for non-compact.
\url{https://www.ams.org/journals/tran/1947-062-02/S0002-9947-1947-0024128-6/S0002-9947-1947-0024128-6.pdf} mentions a non-invariant property "null-bounding"

Update: Cech homology (as usually defined) and Vietoris homology (as defined originally) are only isomorphic for compact metric spaces. In our setting, this is enough. Decide which homology theories are really relevant to us.
Vietoris (metric),
Vietoris (topological, i.e. as the limit over all open covers);
eqivalently (by Dowker duality):
Cech (metric),
Cech (topological);
(Massey also defined Alexander-Spanier homology \url{https://doi.org/10.2307/2321782}. It should be equivalent to the "topological" variants here.)
Singular;
}



\section{Filtered spaces and persistence theory} \label{s:persistence}

In this section we present an overview of the theory of persistence as it is used in geometric contexts, through filtrations by sublevel sets of real-valued functions.
For a detailed exposition we refer to \cite{polterovich2020topological} and \cite{Chazal.2016a, MR3408277}.

The ``pipeline'' of topological persistence traverses through geometry, algebra and discrete mathematics as follows:
Given a space $X$ filtered by the sublevel sets of a function $f$, the application of some homology theory in degree $n$ with coefficients in a field,  or, more generally, a functor from topological spaces to vector spaces, produces a \textit{persistence module}, an algebraic object equipped with a structure theory leading in favorable cases to a powerful discrete invariant called \textit{persistence diagram}.

These invariants are key to applications.
For example, in the next section we will see how they lead to a generalization of the classical Morse inequalities.
Much of their usefulness comes from a remarkable geometric fact known as stability, stating that the map from functions on $X$ with the supremum norm to the set of all persistence diagrams is 1-Lipschitz with respect to a certain natural metric, called the \emph{bottleneck distance}.

To explain the meaning of the above terms in detail, we consider a space $X$ and a function $f \colon X \to \R$.
Unless noted otherwise, the functions we consider need not be continuous.
We pass to filtered spaces by considering the \textit{sublevel set filtration $f_{\leq \bullet}$ of $X$ induced by $f$}, which is defined by
\begin{equation*}
f_{\leq t} = f^{-1}(-\infty, t].
\end{equation*}
We say that this filtration is \textit{compact} if $X$ is a locally compact Hausdorff space and all sublevel sets are compact.

For the next step in the persistence pipeline, we will need a coherent assignment of a vector space to any topological space, more precisely, a functor from $\Top$ to $\Vect$.
The key examples are provided by \emph{homology theories}, which for now are only assumed to be $\Z$-graded families $\H = (\H_d)_{d \in \Z}$ of \emph{homotopy invariant functors}, meaning that they assign the same morphism to homotopic maps.
Of particular importance to us are \v{C}ech homology \cite[Section IX--X]{MR0050886} with coefficients in some fixed field $\F$, and homology theories in the sense of Eilenberg--Steenrod \cite[Section I]{MR0050886}, such as singular homology, again with coefficients in some field $\F$ \cite{Eilenberg.1944}.

By definition, applying to a filtered space $\{X_t\}_{t \in \R}$ a functor from $\Top$ to $\Vect$ yields for every $t \in \R$ a vector space~$M_t$ and for any pair~$s, t \in \R$ with~$s \leq t$ a linear map~$M_{s,t} \colon M_s \to M_t$ such that $M_{t,t}$ is the identity and the composition $M_{s,t} \circ M_{r,s}$ is equal to $M_{r,t}$ for any triple $r \leq s \leq t$.
In other words, we obtain a functor from the real numbers, considered as a poset category, to the category of vector spaces.
Such functors will be called \emph{persistence modules}.
A morphism of persistence modules $\varphi \colon M \to N$ is a natural transformation, i.e., an assignment of a linear map $\varphi_t \colon M_t \to N_t$ for every $t \in \R$ making the diagram
\begin{equation*}
    \begin{tikzcd}
    M_{s} \arrow[r, "M_{s,t}"] \arrow[d] & M_{t} \arrow[d] \\
    N_{s} \arrow[r, "N_{s,t}"] & N_{t}
    \end{tikzcd}
\end{equation*}
commute for all pairs $s \leq t$.

Usual constructions that work in the category of vector spaces over the field $\F$ can be transferred to the category of persistence modules by applying them pointwise.
For example, the kernel and cokernel of a morphism, as well as the direct sum of persistence modules are well-defined.
Persistence modules that are \emph{indecomposable}, i.e., those that have only trivial direct sum decompositions, play an important role in persistence theory.
A rich family of indecomposable persistence modules is given by \emph{interval modules}, which for an interval $I \subseteq \R$ are defined by
\begin{equation} \label{e:interval module}
    C(I)_t =
    \begin{cases}
        \mathbb{F} & \text{if } t \in I, \\
        0          & \text{otherwise},
    \end{cases}
    \qquad
    \qquad
    C(I)_{s, t} =
    \begin{cases}
        \operatorname{id}_{\mathbb{F}} & \text{if } s, t \in I,\\
        0 & \text{otherwise}.
    \end{cases}    
\end{equation}
These indecomposable interval modules can be used as building blocks for \emph{barcode modules}, which are direct sums of interval modules.
% \[
% \bigoplus_{\lambda \in \Lambda} C(I_{\lambda}).
% \]
The multiset of intervals $\{I_{\lambda}\}_{\lambda \in \Lambda}$ associated to a barcode module is known as its \textit{barcode}. By a version of the Krull--Remak--Schmidt--Azumaya Theorem \cite{MR37832} (see also \cite[Theorem 2.7]{Chazal.2016a} for a specialization to barcode modules), two isomorphic barcode modules have the same barcodes up to a choice of the index set $\Lambda$. Thus, if a persistence module $M$ is isomorphic to a barcode module, the associated barcode is a complete invariant of $M$, and the (non-unique) isomorphism to the barcode module is called a \emph{barcode decomposition}.

We want to use the passage from persistence modules to invariants obtained from decompositions as the last step in our pipeline, so we have to understand which persistence modules admit barcode decompositions.
The most commonly used existence result for barcode decompositions is due to Crawley-Boevey's theorem \cite{Crawley-Boevey.2015}. It guarantees the existence of a barcode decomposition for any \emph{pointwise finite dimensional ($\PFD$)} persistence module, which is a persistence modules $M$ such that $M_t$ is a finite dimensional vector space for all $t \in \R$.

Unfortunately, the $\PFD$ condition is too restrictive for our purposes. In particular, it is not necessarily satisfied in the historical setup of Morse and Tompkins' work on minimal surfaces. In this context, the persistence modules $M$ only have the slightly weaker property of being \emph{q-tame}, which is defined by the rank of the maps $M_{s,t} \colon M_s \to M_t$ being finite for all $s < t$  \cite{Chazal.2016a}. 
Not every q-tame persistence module admits a barcode decomposition in the above sense, as exemplified by the infinite product of interval modules $\prod_{n \in \N_{> 0}} C([0,1/n))$. Yet, there are multiple ways of working around this to still get discrete invariants in the spirit of barcodes \cite{Chazal.2016a, Chazal.2016b, schmahl2020structure}. We briefly recall the approach from \citet{Chazal.2016b}.

A persistence module $K$ is called \emph{ephemeral} if the maps $K_{s,t} \colon K_s \to K_t$ are $0$ for all $s < t$.
The \emph{radical} $\rad M$ of a persistence module $M$ is defined as the unique minimal submodule of $M$ such that the cokernel of the inclusion $\rad M \to M$ is an ephemeral persistence module.
Specifically, $(\rad M)_t = \sum_{s<t}\im M_{s,t}$.
If $M$ is q-tame, then its radical admits a barcode decomposition \cite[Corollary~3.6]{Chazal.2016b},
with the associated barcode describing the isomorphism type of~$M$ ``up to ephemerals".
This can be formalized by constructing the \emph{observable category of persistence modules}, which is equivalent to the quotient of the category of persistence modules by the full sub-category of ephemeral persistence modules.
Intuitively, one may think of the observable category as forgetting all information in persistence modules that does not persist over a non-zero amount of time.
The barcode of the radical of a q-tame persistence module $M$ is then a complete invariant of $M$ in the observable category.

Instead of talking about the barcode of the radical of a q-tame persistence module, it will be more convenient for our treatment of Morse inequalities to talk about the \emph{(undecorated) persistence diagram} associated to a barcode $\{I_{\lambda}\}_{\lambda \in \Lambda}$, which is the multiset defined by the \emph{multiplicity function} $\mathfrak{m} \colon \multiplicityDomain \to \mathbb{N}$ that associates to an element in
\[
\multiplicityDomain =
\big\{ (p,q) \mid p \in \R \cup \{-\infty\}, \ q \in \R \cup \{+\infty\}, \ p < q \big\}
\]
the cardinality of the set $\{ \alpha \in A \mid \inf I_{\alpha} = p,\ \sup I_{\alpha} = q\}$.

Two q-tame barcode modules are observably isomorphic if and only if their persistence diagrams agree, so the persistence diagram associated to the barcode of $\rad M$ is still a complete invariant of the q-tame persistence module $M$ in the observable category.
We summarize:

\begin{thm}[{{\citet{Chazal.2016a,Chazal.2016b}}}] \label{thm:q-tame modules have barcodes}
Every q-tame persistence module has a unique persistence diagram that completely describes its isomorphism type in the observable category.
\end{thm}

We have thus seen how to obtain a persistence diagram from a real-valued function $f$ by applying a functor $\H \colon \Top \to \Vect$ to its sublevel set filtration and considering the persistence diagram of the resulting persistence module $\H(f_{\leq \bullet})$, which is well-defined provided that $\H(f_{\leq \bullet})$ is q-tame. If this is the case, we will call the function \emph{q-tame} with respect to the functor $\H$. If $\H$ is some kind of homology theory, we will call $\H(f_{\leq \bullet})$ the \emph{persistent homology} of the filtration $f_{\leq \bullet}$.

As mentioned earlier, the passage from functions to persistence diagrams is 1-Lipschitz for appropriate metrics \cite{MR3333456}. The metric of choice to consider on the space of $\R$-valued functions on $X$ is the distance induced by the supremum norm. The metric of choice on the space of persistence diagrams is the \emph{bottleneck distance}, which expresses the distance between persistence diagrams by an optimal matching of the intervals in the corresponding barcodes: two diagrams are within distance $\delta$ if any two matched intervals are within Hausdorff distance $\delta$, and any unmatched interval has length at most $2\delta$.

The most general stability result is shown by considering as an intermediate step the \emph{interleaving distance} for filtrations and persistence modules \cite{MR2279866}, which measures how far two $\R$-indexed diagrams $M,N$ are from being isomorphic.
A $\delta$-interleaving consists of a pair of natural transformations $(M_t \to N_{t+\delta})_t, (N_t \to M_{t+\delta})_t$ from one diagram to a shifted version of the other and vice versa, which compose to the internal structure maps $(M_t \to M_{t+2\delta})_t, (N_t \to N_{t+2\delta})_t$. Clearly, the case $\delta=0$ describes an isomorphism, and the infimum $\delta$ admitting a $\delta$-interleaving is the interleaving distance between the two diagrams.
As it turns out, the stability of persistence barcodes can then be described as a sequence of 1-Lipschitz transformations: from functions (with the supremum norm) to filtrations by sublevel sets (with the interleaving distance on diagrams of topological spaces), to persistent homology (with interleaving distance on persistence modules), and finally to persistence diagrams (with the bottleneck distance).
In this approach, by far the most difficult step is showing that passing from persistence modules to persistence diagrams is stable, a result which is known as the Algebraic Stability Theorem  \cite{10.1145/1542362.1542407,Chazal.2016a,MR3333456}.


\section{Persistence diagrams from local connectivity} \label{s:connectivity}

We have seen that q-tame functions admit persistence diagrams, which can be used to formulate Morse inequalities.
With q-tameness being a rather abstract condition, we now establish more concrete topological conditions that ensure the q-tameness of a function.
Our definitions are motivated by similar conditions that Morse considered in his work on functional topology.
We will present a historical account in \cref{s:surfaces}.

Whether a function is q-tame or not depends on the functor that is used to pass from the sublevel set filtration of the function to a persistence module.
The general idea that we want to employ is to deduce global finiteness properties like q-tameness from local ones.
Thus, the functors we consider should have some property that allows us to do so.

Concretely, let $\H = (\H_d)_{d \in \Z} \colon \Top \to \Vect$ be a fixed graded homotopy invariant functor, which we call a \emph{homology theory}.
A triple of spaces $X_{1}, X_{2} \subseteq X$ is said to have a \emph{Mayer-Vietoris sequence} for $\H$ if the inclusion-induced maps fit in a long exact sequence
\begin{equation*}
\begin{tikzcd}[column sep = small]
\cdots \arrow[r] & \H_{n}(X_{1} \cap X_{2}) \arrow[r] & \H_{n}(X_{1}) \oplus \H_{n}(X_{2}) \arrow[r] & \H_{n}(X) \arrow[r] & \H_{n-1}(X_{1} \cap X_{2}) \arrow[r] & \cdots \ .
\end{tikzcd}
\end{equation*}
We say that $\H$ has the \emph{open} (resp.\@\  \emph{compact}) \emph{Mayer-Vietories property} if there are natural Mayer-Vietoris sequences for all triples $X_{1}, X_{2} \subseteq X$ with $X = X_1 \cup X_2$ and $X_i \subseteq X$ open (resp.\@\ compact Hausdorff).

For the rest of this section, we will assume that $\H$ is a homology theory that has either the open or the compact Mayer-Vietoris property and for which there is $n_0$ such that $\H_{n}$ is 0 for all $n \leq n_0$.
Note that this includes singular homology with field coefficients, which, like any homology theory in the sense of the Eilenberg-Steenrod axioms \cite[Section I]{MR0050886}, has the open Mayer-Vietoris property, and it also includes \v{C}ech homology with field coefficients, which has the closed Mayer-Vietoris property.

\begin{defi} \label{defi:local_connectedness}
	For $n \in \Z$ a continuous map is said to be \textit{$n$-homologically small} or $\HS_n$ if the image of the map induced by $\H_{n}$ is finite dimensional.
	We omit references to $n$ if the condition holds for all integers.
\end{defi}

\begin{defi} \label{defi:local_connectedness_filtrations}
	The sublevel set filtration of a function $f \colon X \to \R$ is said to be \emph{locally homologically small} or $\HLC$ if for any $x \in X$, any neighborhood $V$ of $x$, and any index $t > f(x)$, there is an index $s$ with  $f(x) < s < t$ and a neighborhood $U$ of $x$ with $U \subseteq V$ such that the inclusion $f_{\leq s} \cap U \to f_{\leq t} \cap V$ is $\HS$.
\end{defi}
Similarly, we have the following stronger version of this property.
\begin{defi}
	The sublevel set filtration is called \emph{strongly locally homologically small $\HLC$} or \emph{strongly $\HLC$} if for any $x \in X$, any neighborhood $V$ of $x$, and any pair of indices $s,t$ with $f(x) < s < t$ there is a neighborhood $U$ of $x$ with $U \subseteq V$ such that the inclusion $f_{\leq s} \cap U \to f_{\leq t} \cap V$ is $\HS$.
\end{defi}

Clearly, any strongly $\HLC$ sublevel set filtration is also $\HLC$.
If the filtration is induced by a continuous function, the converse also holds as the following theorem shows.

\begin{thm}\label{thm:hlc_to_strong_hlc}
    If the sublevel set filtration of a continuous function $f \colon X \to \R$ is $\HLC$, then it is also strongly $\HLC$.
\end{thm}
\begin{proof}
    Fix $x \in X$, a neighborhood $V$ of $x$ and indices $f(x) < s < t$.
	We need to show that there is a neighborhood $U \subseteq V$ of $x$ such that the inclusion $f_{\leq s} \cap U \to f_{\leq t} \cap V$ is $\HS$.
    
    To do so, we start by using the $\HLC$ property to choose a neighborhood $U' \subseteq V$ of $x$ and an index $s' \in (f(x),\, t)$ such that the inclusion $f_{\leq s'} \cap U' \to f_{\leq t} \cap V$ is $\HS$.
    Now, we choose $U = f_{< s'} \cap U'$, where $f_{< s'} = f^{-1} (-\infty, s')$.
    Note that this choice of $U$ still defines a neighborhood of $x$ because $f$ is assumed to be continuous, so that $f_{< s'}$ is an open subset of $X$.
    
    We obtain that $f_{\leq s} \cap U \subseteq f_{\leq s'} \cap U'$, so that the inclusion $f_{\leq s} \cap U \to f_{\leq t} \cap V$ factors through the inclusion $f_{\leq s'} \cap U' \to f_{\leq t} \cap V$.
	This second map is $\HS$ by construction.
	Any map that factors through an $\HS$ map is also $\HS$, so the proof is finished.
\end{proof}

As our main result, we will now show that for compact sublevel set filtrations the strong $\HLC$ condition implies \mbox{q-tameness}, and consequently also implies the existence of a persistence diagram which yields persistent Morse inequalities.

\begin{thm} \label{t:strong local connectedness implies q-tameness}
	If the sublevel set filtration of a function $f \colon X \to \R$ is compact and strongly $\HLC$, then it is also q-tame.
\end{thm}

The general proof strategy is inspired by the proof of a result attributed to Wilder as presented by Bredon \cite[Section II.17]{Bredon.1997}.
We collect the main ideas in three lemmas.

\begin{lem} \label{l:commutative algebra}
	Given a commutative diagram of modules over a principal ideal domain
	\begin{equation*}
	\begin{tikzcd}
	A_{1,1} \arrow[r] & A_{1,2} & \\
	A_{2,1} \arrow[r] \arrow[u] & A_{2,2} \arrow[r] \arrow[u] & A_{2,3} \\
	& A_{3,2} \arrow[r] \arrow[u] & A_{3,3} \arrow[u]
	\end{tikzcd}
	\end{equation*}
	where the middle row is exact and both $A_{2,1} \to A_{1,1}$ and $A_{3,3} \to A_{2,3}$ have finitely generated images, then so does $A_{3,2} \to A_{1,2}$.
\end{lem}

\begin{proof}
	This is proven via a straightforward diagram chase.
For more details see \cite[Lemma II.17.3]{Bredon.1997}.
\end{proof}

\begin{lem} \label{l:neighborhood third}
	Let $X$ be locally compact space.
	For any compact subset $K$ and open set $U$ with $K \subseteq U$ there exists a compact set $K^\prime$ such that
	\begin{equation*}
	K \subseteq \interior(K^\prime) \subseteq K^\prime \subseteq U.
	\end{equation*}
\end{lem}

\begin{proof}
	For any $x \in K$ choose a compact neighborhood $C(x) \subseteq U$.
	We have
	\begin{equation*}
	K \subseteq \bigcup_K \interior(C(x)) \subseteq \interior\left(\bigcup_K C(x)\right) \subseteq \bigcup_K C(x) \subseteq U
	\end{equation*}
	Since $K$ is compact, the first inclusion above is achieved over a finite subset $\{x_1, \dots, x_m\}$ of elements in $K$.
	Defining $K^\prime = \bigcup_{i=1}^m C(x_i)$ finishes the proof.
\end{proof}

\begin{lem} \label{l:key lemma for q-tameness}
    Let $f \colon X \to \R$ be a function whose sublevel set filtration is compact and strongly $\HLC$, and consider subsets $C \subseteq L \subseteq X$ with $C$ compact and $L$ open.
	For any $s < t$ the inclusion
	\begin{equation*}
	C \cap f_{\leq s} \to L \cap f_{\leq t}
	\end{equation*}
	is $\HS$.
\end{lem}

\begin{proof}
	Recall that we are assuming that the underlying homology theory $\H$ has either the open or the compact Mayer-Vietoris property and that there is some $n_0$ such that $\H_{n}$ is 0 for all $n \leq n_0$.
    The statement of the lemma holds for $\HS_{n}$ in place of $\HS$ for any $n \leq n_0$ since $\H_{n}$ induces the zero map.
    We will proceed by induction on $n \geq n_0$ assuming the statement for $\HS_{n-1}$.

    We define $\Sigma_{s, t}$ to be the collection of all open subsets $V \subseteq X$ whose closure $\overline{V}$ is compact, contained in $L$, and has an open neighborhood $U$ with 
	$\overline{V} \subseteq U \subseteq L$
	for which there exists $s' \in (s,\, t)$ such that the inclusion
    $U \cap f_{\leq s'} \to L \cap f_{\leq t}$
	is $\HS_n$.
	We will show that $\Sigma_{s, t}$ has the following three properties:
	\begin{enumerate}
	    \item Any point $x \in L \cap f_{\leq s}$ has a neighborhood $V_x \in \Sigma_{s,t}$.
	    \item If $V_1,\, V_2 \in \Sigma_{s,t}$, then $V_1 \cup V_2 \in \Sigma_{s,t}$.
	    \item For each $V \in \Sigma_{s,t}$ the inclusion 
	    $V \cap f_{\leq s} \to L \cap f_{\leq t}$ 
	    is $\HS_n$.
	\end{enumerate}
	
	Assuming them for the moment, the first property allows us to cover $C$ by sets $V_x \in \Sigma_{s,t}$, $x \in C$.
	Because $C$ is compact, this cover can be chosen finite, represented by say $x_1,\dots, x_m$.
	By the second property, we have $V = \bigcup_{i = 1}^m V_{x_i} \in \Sigma_{s,t}$.
	Using the third property, the inclusion 
	$V \cap f_{\leq s} \to L \cap f_{\leq t}$ 
	is $\HS_n$, so the inclusion 
	$C \cap f_{\leq s} \to L \cap f_{\leq t}$ 
	is also $\HS_n$ because this map factors through the aforementioned one.
	What is left to do is to show that $\Sigma_{s,t}$ has the properties we want.
	
	For the third property, let $V \in \Sigma_{s,t}$ with $U$ an open neighborhood of $\overline{V}$ in $L$ and $s' \in (s,\, t)$ such that 
	$U \cap f_{\leq s'} \to L \cap f_{\leq t}$
	is $\HS_n$.
	The inclusion
	$V \cap f_{\leq s} \to L \cap f_{\leq t}$
	factors through the previous one so it is $\HS_n$ as well.
	Thus, $\Sigma_{s, t}$ has the third property we want.
	
	Next, we will show using the strong $\HLC$ property that $\Sigma_{s, t}$ has the first required property, i.e., that any point $x \in L \cap f_{\leq s}$ has a neighborhood in $\Sigma_{s, t}$.
	Choose an arbitrary $s' \in (s,\, t)$.
	Since the sublevel set filtration of $f$ is strongly $\HLC$, there is an open neighborhood $U_x \subseteq L$ such that the inclusion
	$U_x \cap f_{\leq s'} \to L \cap f_{\leq t}$
	is $\HS$, so in particular $\HS_n$.
	By local compactness of $X$ we can choose a compact neighborhood $K_x$ of $x$ contained in $U_x$.
	Now $V_x = \interior (K_x)$ is a neighborhood of $x$ with $V_x \in \Sigma_{s,t}$.
	
	Finally, using Mayer-Vietoris and the induction hypothesis we will show that $\Sigma_{s,t}$ has the second required property, i.e., that it is closed under finite unions.
	So for $i \in \{1, 2\}$ let $V_i \in \Sigma_{s,t}$ with $U_i$ and $s'_i \in (s,\, t)$ such that 
	$\overline{V_i} \subseteq U_i \subseteq L$ 
	and
	$U_{i} \cap f_{\leq s'_i} \to L \cap f_{\leq t}$
	is $\HS_n$.
	Writing $K_i = \overline{V_i}$, we use \cref{l:neighborhood third} to construct compact sets $K'_i$ such that
	\begin{equation*}
	V_i \subseteq K_i \subseteq V'_i \subseteq K'_i \subseteq U_i \subseteq L
	\end{equation*}
	where $V'_i = \interior(K'_i)$.
	We have that the union $V_1 \cup V_2 \subseteq L$ is open, its closure $\overline{V_1 \cup V_2}$ is compact, and we have $\overline{V_1 \cup V_2} \subseteq V'_1 \cup V'_2 \subseteq L$.
	Thus, we obtain $V_1 \cup V_2 \in \Sigma_{s,t}$ if we can show that there is an $s' \in (s,\, t)$ such that the inclusion 
	$\left(V'_1 \cup V'_2 \right) \cap f_{\leq s'} \to L \cap f_{\leq t}$
	is $\HS_n$.
	To do so, we set $s'' = \min_i s'_i$ and choose $s' \in (s,\, s'')$.
	For proving that $\left(V'_1 \cup V'_2 \right) \cap f_{\leq s'} \to L \cap f_{\leq t}$ is $\HS_n$ we now distinguish the two cases where $\H$ has either the open or the compact Mayer-Vietoris property.
	
	In the open case, notice that for both choices of $i$ the inclusions
	$U_i \cap f_{\leq s''} \to L \cap f_{\leq t}$
	are $\HS_n$.
	Additionally, the inclusion
	$V'_1 \cap V'_2 \cap f_{\leq s'} \to U_1 \cap U_2 \cap f_{\leq s''}$
	is $\HS_{n-1}$ because it factors through the inclusion
	$K'_1 \cap K'_2 \cap f_{\leq s'} \to U_1 \cap U_2 \cap f_{\leq s''}$,
	which is $\HS_{n-1}$ by the induction hypothesis.
    Because the $V_i$ and $V'_i$ are open and because $\H$ has the open Mayer-Vietoris property, we obtain the following commutative diagram satisfying the assumptions of \cref{l:commutative algebra}:
	\begin{equation*}
	\begin{tikzcd}[column sep=small]
	\H_n(L \cap f_{\leq t}) \oplus \H_n(L \cap f_{\leq t}) \arrow[r] &
	\H_n(L \cap f_{\leq t}) & \\
	\H_{n}(U_1 \cap f_{\leq s''}) \oplus \H_n(U_2 \cap f_{\leq s''}) \arrow[r] \arrow[u] & 
	\H_{n}((U_1 \cup U_2) \cap f_{\leq s''}) \arrow[r] \arrow[u] &
	\H_{n-1}(U_1 \cap U_2 \cap f_{\leq s''}) \\ & 
	\H_{n}((V'_1 \cup V'_2) \cap f_{\leq s'}) \arrow[r] \arrow[u] &
	\H_{n-1}(V'_1 \cap V'_2 \cap f_{\leq s'}). \arrow[u]
	\end{tikzcd}
	\end{equation*}
	We conclude that the inclusion 
	$\left(V'_1 \cup V'_2 \right) \cap f_{\leq s'} \to L \cap f_{\leq t}$ 
	is $\HS_n$, which finishes this part of the proof.
	
	In the compact case, we apply \cref{l:neighborhood third} again to obtain compact sets $K''_i$ such that
	\begin{equation*}
	V_i \subseteq K_i \subseteq V'_i \subseteq K'_i \subseteq V''_i \subseteq K''_i \subseteq U_i \subseteq L
	\end{equation*}
	where $V''_i = \interior(K''_i)$.
	The rest of the proof is then analogous to the previous case:
	We have that for both choices of $i$ the inclusions
	$K''_i \cap f_{\leq s''} \to L \cap f_{\leq t}$
	are $\HS_n$ because this is true for the corresponding inclusions with $K''_i$ replaced by $U_i$.
	Moreover, the inclusion
	$K'_1 \cap K'_2 \cap f_{\leq s'} \to K''_1 \cap K''_2 \cap f_{\leq s''}$
	is $\HS_{n-1}$ because it factors through the inclusion
	$K'_1 \cap K'_2 \cap f_{\leq s'} \to V''_1 \cap V''_2 \cap f_{\leq s''}$,
	which is $\HS_{n-1}$ by the induction hypothesis.
	Because the $K'_i$ and $K''_i$ as well as the sublevel sets of $f$ are all compact and because $\H$ has the compact Mayer-Vietoris property, we obtain the following commutative diagram satisfying the assumptions of \cref{l:commutative algebra}:
	\begin{equation*}
	\begin{tikzcd}[column sep=small]
	\H_n(L \cap f_{\leq t}) \oplus \H_n(L \cap f_{\leq t}) \arrow[r] &
	\H_n(L \cap f_{\leq t}) & \\
	\H_{n}(K''_1 \cap f_{\leq s''}) \oplus \H_n(K''_2 \cap f_{\leq s''}) \arrow[r] \arrow[u] & 
	\H_{n}((K''_1 \cup K''_2) \cap f_{\leq s''}) \arrow[r] \arrow[u] &
	\H_{n-1}(K''_1 \cap K''_2 \cap f_{\leq s''}) \\ & 
	\H_{n}((K'_1 \cup K'_2) \cap f_{\leq s'}) \arrow[r] \arrow[u] &
	\H_{n-1}(K'_1 \cap K'_2 \cap f_{\leq s'}). \arrow[u]
	\end{tikzcd}
	\end{equation*}
	We conclude that the inclusion 
	$\left(K'_1 \cup K'_2 \right) \cap f_{\leq s'} \to L \cap f_{\leq t}$
	is $\HS_n$, so the same is true for the inclusion
	$\left(V'_1 \cup V'_2 \right) \cap f_{\leq s'} \to L \cap f_{\leq t}$ 
	because it factors through the previous one.
\end{proof}

We can now complete the proof of the claim stating that for compact sublevel set filtrations, strong $\HLC$ implies q-tameness.

\begin{proof}[Proof of \cref{t:strong local connectedness implies q-tameness}]
    By definition, the sublevel set filtration of $f$ is q-tame if and only if the inclusion $f_{\leq s} \to f_{\leq t}$ is $\HS$ for all pairs $s < t$.
    Thus, the theorem follows by applying \cref{l:key lemma for q-tameness} with $C = f_{\leq s}$, which is compact by assumption, and $L = X$.
\end{proof}

Combining \cref{thm:hlc_to_strong_hlc} with \cref{t:strong local connectedness implies q-tameness} also yields the following result for continuous functions.

\begin{cor}
    If the sublevel set filtration of a continuous function $f \colon X \to \R$ is compact and $\HLC$, then it is also q-tame.
\end{cor}



%\input{outlook}

\section{Historical remarks} \label{s:historical}

Morse and Tompkins considered the following setting introduced in even greater generality by Douglas:
Let $g=(g_i)\colon\mathbb{R}\to\mathbb{R}^n$ be a simple closed curve such that $g$ is differentiable and such that each $g'_i$ is Lipschitz.
Assume that $g$ is $2\pi$-periodic. Let $\tilde{\Omega}$ be the space of continuous functions $\varphi \colon \mathbb{R} \to \mathbb{R}$ with $\varphi((t)+2\pi) = \varphi(t) + 2\pi$ for all $t$ and such that there are three distinct points $\alpha_i \in [0,2\pi)$ with $\varphi(\alpha_i)=\alpha_i$.
The \emph{Douglas functional} associated to the curve $g$ on $\tilde{\Omega}$ by \begin{equation*}
A_g(\varphi)=\frac{1}{16}\int_0^{2\pi}\int_0^{2\pi}\sin\left(\frac{\alpha-\beta}{2}\right)^{-2} \! \lVert g(\varphi(\alpha))-g(\varphi(\beta)) \rVert_2^2 \ \mathrm{d}\alpha \ \mathrm{d}\beta
\end{equation*}
and set
\begin{equation*}
\Omega_g=\{\varphi\in\tilde\Omega\mid A_g(\varphi)<\infty\}.
\end{equation*}
Douglas proved that, for even more general $g$ than above, $\Omega_g$ is non-empty and that sublevel sets of $A_g$ on $\Omega_g$ are compact. Since $A_g$ is clearly bounded below by $0$, this implies that $A_g$ has a unique global minimizer. Douglas then used harmonic analysis to show that this minimizer of $A_g$ corresponds to a solution of Plateau's Problem.
More generally, each critical point of this functional $A_g$ corresponds to a critical point of the area functional.

In \cite{Morse.1939}, Morse and Tompkins introduce the following topological version of a critical point, which they will use to study the pair $(\Omega_g, A_g)$.

Let $(M,d)$ be a metric space and $f\colon M\to\mathbb{R}$ a function.
A point $p \in M$ is a \emph{homotopically ordinary point of $f$} if there is a neighborhood $U$ of $p$ in $M_{\leq f(p)}$ and a continuous map $H \colon U \times [0,1] \to M$ such that
\begin{enumerate}
	\item $H(q,0) = q$ for all $q$,
	\item $H(p,1) \neq p$,
	\item For $e>0$ there is $\delta>0$ such that $s<s'$ and $d(H(q,s), H(q,s')) > e$ imply $f(H(q,s)) - f(H(q,s')) > \delta$.
\end{enumerate}
A point $p$ is a \emph{homotopically critical point of $f$} if it is not homotopically ordinary.

Roughly, one can think of homotopically critical points of $f$ as points $p$ that have no neighborhood in $f_{\leq f(p)}$ that can be mapped by a homotopy into $M_{\leq t}$ for some $t<f(p)$.

In the case of isolated critical points, one can then choose a neighborhood $U$ of a homotopically critical point $p$ not containing any other critical point and define the \emph{cap type numbers of $p$} as the cap type numbers of the relative homology lattice \anibal{Lattice?} associated to the restriction of $f$ to this neighborhood.
The total cap type numbers are then defined as the sums of all of these cap type numbers. \anibal{intuition for cap type numbers in terms of modern language?}
Assuming a finite number of critical points, Morse and Tompkins then proceed to state Morse inequalities for these total cap type numbers, referring to \cite{Morse.1940}, which was still in preparation at that time, for proof. 

More generally, in the non-isolated case, they consider closed sets of critical points possessing a neighborhood that does not contain any other critical points and call these \emph{critical sets}.
Again, one can then define the cap type numbers of such sets as the cap type numbers of $f$ restricted to the corresponding neighborhood.
In order to get Morse inequalities in this setting, without assuming that the critical points are finite in number, \anibal{One needs q-tameness}.

To achieve q-tameness, Morse wants to use the following condition, which he refers to as \textit{local $F$-connectivity}.
\begin{displaycquote}[p.431]{Morse.1940}
	The space $M$ is said to be \textit{locally $F$-connected} of order $r$ at $p$ if corresponding to each positive constant $e$ there exists a positive constant $\delta$ such that each singular $r$-sphere on the $\delta$-neighborhood of $p$ on $F_{c+\delta}$ bounds an $(r+1)$-cell of norm $e$ on $F_{c+e}$.
\end{displaycquote}
Morse then states that as consequence of bounded compactness and local $F$-connectivity the rank of the persistence Vietoris homology module associated to $(M, F)$ is q-tame.
In the original it reads:
\begin{displaycquote}[Theorem 6.3, p.432]{Morse.1940}
	Let $a$ and $c$ be positive constants such that $a < c < 1$.
	The $k^{\mathrm{th}}$ connectivity $R^k(a,c)$ of $F_a$ on $F_c$ is finite.
\end{displaycquote}
The proof Morse cites, found in \cite[Theorem 6.1]{Morse.1938}, is not correct; neither is the statement.
A stronger notion of local connectivity for the pair $(M, F)$ appeared in print, introduced by Morse himself three years earlier.
\begin{displaycquote}[p.421-422]{Morse.1937}
	The space $M$ will be said to be locally $F$-connected for the order $n$ if corresponding to $n$, an arbitrary point $p$ on $M$, and an arbitrary positive constant $e$, there exists a positive constant $\delta$ with the following property. For $c \geq F(p)$ any singular $n$-sphere on $F \leq c$ (the continuous image on $F \leq c$ of an ordinary $n$-sphere) on the $\delta$-neighborhood $p_{\delta}$ of $p$ is the boundary of a singular $(n + 1)$-cell on $F \leq c + e$ and on $p_e$.
\end{displaycquote}
This notion of local-$F$-connectivity is stronger than the one first quoted, with the key difference being the freedom granted by the constant $c \geq F(p)$.
This is precisely the difference between our notions of $\HLC$ and weakly $\HLC$, presented in Definition~\ref{defi:local_connectedness_filtrations}.
Following the introduction of this stronger definition Morse states as \cite[Theorem~9.2, p.422]{Morse.1937} that q-tameness follows from this form of local $F$-connectivity and bounded compactness, adding that the proof ``while not difficult involves too great detail to be presented here".

%Morse and Tompkins additionally assume local-$f$-connectedness and again refer to \cite{Morse.1940} for proof.
%However, the proofs that Morse presents there repeatedly make use of his erroneous claim that local-$f$-connectedness implies q-tameness, so they cannot be seen as valid.

Let us return now to the Douglas functional on the space $\Omega_g$.
Morse and Tompkins first assume that $A_g$ has finitely many homotopically critical points. They show that any homotopically critical point of $A_g$ is also a critical point in the classical sense.
They verify the other assumptions they make for the Morse inequalities and use them to obtain their Critical surface Theorem: if $A_g$ has two distinct minimizers, it must also have a critical point of index 1, i.e. with non-zero first cap type number.

In order to remove the finiteness assumptions, they go on to prove local-$A_g$-connectivity for $\Omega_g$ and require that the two minimizers are contained in disjoint critical sets. Again using the Morse inequalities, they infer the existence of an unstable minimal surface.
As we have mentioned before, Morse's proof in \cite{Morse.1940} of the Morse inequalities in this general case is faulty since local-$A_g$-connectedness is not enough to imply the necessary q-tameness.
However, the proof of local-$A_g$-connectedness in \cite{Morse.1939} actually shows that $\Omega_g$ is strongly locally-$A_g$-connected. In fact, Morse and Tompkins even show that each sublevel set of $A_g$ is locally contractible in all larger sublevel sets.
So if our previous conjectures are true, the general theory can be applied and their approach still works.


%The approach of Morse and Tompkins in \cite{Morse.1939} to minimal surface theory was based on the Douglas functional.
%topological definition of critical points in the general setting, prove existence of these via Morse inequalities for the so-called Douglas functional, and then to prove that these topological critical points correspond to minimal surfaces.
%For a more detailed review of their work we refer to \cite{Struwe.1988}. In the minimal surface community, these methods do not seem to be deemed very helpful, as is reflected e.gchoosing.\@ in \cite[p.472]{MR2566897}, where the authors state that Morse's "...general Morse-theoretic statements are more or less useless as they are based on topological assumptions which cannot be verified in a concrete situation".
%More comments on the perceived unsuitability of the theory of Functional Topology for minimal surfaces are made by Struwe in \cite{MR850612}. 

%For the development of functional topology, Morse assumes the spaces under consideration are equipped with a metric -- in particular are locally compact -- and uses Vietoris homology, which agrees with \v{C}ech homology by Dowker's Theorem \cite{Dowker.1952}.
%The functionals he uses are ``bounded compact", that is, all sublevel sets are compact.
%
%
%This implies semi-continuity by a general result stating that, for field coefficients, \v{C}ech homology commutes with inverse limits for compact Hausdorff spaces \cite[Theorem VIII.3.6 and Theorem X.3.1]{MR0050886}.
%
%
%
%
%
%\subsection{Critical Points}
%So far, we have mostly talked about changes in the homology of sublevel sets at certain \emph{values} since this is what is studied in persistent homology. The original motivation to study Functional Topology was, however, the study of critical \emph{points}, which we briefly touched upon in section \ref{sec:morse}.



%
\subsection{Two Alternative Conjectures}
In order to relate Morse's conditions to the above, we need a relative version of the $HLC^{\infty}$ condition.

\begin{defi}\label{defi:hlc}
Let $\mathcal{C}$ be a category with $0$ object and let $A\colon\mathbf{Top}\to\mathcal{C}$ be a functor. If $Y\subseteq X$ is a pair of topological spaces, we say that $Y$ is \emph{locally connected with respect to $A$ in $X$} if for each point $p\in Y$ and any open neighborhood $V$ of $p$ in $X$ there is an open neighborhood $U\subseteq V$ of $p$ such that $A(U\cap Y\hookrightarrow V)$ factors through 0.

We say that $Y$ is \emph{$HLC^{\infty}$ in $X$} if it is locally connected with respect to $\tilde{H}_*(-;\mathbb{Z})$ in $X$.
\end{defi}

We get the following rephrasing of strong local-$f$-connectedness.

\begin{prop}
Let $M$ be a metric space and $f\colon M\to\mathbb{R}$ a function. Then $M$ is strongly locally-$f$-connected if and only if $f_{\leq s}$ is $HLC^{\infty}$ in $f_{\leq t}$ whenever $s<t$.
\end{prop}
\begin{proof}
If each sublevel set is $HLC^{\infty}$ in larger sublevel sets, $M$ is clearly strongly locally-$f$-connected. For the converse direction, fix $s<t$ and choose some point $p\in f_{\leq s}$ with a neighborhood $V$ of $p$ in $f_{\leq t}$. We may choose $e\in(0, t-s)$ such that $B_e(p)\cap f_{\leq t}\subseteq V$. By strong local-$f$-connectedness, for each $d$ there is some $\delta\in(0,e)$ such that the map in reduced $d$-dimensional homology induced by $B_{\delta}(p)\cap f_{\leq s}\hookrightarrow B_e(p)\cap f_{\leq t}$ is trivial. The map $B_{\delta}(p)\cap f_{\leq s}\hookrightarrow V$ factors through the aforementioned map, so it is trivial in reduced homology, too. Thus, $f_{\leq s}$ is $HLC^{\infty}$ in $f_{\leq t}$.
\end{proof}

Note that the proposition gives a purely topological description of strong local-$f$-connectedness, so we can also use it to define strong local-$f$-connectedness for spaces that are not metrizable. It also yields the following corollary.

\begin{cor}
Let $M$ be a topological space and $f\colon M\to\mathbb{R}$ a function. If all sublevel sets are $HLC^{\infty}$, then $M$ is strongly locally-$f$-connected.
\end{cor}

We now propose the following generalization of \cref{prop:fin_dim_sing_hom}.

\begin{conj}\label{conj:q-tame_singular}
Let $Y\subseteq X$ be compact Hausdorff spaces such that $Y$ is $HLC^{\infty}$ in $X$. Then $\im H_d(Y\hookrightarrow X)$ is finite-dimensional for all $d$.
\end{conj}

As a persistent version, we have the following obvious consequence.

\begin{prop}\label{prop:q-tame_singular}
Let $M$ be a topological space and $f\colon M\to\mathbb{R}$ a function such that $M$ is strongly locally-$f$-connected, i.e., each sublevel set is $HLC^{\infty}$ in all larger sublevel sets. If \cref{conj:q-tame_singular} is true, then $H_{d}(f_{\leq\bullet})$ is q-tame for all $d$.
\end{prop}

In a similar spirit, we also propose the following conjecture as an analogue to \cref{cor:cech_sing_persistent_iso}.

\begin{conj}\label{con:d_I_0_sing_cech}
Let $M$ be a topological space and $f\colon M\to\mathbb{R}$ a function such that all sublevel sets are paracompact Hausdorff. If $M$ is strongly locally-$f$-connected, i.e., each sublevel set is $HLC^{\infty}$ in all larger sublevel sets, then 
\[
d_I(H_d(f_{\leq\bullet}),\CH_d(f_{\leq\bullet}))=0
\]
for all $d$.
\end{conj}

Here, $d_{I}$ denotes the interleaving distance between persistence modules. The conjecture is relevant because of the following criterion for q-tameness.

\begin{prop}
Let $M$ and $N$ be persistence modules such that $M$ is q-tame and assume $d_I(M,N)=0$. Then $N$ is q-tame.
\end{prop}
\begin{proof}
Pick indices $s<t$. Because $d_I(M,N)=0$, we may choose a $\delta$-interleaving with $\delta\in\left(0,\frac{t-s}{2}\right)$. Then we have a commutative diagram 

\[
\begin{tikzcd}
N_s\arrow[rrr]\arrow[rd]&&&N_{t}\\
&M_{s+\delta}\arrow[r]&M_{t-\delta}\arrow[ru]&
\end{tikzcd}
\]
so that the rank of $N_s\to N_t$ is bounded above by the rank of $M_{s+\delta}\to M_{t-\delta}$. The rank of $M_{s+\delta}\to M_{t-\delta}$ is finite by assumption, so $N$ is q-tame.
\end{proof}

Combining the previous result with \cref{prop:q-tame_singular}, we get the following.

\begin{cor}
If \cref{conj:q-tame_singular,con:d_I_0_sing_cech} are true, then so is \cref{con:q-tame_pers_cech_hom}.
\end{cor}

Thus, we have reduced our main conjecture to two smaller ones.

A possible proof strategy for \cref{conj:q-tame_singular} might be to adapt the proof strategy of Wilder's Finiteness Theorem, which is an analogue of \cref{prop:fin_dim_sing_hom} for sheaf cohomology, from \cite[Section II.17]{MR1481706}. Essentially the same proof also appears in \cite[Section III.10]{MR842190}.

A possible proof strategy for \cref{con:d_I_0_sing_cech} might be to adapt Bredon's proof of \cref{prop:cech_sing_hom_hlc} using his spectral sequence that relates \v{C}ech and singular homology.

\subsection{Q-Tameness of Persistent \v{C}ech Homology via Cosheaf Homology}\label{sec:cosheaf}
Note that what follows is rather informal and speculative.

When working with cohomology, the usual modern way to relate the singular and the \v{C}ech theory is via an intermediate step in the form of sheaf cohomology and then use the Leray and Grothendieck spectral sequences \cite[Chapter III]{MR1481706}. One might hope to use similar cosheaf theoretic methods, such as a possible theory of cosheaf homology, to obtain similar results for homology. So far, treatments of cosheaf homology are sparse. Among them are \cite{Andreotti.1973}, \cite{Schneiders.1987} and \cite{prasolov2018cosheaves}, but we are presently unable to determine whether the techniques in these texts can be adapted to our problems. Bredon, in \cite{Bredon.1968} and \cite[Chapter VI]{MR1481706}, also defines \v{C}ech homology with coefficients in a precosheaf and compares it to the singular theory for paracompact $HLC^{\infty}$-spaces \cite[Section VI.4 and Theorem VI.12.6]{MR1481706} using cosheaf theoretic methods. All of his results are still absolute comparison and finiteness results, i.e., they work for a single space $X$, so another obstacle, besides cosheaf homology being underdeveloped, is our desired generalization to subspaces $Y\subseteq X$ that are only $HLC^{\infty}$ in $X$ but not in themselves. That cosheaves might also help in overcoming this problem is suggested by the very nice formulation of the $HLC^{\infty}$ property in terms of precosheaves. The following definition appears in \cite{MR1481706}.

\begin{defi}
Let $\mathfrak{A}$ be a precosheaf on a space $X$ with values in a category with 0 object, i.e., a covariant functor on the poset category of open sets in $X$. We call $\mathfrak{A}$ \emph{locally trivial} if for all points $x\in X$ and all open neighborhoods $V$ of $x$ there is another open neighborhood $U\subseteq V$ of $x$ such that $\mathfrak{A}(U)\to\mathfrak{A}(V)$ factors through 0.
\end{defi}

The following definition is, up to our use of reduced homology, taken from \cite{Curry.2015}.

\begin{defi}
For any topological space $X$, we define the \emph{reduced Leray precosheaf} $\mathfrak{L}_X$ on $X$ as the precosheaf mapping $U\mapsto \tilde{H}_*(U;\mathbb{Z})$.
\end{defi}

\begin{defi}
If $f\colon Y\to X$ is a continuous map between topological spaces and $\mathfrak{A}$ is a precosheaf on $Y$, we define the \emph{pushforward of $\mathfrak{A}$ along $f$} as the precosheaf on $X$ mapping $U\mapsto\mathfrak{A}(f^{-1}(U))$. We denote it by $f_*\mathfrak{A}$.
\end{defi}

Now if $Y\subseteq X$ are topological spaces, we get an obvious morphism $i_*\mathfrak{L}_Y\to\mathfrak{L}_X$ of precosheaves on $X$, where $i\colon Y\hookrightarrow X$ is the inclusion, given by $\tilde{H}_*(U\cap Y\hookrightarrow U;\mathbb{Z})$. Note that the category of precosheaves with values in abelian groups is just a functor category from a small category to an abelian category. In particular, it has images given "pointwise". We get the following rephrasing of the relative $HLC^{\infty}$ property.

\begin{prop}
Let $Y\subseteq X$ be topological spaces. Then $Y$ is $HLC^{\infty}$ in $X$ if and only if $\im(i_*\mathfrak{L}_Y\to\mathfrak{L}_X)$ is locally trivial.
\end{prop}

Our hope is that one might be able to use this characterization in the future to prove the previously made conjectures.

\todo{Given $f\colon M\to\mathbb{R}$, is there a way of finding a single space (e.g. M itself, disjoint union of sublevel sets, colim of sublevel sets, hocolim of sublevel sets, continuously indexed mapping telescope (how to construct this?) of inclusions of sublevel sets, etc.) and a precosheaf with values in a category of persistence module (maybe the observable category) on that space such that $M$ is locally-$f$-connected iff that precosheaf is locally trivial?}

\section*{Acknowledgements}
This research has been supported by German Research Foundation (DFG) through the Collaborative Research Center SFB/TRR 109 \emph{Discretization in Geometry and Dynamics}, the Collaborative Research Center SFB/TRR 191 \emph{Symplectic Structures in Geometry, Algebra and Dynamics}, the Cluster of Excellence EXC-2181/1 \emph{STRUCTURES}, and the Research Training Group RTG 2229 \emph{Asymptotic Invariants and Limits of Groups and Spaces}.

\bibliographystyle{abbrvnaturl}
\bibliography{biblio}

\todos
\end{document}