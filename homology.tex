
\section{Homology theories} \label{s:homology}

Let $R$ be a ring and let $\H = \big\{ \H_{n} \big\}_{n \in \Z}$ be a collection of homotopy invariant functors from pairs of topological spaces to $R$-modules.
Consider the following two properties known respectively as \textit{excision} and \textit{exactness}:
\begin{itemize}
	\item If $(X,A)$ is a pair and $U \subseteq X$ is an open subspace whose closure in $X$ is contained in $A$, then the inclusion $(X \setminus U, A \setminus U) \to (X, A)$ is sent to an isomorphism by $\H_{n}$ for all $n$.
	\item For each pair $(X,A)$ there exists a natural long exact sequence
	\begin{equation*}
    \begin{tikzcd}[column sep = small]
	\cdots \arrow[r] & \H_{n+1}(X,A) \arrow[r] & \H_{n}(A, \emptyset) \arrow[r] & \H_{n}(X, \emptyset) \arrow[r] & \H_{n}(X,A) \arrow[r] & \cdots
	\end{tikzcd}
	\end{equation*}
\end{itemize}

If $\H$ satisfies these properties, we call it a \emph{homology theory}.
Nowadays, it is common to also ask of a homology theory to satisfy some form of additivity, but this property, suggested by Milnor \cite{MR159327}, was not part of Eilenberg-Steenrod axioms as presented in \cite[Chapter I]{MR0050886} and,
since it does not play a role in this work, we do not demand it.

Similarly, point axiom ....

Ulrich: uniqueness theorems.

For us, a weaker version of exactness will be important.
We say $\H$ satisfies \textit{partial exactness} if pairs of spaces give rise to long sequences satisfying the chain complex property but not needing to be exact, i.e., satisfying a condition of the form ``$\im \subseteq \ker$" instead of ``$\im = \ker$".
We say that $\H$ is a \emph{partially exact homology theory} if it satisfies excision and partial exactness.

Let us now review the Mayer-Vietories long exact sequence.
We say that a triple of spaces $X_{1}, X_{2} \subseteq X$ is \emph{excisive} with respect to $\H$ if the inclusion of $(X_{1}, X_{1} \cap X_{2})$ into $(X, X_{2})$ is sent to an isomorphism by $\H_{n}$ for all $n$.
Note that if $\H$ satisfies excision then a triple $X_{1}, X_{2} \subseteq X$ is excisive if $X = X_{1} \cup X_{2}$ and $X_{1}$, $X_{2}$ are open in $X$.

If $X_{1}, X_{2} \subseteq X$ are excisive with respect to $\H$ and the long sequences associated to $(X_{1}, X_{1} \cap X_{2})$ and $(X, X_{2})$ are exact, then there is a natural long exact sequence
\[
\begin{tikzcd}[column sep = small]
\cdots \arrow[r] & \H_{n}(X_{1} \cap X_{2}) \arrow[r] & \H_{n}(X_{1}) \oplus \H_{n}(X_{2}) \arrow[r] & \H_{n}(X) \arrow[r] & \H_{n-1}(X_{1} \cap X_{2}) \arrow[r] & \cdots
\end{tikzcd}
\]
referred to as the \emph{Mayer-Vietoris sequence} of $X_{1}, X_{2} \subseteq X$. 

In particular, for any homology theory there is an exact Mayer-Vietoris sequence for any triple of spaces $X_{1}, X_{2} \subseteq X$ if $X = X_{1} \cup X_{2}$ and $X_{1}$, $X_{2}$ are open in $X$.

An example of a homology theory consider in this work is the well-known singular theory of Eilenberg \cite{Eilenberg.1944}, whereas an example of a partially exact homology theory, reviewed next, is the \v{C}ech theory, for which a historical account can be found in \cite{Edwards.1980}.

Let $(X_{1},X_{2})$ be a pair of topological spaces. We say that $\alpha = (\alpha_1, \alpha_2)$ is an \emph{open cover} of $(X_1, X_2)$ if $\alpha_1$ is an open cover of $X_1$ and $\alpha_2 \subseteq \alpha_1$ is a subcover such that $X_1 \subseteq \bigcup_{U \in \alpha_2} U$.
We denote by $\Cov(X_{1}, X_{2})$ the set of open covers of $(X_{1}, X_{2})$ directed by the refinement relation. 

If $\alpha = (\alpha_{1}, \alpha_{2})$ is a cover of $(X_{1}, X_{2})$, we write $\Nrv(\alpha)$ for the simplicial pair $(\Nrv(\alpha_{1}),\Nrv(\alpha_{2}))$, where 
\[
\Nrv(\alpha_{i}) =
\big\{ \beta \subseteq \alpha_{i} \mid \beta \text{ is finite and } \textstyle{\bigcap_{U \in \beta}} \, U \neq \emptyset \big\}
\]
denotes the nerve of a cover. The nerve construction defines a functor from $\Cov(X_{1},X_{2})$ regarded as a category to the category of simplicial pairs. 

The \emph{\v{C}ech homology with coefficients in $G$} of $(X_{1},X_{2})$ is defined as
\[
\CH_{*}(X_{1}, X_{2}; G) \ =
\lim_{\alpha \in \Cov(X_{1}, X_{2})} H_{*}(\Nrv(\alpha); G).
\]

\v{C}ech homology can be made into a functor and satisfies the axioms of a partially exact homology theory; we refer to \cite[Section IX--X]{MR0050886} for the constructions and verifications of the axioms.
For arbitrary coefficients it is not in general a homology theory, but for vector space coefficients and compact Hausdorff pairs, the associated long sequences are exact \cite{Kelly.1961}.
If $X_{1}, X_{2} \subseteq X = X_{1} \cup X_2$ are all compact Hausdorff, they are also excisive for \v{C}ech homology \cite[Sections X.1, X.5]{MR0050886}, so such triples have exact Mayer-Vietoris sequences for \v{C}ech homology with field coefficients.

Another important property of \v{C}ech homology is its compatibility with cofiltered limits \cite[Theorems VIII.3.6.\@ and X.3.1.]{MR0050886}. More specifically, if $(X_i, Y_i)_i$ is a totally ordered diagram of compact Hausdorff spaces, its limit $(X, Y) = \lim_i (X_i, Y_i)$ in the category of pairs of spaces is again a compact Hausdorff pair and the natural map $\CH_*(X, Y; G) \to \lim_i \CH_*(X_i, Y_i; G)$ is an isomorphism.

%\todo{Uli: I saw a discussion about the topological invariance of Vietoris homology for non-compact metric spaces.
%I thought that it suggested that invariance holds only for compact spaces,
%but I can't find it anymore. 

%What I found was this:
%\url{http://pldml.icm.edu.pl/pldml/element/bwmeta1.element.zamlynska-8f369e96-b1eb-4ec9-99e2-8c366367a5a4} claims invariance also for non-compact.
%\url{https://www.ams.org/journals/tran/1947-062-02/S0002-9947-1947-0024128-6/S0002-9947-1947-0024128-6.pdf} mentions a non-invariant property "null-bounding"

%Update: Cech homology (as usually defined) and Vietoris homology (as defined originally) are only isomorphic for compact metric spaces. In our setting, this is enough. Decide which homology theories are really relevant to us.
%Vietoris (metric),
%Vietoris (topological, i.e. as the limit over all open covers);
%eqivalently (by Dowker duality):
%Cech (metric),
%Cech (topological);
%(Massey also defined Alexander-Spanier homology \url{https://doi.org/10.2307/2321782}. It should be equivalent to the "topological" variants here.)
%Singular;
%}

