
\section{Introduction}

The interplay between the critical set of a function and the topology of its domain is a cornerstone of modern mathematics.
Nowadays, when thinking about the pioneering work of Marston Morse, our first thought probably considers a differentiable function on a closed smooth manifold, but more general settings should also be regarded.
Morse theory in the smooth context was masterfully presented in Milnor's famous book on the subject \cite{Milnor.1963}, where he also gave a new proof of Bott's periodicity by applying Morse theory to the energy functional of paths in a Riemannian manifold, which notably goes beyond the compact setting.
Another important example of the use of Morse's insights in an infinite context is Floer's work on the Arnold conjecture and its many ramifications in symplectic topology, as surveyed for example in \cite{Salamon.1999}.
Morse himself worked in a very general setting, publishing in the 1930s a series of papers \cite{Morse.1937, Morse.1938, Morse.1940} in which he established the key results of Morse theory in the broad context defined by semi-continuous functionals on metric spaces.
He named this body of work \emph{functional topology} and used it to study questions about minimal surfaces motivated by Douglas' solution to Plateau’s Problem \cite{Douglas.1931}.
In particular, Morse and Tompkins \cite{Morse.1939, Morse.1941} used these techniques to prove the Unstable Minimal Surface Theorem, a result providing conditions for the existence of critical points of the Douglas functional that are not local minima.

While Morse's work on functional topology did not have a long lasting impact on minimal surface theory or the calculus of variations in general ---possibly in part because, as Hildebrandt [4, p. 324] puts it, ``Morse--Tompkins' original paper [is] unreadable and inaccessible for the non-specialist''---
many of his ideas have resurfaced and flourished in other domains.
In particular, in applied topology and topological data analysis, several key insights of Morse have been independently rediscovered as part of the development of \emph{persistent homology}, a technique that provides robust and efficiently computable invariants of filtered spaces using the functorial properties of homology.
Its success in applied topology has motivated a refined abstract theory of persistence that lies on the intersection of geometry, topology, and representation theory.

The homology of a filtered space is an example of what is called a \emph{persistence module}, a functor to vector spaces from the real numbers considered as a poset category.
In many important cases, a persistence module admits an essentially unique decomposition into indecomposable direct summands, and the structure of this decomposition yields the basic invariant known as its \emph{persistence diagram}.

The influence of persistence theory and its sophistication have seen rapid growth in recent years.
One of the primary reasons for this is the fact that persistence diagrams can be organized into a metric space, often allowing to recast geometric questions about general filtered spaces in a combinatorial metric model.
This is because the passage via the homology construction to this metric space is Lipschitz, a statement commonly known as the \emph{stability} of persistence diagrams.

The most remarkable connections between functional topology and persistence theory come from Morse's paper \cite{Morse.1940}, where he developed the theory of \emph{caps} and their \emph{spans}, capturing much of the same information as the modern notion of persistence diagram, including concepts such as the persistence or birth-death of a topological feature.
Morse used his theory of caps to study functionals on a metric space by analyzing the evolution of the topology of their sublevel sets.
A key tool for this end is a version of his eponymous inequalities for cap numbers, which expands their usual version in the compact and smooth setting.
In this work, using persistence diagrams, we generalize the definition of these cap numbers to persistence modules and prove the existence of Morse inequalities for a large class of them (\cref{t:inequalities}).
Our approach makes these inequalities accessible in new contexts beyond those originally covered by functional topology, including, for example, symplectic geometry.

Given the importance of persistence diagrams for stating and proving Morse inequalities, our focus will then be on the study of topological properties ensuring their existence for a broad class of filtered spaces and homology constructions.
For general persistence modules, a well studied condition for the existence of persistence diagrams is \emph{q-tameness} \cite{Chazal.2016a,Chazal.2016b}, which simply states that all linear maps between different real values in the persistence module have finite dimensional rank.
The motivating question can then be reformulated as asking for topological conditions on a filtered space that ensure its persistent homology to be q-tame.
In fact, the problem of finding such conditions was already considered by Morse, who studied certain local connectivity properties of a filtered metric space and claimed them to suffice for the q-tameness of the associated persistent homology.
Contrary to these claims, we show that the local connectivity condition used by Morse and Tompkins in the proof of the Unstable Minimal Surface Theorem is insufficient (\cref{c: counterexample}).
We introduce two similar but more widely applicable conditions, and show that the first suffices for the \mbox{q-tameness} of filtrations defined by continuous functionals (\cref{c:q-tameness for continuous functions}), and the second for those defined by functionals with compact sublevel sets (\cref{t:strong local connectedness implies q-tameness}).
Since Douglas' functional has compact sublevel sets but is not continuous in the $C^0$ topology, which Morse and Tompkins considered, we use this second condition to correct their proof.

\subsection*{Summary}

The primary contribution of this work consists in the introduction of topological conditions on a broad class of filtered spaces ensuring their associated persistent homology modules satisfy Morse inequalities.
As an application, we identify a mistake in the proof the Unstable Minimal Surface Theorem of Morse and Tompkins and fix it with our results. 

\subsection*{Outline}

In \cref{s:persistence} we recall the foundations of persistence theory where, for us, the persistent homology of a sublevel set filtration is the key example.
We present a persistence-theoretic point of view on Morse inequalities in \cref{s:inequalities}.
It generalizes both their version in the smooth and compact setting as well as the one used in functional topology.
The core of this work is presented in \cref{s:connectivity}, where we define two natural notions of local-connectivity for a sublevel set filtration and show under what circumstances they imply q-tameness for its associated persistence module.
We close with a historical overview of Morse's functional topology and its relation to our results in \cref{s:surfaces}.
\cref{s:vietoris} contains a brief discussion on the definitions of Vietoris and \v{C}ech homology and a comparison between them based on Dowker's Theorem.