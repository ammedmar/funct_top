
\section{Introduction}

\begin{itemize}
	\item Persistence homology, 
	\item applications in applied topology, application in theoretical topology, 
	\item barcode, stability, existence of barcode motivates q-tameness. 
	\item When can we ensure this condition? 
	\item Morse as a precursor. Geodesic study, natural next context: minimal surfaces, \item proposed condition for q-tameness, it's wrong, 
	\item a similar condition, assumed equivalent, we prove this implies it.
	\item We expect this paper to help the ongoing development of persistence methods in geometry.
	\item Floer theory and simplectic geometry.
\end{itemize}

Nowadays, when thinking of Marston Morse's work, the first thing that comes to mind is the theory of critical points of his eponymous Morse functions, as presented e.g.\@ in Milnor's famous book \cite{MR0163331}, with a key result being the Morse inequalities for the number of critical points of a function. In the 1930s, however, Morse published a series of papers \cite{Morse.1937, Morse.1938, Morse.1940, MR9102} in which he established a much more general theory of critical points and Morse inequalities for not even necessarily continuous functions on metric spaces. He termed this general theory Functional Topology because his goal was to use it to study critical points of functionals by topological means. In particular, he was interested in existence results for unstable minimal surfaces, i.e., critical points of the area functional for a given boundary curve that are not minimizers of area \cite{Morse.1939,Morse.1941}. This work with Tompkins was building on the work of Douglas who solved the general version of Plateau's Problem \cite{Douglas.1931}.

The approach of Morse and Tompkins in \cite{Morse.1939} was to give a topological definition of critical points in the general setting, prove existence of these via Morse inequalities for the so-called Douglas functional, and then to prove that these topological critical points correspond to minimal surfaces. For a more detailed review of their work we refer to \cite{Struwe.1988}. In the minimal surface community, these methods do not seem to be deemed very helpful, as is reflected e.gchoosing.\@ in \cite[p.472]{MR2566897}, where the authors state that Morse's "...general Morse-theoretic statements are more or less useless as they are based on topological assumptions which cannot be verified in a concrete situation." More comments on the perceived unsuitability of the theory of Functional Topology for minimal surfaces are made by Struwe in \cite{MR850612}. 

By now, however, Morse's ideas from Functional Topology, and in particular from \cite{Morse.1940}, have been rediscovered in a different context, namely topological data analysis. More specifically, Morse studies how the homology of the sublevel sets of a function change as the level changes. He defines spaces that encode the same information as the modern persistent homology groups, as well as quantities that correspond to the modern notions of birth and death. Even though he did not have a structure theory of persistence modules via barcodes, he emphasized many points that are now known to be connected to this theory: The use of field coefficients, as well as ensuring that the occuring persistence modules are upper semi-continuous and q-tame (which implies the existence of interval decompositions \cite{schmahl2020structure}).