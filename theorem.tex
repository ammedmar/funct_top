
\section{Theorem} \label{sec:theorem}

In this section we introduce a stronger version of $\HLC$ for sublevel filtrations, and we show that Morse filtrations with this property are q-tame.

\begin{defi}
	A sublevel filtration $(X,f)$ is said to be \textit{strongly} $\HLC$ if for each $x \in X$, neighborhood $V$ of $x$ and $\epsilon > 0$ there exists $U$ a neighborhood of $x$ with $U \subseteq V$ and $\delta \in (0, \epsilon)$ such that
	\begin{equation*}
	X_{\leq f(x) + \delta + c} \cap U \to X_{f(x) + \epsilon + c} \cap V
	\end{equation*}
	is $\HT$ for every $c \geq 0$.
\end{defi}

\begin{thm} \label{t:strong local connectenss implies q-tameness}
	Any be a strongly $\HLC$ Morse filtration $(X,f)$ is q-tame, i.e., for every $s < t$ the inclusion $X_s \to X_t$ is $\HS$.
\end{thm}

\begin{lem} \label{l:commutative algebra}
	Given a commutative diagram of modules over a principal ideal domain
	\begin{equation*}
	\begin{tikzcd}
	A_{1,1} \arrow[r] & A_{1,2} & \\
	A_{2,1} \arrow[r] \arrow[u] & A_{2,2} \arrow[r] \arrow[u] & A_{2,3} \\
	& A_{3,2} \arrow[r] \arrow[u] & A_{3,3} \arrow[u]
	\end{tikzcd}
	\end{equation*}
	where the middle row is exact and both $A_{2,1} \to A_{1,1}$ and $A_{3,3} \to A_{2,3}$ have finitely generated images, then $A_{3,2} \to A_{1,2}$ does as well.
\end{lem}

\begin{proof}
	This is proven via a straightforward diagram chase. For complete details see Lemma 17.3 in \cite{Bredon.1968}.
\end{proof}

\begin{lem} \label{l:neighborhood third}
	Let $X$ be locally compact space.
	For any compact subset $K$ and open set $U$ with $K \subseteq U$ there exists a compact set $K^\prime$ such that
	\begin{equation*}
	K \subseteq \interior(K^\prime) \subseteq K^\prime \subseteq U.
	\end{equation*}
\end{lem}

\begin{proof}
	For any $x \in K$ choose a compact neighborhood $C(x) \subseteq U$.
	We have
	\begin{equation*}
	K \subseteq \bigcup_K \interior(C(x)) \subseteq \interior\left(\bigcup_K C(x)\right) \subseteq \bigcup_K C(x) \subseteq U
	\end{equation*}
	Since $K$ is compact, the first inclusion above is achieved over a finite subset $\{x_1, \dots, x_m\}$ of elements in $K$.
	Defining $K^\prime = \bigcup_{i=1}^m C(x_i)$ finishes the proof.
\end{proof}

\begin{lem} \label{l:key lemma for q-tameness}
	Fix a homology theory. Let $M$ and $f$ be as in Theorem \ref{t:strong local connectenss implies q-tameness}.
	Consider sets $K, L \subseteq M$ with $K$ compact and $K \subseteq \interior(L)$. For any $s < t$ there is $\delta \in (0,\, t-s)$ such that $K \cap M_{s+\delta} \to L \cap M_{t}$ is $\HS$.
\end{lem}

\begin{proof}
	The lemma holds for $\HS$ replaced by $\HS_{(n-1)}$ for any $n \leq 0$ since $H_{n-1}(-)$ induces the zero map. We will proceed by induction on $n$ assuming the lemma for $\HS_{(n-1)}$. 
	
	Given a compact set $L \subseteq M$ and $s < t$ let $\Sigma_{s, t}$ be the collection of all compact subsets $K$ of $\interior(L)$ for which there exists $\delta_K > 0$ and an open neighborhood $U_K$ of $K$ such that $U_K \cap M_{s+\delta_K} \to L \cap M_{t}$ is $\HS_n$.
	
	We start by showing that any point in $\interior(L) \cap M_s$ has a neighborhood in $\Sigma_{s, t}$.
	Let $x \in \interior(L) \cap M_{s}$ and take $e > 0$ such that $s + e < t$ and $B_e(x) \subseteq \interior(L)$.
	By the strong local-$f$-connectivity of $M$, there exists $\delta \in (0, e)$ such that for $c = s - f(x)$ the following composition is $\HS$:
	\begin{equation*}
	\overline B_{\delta/2}(x) \cap M_{s + \delta} \to
	B_\delta(x) \cap M_{s + \delta} =
	B_\delta(x) \cap M_{f(x) + c + \delta} \to
	B_e(x) \cap M_{f(x) + c + e} \to
	L \cap M_{t}.
	\end{equation*}  
	
	We will now show that the class $\Sigma_{s,t}$ is closed under finite unions.
	For $i \in \{1, 2\}$ let $K_i$ be in $\Sigma_{s,t}$ with $\delta_i > 0$ and $K_i \subseteq U_i$ open such that $U_{i} \cap M_{s+\delta_i} \to L \cap M_{t}$ is $\HS_n$.
	We Use Lemma \ref{l:neighborhood third} to construct sets $K_i^\prime$ such that
	\begin{equation*}
	K_i \subseteq \interior(K_i^\prime) \subseteq K_i^\prime \subseteq U_i.
	\end{equation*}
	Notice that for $\delta = \min(\delta_i)$ we have $K_i^\prime \cap M_{s+\delta} \to L \cap M_t$ is $\HS_n$.
	Additionally, the induction hypothesis implies that $K_1 \cap K_2 \cap M_s \to K_1^\prime \cap K_2^\prime \cap M_{s+\delta}$ is $\HS_{(n-1)}$.
	We therefore have the following commutative diagram satisfying the assumptions of Lemma~\ref{l:commutative algebra}:
	\begin{equation*}
	\begin{tikzcd}
	\H_n(L \cap M_t) \oplus \H_n(L \cap M_t) \arrow[r] &
	\H_n(L \cap M_t) & \\
	\H_{n}(K_1^\prime \cap M_{s+\delta}) \oplus \H_n(K_2^\prime \cap M_{s+\delta}) \arrow[r] \arrow[u] & 
	\H_{n}((K_1^\prime \cap M_{s+\delta}) \cup (K_2^\prime \cap M_{s+\delta})) \arrow[r] \arrow[u] &
	\H_{n-1}(K_1^\prime \cap K_2^\prime \cap M_{s+\delta}) \\ & 
	\H_{n}((K_1 \cup K_2) \cap M_s) \arrow[r] \arrow[u] &
	\H_{n-1}(K_1 \cap K_2 \cap M_s). \arrow[u]
	\end{tikzcd}
	\end{equation*}
	We conclude that $K_1 \cup K_2 \in \Sigma_{s, t}$.
	Since any compact $K \subseteq \interior(L)$ can be expressed as a finite union of sets in $\Sigma_{s,t}$ the induction step and the lemma are proven.
\end{proof}

\begin{proof}[Proof of Theorem \ref{t:strong local connectenss implies q-tameness}]
	It follows from applying Lemma~\ref{l:key lemma for q-tameness} to $K = M_{\leq s}$ and $L = M$.
\end{proof}