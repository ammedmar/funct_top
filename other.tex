
\subsection{Two Alternative Conjectures}
In order to relate Morse's conditions to the above, we need a relative version of the $HLC^{\infty}$ condition.

\begin{defi}\label{defi:hlc}
Let $\mathcal{C}$ be a category with $0$ object and let $A\colon\mathbf{Top}\to\mathcal{C}$ be a functor. If $Y\subseteq X$ is a pair of topological spaces, we say that $Y$ is \emph{locally connected with respect to $A$ in $X$} if for each point $p\in Y$ and any open neighborhood $V$ of $p$ in $X$ there is an open neighborhood $U\subseteq V$ of $p$ such that $A(U\cap Y\hookrightarrow V)$ factors through 0.

We say that $Y$ is \emph{$HLC^{\infty}$ in $X$} if it is locally connected with respect to $\tilde{H}_*(-;\mathbb{Z})$ in $X$.
\end{defi}

We get the following rephrasing of strong local-$f$-connectedness.

\begin{prop}
Let $M$ be a metric space and $f\colon M\to\mathbb{R}$ a function. Then $M$ is strongly locally-$f$-connected if and only if $f_{\leq s}$ is $HLC^{\infty}$ in $f_{\leq t}$ whenever $s<t$.
\end{prop}
\begin{proof}
If each sublevel set is $HLC^{\infty}$ in larger sublevel sets, $M$ is clearly strongly locally-$f$-connected. For the converse direction, fix $s<t$ and choose some point $p\in f_{\leq s}$ with a neighborhood $V$ of $p$ in $f_{\leq t}$. We may choose $e\in(0, t-s)$ such that $B_e(p)\cap f_{\leq t}\subseteq V$. By strong local-$f$-connectedness, for each $d$ there is some $\delta\in(0,e)$ such that the map in reduced $d$-dimensional homology induced by $B_{\delta}(p)\cap f_{\leq s}\hookrightarrow B_e(p)\cap f_{\leq t}$ is trivial. The map $B_{\delta}(p)\cap f_{\leq s}\hookrightarrow V$ factors through the aforementioned map, so it is trivial in reduced homology, too. Thus, $f_{\leq s}$ is $HLC^{\infty}$ in $f_{\leq t}$.
\end{proof}

Note that the proposition gives a purely topological description of strong local-$f$-connectedness, so we can also use it to define strong local-$f$-connectedness for spaces that are not metrizable. It also yields the following corollary.

\begin{cor}
Let $M$ be a topological space and $f\colon M\to\mathbb{R}$ a function. If all sublevel sets are $HLC^{\infty}$, then $M$ is strongly locally-$f$-connected.
\end{cor}

We now propose the following generalization of \cref{prop:fin_dim_sing_hom}.

\begin{conj}\label{conj:q-tame_singular}
Let $Y\subseteq X$ be compact Hausdorff spaces such that $Y$ is $HLC^{\infty}$ in $X$. Then $\im H_d(Y\hookrightarrow X)$ is finite-dimensional for all $d$.
\end{conj}

As a persistent version, we have the following obvious consequence.

\begin{prop}\label{prop:q-tame_singular}
Let $M$ be a topological space and $f\colon M\to\mathbb{R}$ a function such that $M$ is strongly locally-$f$-connected, i.e., each sublevel set is $HLC^{\infty}$ in all larger sublevel sets. If \cref{conj:q-tame_singular} is true, then $H_{d}(f_{\leq\bullet})$ is q-tame for all $d$.
\end{prop}

In a similar spirit, we also propose the following conjecture as an analogue to \cref{cor:cech_sing_persistent_iso}.

\begin{conj}\label{con:d_I_0_sing_cech}
Let $M$ be a topological space and $f\colon M\to\mathbb{R}$ a function such that all sublevel sets are paracompact Hausdorff. If $M$ is strongly locally-$f$-connected, i.e., each sublevel set is $HLC^{\infty}$ in all larger sublevel sets, then 
\[
d_I(H_d(f_{\leq\bullet}),\CH_d(f_{\leq\bullet}))=0
\]
for all $d$.
\end{conj}

Here, $d_{I}$ denotes the interleaving distance between persistence modules. The conjecture is relevant because of the following criterion for q-tameness.

\begin{prop}
Let $M$ and $N$ be persistence modules such that $M$ is q-tame and assume $d_I(M,N)=0$. Then $N$ is q-tame.
\end{prop}
\begin{proof}
Pick indices $s<t$. Because $d_I(M,N)=0$, we may choose a $\delta$-interleaving with $\delta\in\left(0,\frac{t-s}{2}\right)$. Then we have a commutative diagram 

\[
\begin{tikzcd}
N_s\arrow[rrr]\arrow[rd]&&&N_{t}\\
&M_{s+\delta}\arrow[r]&M_{t-\delta}\arrow[ru]&
\end{tikzcd}
\]
so that the rank of $N_s\to N_t$ is bounded above by the rank of $M_{s+\delta}\to M_{t-\delta}$. The rank of $M_{s+\delta}\to M_{t-\delta}$ is finite by assumption, so $N$ is q-tame.
\end{proof}

Combining the previous result with \cref{prop:q-tame_singular}, we get the following.

\begin{cor}
If \cref{conj:q-tame_singular,con:d_I_0_sing_cech} are true, then so is \cref{con:q-tame_pers_cech_hom}.
\end{cor}

Thus, we have reduced our main conjecture to two smaller ones.

A possible proof strategy for \cref{conj:q-tame_singular} might be to adapt the proof strategy of Wilder's Finiteness Theorem, which is an analogue of \cref{prop:fin_dim_sing_hom} for sheaf cohomology, from \cite[Section II.17]{MR1481706}. Essentially the same proof also appears in \cite[Section III.10]{MR842190}.

A possible proof strategy for \cref{con:d_I_0_sing_cech} might be to adapt Bredon's proof of \cref{prop:cech_sing_hom_hlc} using his spectral sequence that relates \v{C}ech and singular homology.

\subsection{Q-Tameness of Persistent \v{C}ech Homology via Cosheaf Homology}\label{sec:cosheaf}
Note that what follows is rather informal and speculative.

When working with cohomology, the usual modern way to relate the singular and the \v{C}ech theory is via an intermediate step in the form of sheaf cohomology and then use the Leray and Grothendieck spectral sequences \cite[Chapter III]{MR1481706}. One might hope to use similar cosheaf theoretic methods, such as a possible theory of cosheaf homology, to obtain similar results for homology. So far, treatments of cosheaf homology are sparse. Among them are \cite{Andreotti.1973}, \cite{Schneiders.1987} and \cite{prasolov2018cosheaves}, but we are presently unable to determine whether the techniques in these texts can be adapted to our problems. Bredon, in \cite{Bredon.1968} and \cite[Chapter VI]{MR1481706}, also defines \v{C}ech homology with coefficients in a precosheaf and compares it to the singular theory for paracompact $HLC^{\infty}$-spaces \cite[Section VI.4 and Theorem VI.12.6]{MR1481706} using cosheaf theoretic methods. All of his results are still absolute comparison and finiteness results, i.e., they work for a single space $X$, so another obstacle, besides cosheaf homology being underdeveloped, is our desired generalization to subspaces $Y\subseteq X$ that are only $HLC^{\infty}$ in $X$ but not in themselves. That cosheaves might also help in overcoming this problem is suggested by the very nice formulation of the $HLC^{\infty}$ property in terms of precosheaves. The following definition appears in \cite{MR1481706}.

\begin{defi}
Let $\mathfrak{A}$ be a precosheaf on a space $X$ with values in a category with 0 object, i.e., a covariant functor on the poset category of open sets in $X$. We call $\mathfrak{A}$ \emph{locally trivial} if for all points $x\in X$ and all open neighborhoods $V$ of $x$ there is another open neighborhood $U\subseteq V$ of $x$ such that $\mathfrak{A}(U)\to\mathfrak{A}(V)$ factors through 0.
\end{defi}

The following definition is, up to our use of reduced homology, taken from \cite{Curry.2015}.

\begin{defi}
For any topological space $X$, we define the \emph{reduced Leray precosheaf} $\mathfrak{L}_X$ on $X$ as the precosheaf mapping $U\mapsto \tilde{H}_*(U;\mathbb{Z})$.
\end{defi}

\begin{defi}
If $f\colon Y\to X$ is a continuous map between topological spaces and $\mathfrak{A}$ is a precosheaf on $Y$, we define the \emph{pushforward of $\mathfrak{A}$ along $f$} as the precosheaf on $X$ mapping $U\mapsto\mathfrak{A}(f^{-1}(U))$. We denote it by $f_*\mathfrak{A}$.
\end{defi}

Now if $Y\subseteq X$ are topological spaces, we get an obvious morphism $i_*\mathfrak{L}_Y\to\mathfrak{L}_X$ of precosheaves on $X$, where $i\colon Y\hookrightarrow X$ is the inclusion, given by $\tilde{H}_*(U\cap Y\hookrightarrow U;\mathbb{Z})$. Note that the category of precosheaves with values in abelian groups is just a functor category from a small category to an abelian category. In particular, it has images given "pointwise". We get the following rephrasing of the relative $HLC^{\infty}$ property.

\begin{prop}
Let $Y\subseteq X$ be topological spaces. Then $Y$ is $HLC^{\infty}$ in $X$ if and only if $\im(i_*\mathfrak{L}_Y\to\mathfrak{L}_X)$ is locally trivial.
\end{prop}

Our hope is that one might be able to use this characterization in the future to prove the previously made conjectures.

\todo{Given $f\colon M\to\mathbb{R}$, is there a way of finding a single space (e.g. M itself, disjoint union of sublevel sets, colim of sublevel sets, hocolim of sublevel sets, continuously indexed mapping telescope (how to construct this?) of inclusions of sublevel sets, etc.) and a precosheaf with values in a category of persistence module (maybe the observable category) on that space such that $M$ is locally-$f$-connected iff that precosheaf is locally trivial?}