
\section{Vietoris and \v{C}ech homology} \label{s:vietoris}

As we have mentioned before, Morse's original treatment of functional topology relied on him studying the Vietoris homology \cite{MR1512371} of the sublevel set filtration induced by a function on a metric space. 
For ease of reference, we have instead talked about \v{C}ech homology, which will now be justified by recalling that these two homology constructions agree for compact metric spaces.

First, we recall the definition of \v{C}ech homology as presented for example by \citet[Section IX--X]{MR0050886}.
Let $X$ be a topological space and let $\Cov(X)$ be the set of all open covers of $X$ ordered by the refinement relation. 
Recall that for an open cover $\alpha \in \Cov(X)$ its \emph{nerve} $\Nrv(\alpha)$ is defined as the simplicial complex
\[
\Nrv(\alpha) =
\big\{ \beta \subseteq \alpha \mid \beta \text{ is finite and } \textstyle{\bigcap_{U \in \beta}} \, U \neq \emptyset \big\}.
\]
The nerve construction defines a functor from the poset $\Cov(X_{1},X_{2})$ regarded as a category to the category of simplicial complexes. 
The \emph{\v{C}ech homology with coefficients in $\mathbb{F}$} of $X$ is defined as
\[
\CH_{*}(X; \mathbb{F}) \ =
\lim_{\alpha \in \Cov(X)} H_{*}(\Nrv(\alpha); \mathbb{F}),
\]
which extends to a homotopy invariant functor and which admits boundary operators in a natural way.

As an alternative to the nerve construction, for a cover $\alpha \in \Cov(X)$ one can define $\Vietoris(\alpha)$ as the simplicial complex
\[
\Vietoris (\alpha) = \left\{ \sigma \subseteq X \mid \sigma \text{ is finite and } \sigma \in U \text{ for some } U \in \alpha  \right\},
\]
which again yields a functor from $\Cov(X)$ to simplicial complexes.
This construction is dual in the sense of Dowker's Theorem \cite{Dowker.1952} to the nerve construction.
As a consequence, we have that
$
H_{*} (\Nrv (\alpha); \mathbb{F}) \cong H_{*} (\Vietoris (\alpha); \mathbb{F}).
$
This isomorphism is natural with respect to refinement and compatible with the boundary operators, so that we get an alternative description of \v{C}ech homology as 
%the inverse limit over the homology of Vietoris complexes of covers.
\[
\CH (X; \mathbb{F}) \ =
\lim_{\alpha \in \Cov (X)} H_{*} (\Vietoris (\alpha); \mathbb{F}).
\]

If $X$ is a metric space, this is still not exactly the same as the construction of Vietoris homology as presented in \cite{MR1512371} and as used by Morse, which in modern notation is the limit
\begin{equation*}
    \lim_{\alpha \in \Balls(X)} H_{*} (\Vietoris (\alpha); \mathbb{F}),
\end{equation*}
where 
\[
\Balls (X) = \left\{ ( B_{\delta} (x) )_{x \in X} \mid \delta > 0 \right\}
\subseteq \Cov (X).
\]
However, if the metric space $X$ is compact, then $\Balls (X)$ is coinitial in $\Cov (X)$, that is to say, they define the same limit.
Thus, we have a natural isomorphism
\[
\CH_{*} (X; \mathbb{F}) \ \cong \,
\lim_{\alpha \in \Balls(X)} H_{*} (\Vietoris (\alpha); \mathbb{F})
\]
for a compact metric space $X$ and so the metric Vietoris homology theory employed in Morse's setting is indeed canonically isomorphic to the \v{C}ech homology we considered earlier on.
