
\section{Vietoris and \texorpdfstring{\v{C}}{}ech homology} \label{s:vietoris}  % command preventing a hyperref warning

An interesting aspect of Morse's work from the modern point of view is his use of metric Vietoris homology with coefficients in a field, which closely relates to the use of filtered Vietoris-Rips and \v{C}ech complexes in topological data analysis.

For greater generality, instead of metric Vietoris homology in this paper we consider \v{C}ech homology instead.
We now justify this choice by showing that these two constructions for homology agree on compact metric spaces.

First, let us recall the definition of \v{C}ech homology as presented for example by \citet[Section IX--X]{Eilenberg.1952}.
Let $X$ be a topological space and let $\Cov(X)$ be the set of all open covers of $X$ ordered by the refinement relation. 
Recall that for an open cover $\alpha \in \Cov(X)$ its \emph{nerve} $\Nrv(\alpha)$ is defined as the simplicial complex
\begin{equation*}
\Nrv(\alpha) =
\big\{ \beta \subseteq \alpha \mid \beta \text{ is finite and } \textstyle{\bigcap_{U \in \beta}} \, U \neq \emptyset \big\}.
\end{equation*}
The nerve construction defines a functor from the poset $\Cov(X)$ regarded as a category to that of simplicial complexes. 
The \emph{\v{C}ech homology with coefficients in $\mathbb{F}$} of $X$ is defined as
\begin{equation*}
\CH(X; \mathbb{F}) \ =
\lim_{\alpha \in \Cov(X)} H(\Nrv(\alpha); \mathbb{F}).
\end{equation*}
As an alternative to the nerve construction, for a cover $\alpha \in \Cov(X)$ one can define $\Vietoris(\alpha)$ as the simplicial complex
\begin{equation*}
\Vietoris (\alpha) = \left\{ \sigma \subseteq X \mid \sigma \text{ is finite and } \sigma \in U \text{ for some } U \in \alpha \right\},
\end{equation*}
which again yields a functor from $\Cov(X)$ to simplicial complexes.
This construction is dual to the nerve construction in the sense of Dowker's Theorem \cite{Dowker.1952}, which asserts that the two complexes $\Nrv (\alpha)$ and $\Vietoris (\alpha)$ are homotopy equivalent.
As a consequence, we have that $H (\Nrv (\alpha); \mathbb{F}) \cong H (\Vietoris (\alpha); \mathbb{F})$.
This isomorphism is natural with respect to refinement of covers, so we get an alternative description of \v{C}ech homology as 
\begin{equation*}
\CH (X; \mathbb{F}) \ \cong
\lim_{\alpha \in \Cov (X)} H (\Vietoris (\alpha); \mathbb{F}).
\end{equation*}

If $X$ is a metric space, this is still not exactly the same as the construction of metric Vietoris homology presented in \cite{Vietoris.1927} and used by Morse, which in modern notation is the limit
\begin{equation*}
\lim_{\alpha \in \Balls(X)} H (\Vietoris (\alpha); \mathbb{F}),
\end{equation*}
where 
\begin{equation*}
\Balls (X) = \left\{ ( B_{\delta} (x) )_{x \in X} \mid \delta > 0 \right\}
\subseteq \Cov (X).
\end{equation*}
However, if the metric space $X$ is compact, then $\Balls (X)$ is coinitial in $\Cov (X)$, that is to say, they both define the same limit.
Thus, we have a natural isomorphism
\begin{equation*}
\CH (X; \mathbb{F}) \ \cong \,
\lim_{\alpha \in \Balls(X)} H (\Vietoris (\alpha); \mathbb{F})
\end{equation*}
for a compact metric space $X$.
In other words, the metric Vietoris homology theory employed in Morse's setting is canonically isomorphic to the \v{C}ech homology we consider in this paper.
