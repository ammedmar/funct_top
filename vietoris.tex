
\section{Vietoris homology} \label{s:vietoris}

Let $(X_{1},X_{2})$ be a pair of topological spaces. We say that $\alpha = (\alpha_1, \alpha_2)$ is an \emph{open cover} of $(X_1, X_2)$ if $\alpha_1$ is an open cover of $X_1$ and $\alpha_2 \subseteq \alpha_1$ is a subcover such that $X_1 \subseteq \bigcup_{U \in \alpha_2} U$.
We denote by $\Cov(X_{1}, X_{2})$ the set of open covers of $(X_{1}, X_{2})$ directed by the refinement relation. 

If $\alpha = (\alpha_{1}, \alpha_{2})$ is a cover of $(X_{1}, X_{2})$, we write $\Nrv(\alpha)$ for the simplicial pair $(\Nrv(\alpha_{1}),\Nrv(\alpha_{2}))$, where 
\[
\Nrv(\alpha_{i}) =
\big\{ \beta \subseteq \alpha_{i} \mid \beta \text{ is finite and } \textstyle{\bigcap_{U \in \beta}} \, U \neq \emptyset \big\}
\]
denotes the nerve of a cover $\alpha_{i}$.
The nerve construction defines a functor from $\Cov(X_{1},X_{2})$ regarded as a category to the category of simplicial pairs. 

The \emph{\v{C}ech homology with coefficients in $\mathbb{F}$} of $(X_{1},X_{2})$ is defined as
\[
\CH_{*}(X_{1}, X_{2}; \mathbb{F}) \ =
\lim_{\alpha \in \Cov(X_{1}, X_{2})} H_{*}(\Nrv(\alpha); \mathbb{F}).
\]

In his work on functional topology, Morse uses Vietoris homology for metric spaces as introduced in \cite{MR1512371}. So far, we have considered \v{C}ech homology instead, applying a topological rather than a metric construction.
We will review the definition of Vietoris homology and recall that these two constructions yield the same result for compact metric spaces.

\v{C}ech homology of a pair of spaces $(X_{1}, X_{2})$ was defined as the inverse limit over the homology of nerves of covers of $(X_{1}, X_{2})$.
As an alternative to the nerve construction, for a cover $\alpha = (\alpha_{1}, \alpha_{2})$ one can define $\Vietoris(\alpha)$ as the simplicial pair $\big(\Vietoris (\alpha_{1}), \Vietoris (\alpha_{2})\big)$, where 
\[
\Vietoris (\alpha_{i}) = \left\{ \sigma \subseteq X_{i} \mid \sigma \text{ is finite and } \sigma \in U \text{ for some } U \in \alpha_{i}  \right\}.
\]
This construction is dual in the sense of Dowker's Theorem \cite{Dowker.1952} to the nerve construction.
As a consequence, we have that
$
H_{*} (\Nrv (\alpha); G) \cong H_{*} (\Vietoris (\alpha); G)
$
for any coefficient group $G$.
This isomorphism is natural with respect to refinement and compatible with the boundary operators, so that we get an alternative description of \v{C}ech homology as 
%the inverse limit over the homology of Vietoris complexes of covers.
\[
\lim_{\alpha \in \Cov (X_{1}, X_{2})} H_{*} (\Vietoris (\alpha); G) \, \cong\, \CH (X_{1}, X_{2}; G)
\]
This is still not exactly the same as the construction of Vietoris homology of a pair of metric spaces $(X_{1}, X_{2})$ as presented in \cite{MR1512371}, which in modern notation is the limit
\begin{equation*}
    \lim_{\alpha \in \Balls(X_{1}, X_{2})} H_{*} (\Vietoris (\alpha); G),
\end{equation*}
where 
\[
\Balls (X_{1}, X_{2}) = \left\{ \big( ( B_{\delta} (x) )_{x \in X_{1}} , ( B_{\delta} (x) )_{x \in X_{2}} \big) \mid \delta > 0 \right\}
%\left\{ \big(B_{\delta}(x),\ B_{\delta}(y) \big) \right\}_{(x,y) \in X_{1} \times X_2} 
\subseteq \Cov (X_{1}, X_{2}).
\]
However, if the metric pair $(X_{1}, X_{2})$ is compact, then $\Balls (X_{1}, X_{2})$ is coinitial in $\Cov (X_{1}, X_{2})$, that is to say, they define the same limit.
Thus, we have natural isomorphisms
\[
\lim_{\alpha \in \Balls(X_{1}, X_{2})} H_{*} (\Vietoris (\alpha); G) \ \cong \,
%\lim_{\alpha \in \Cov (X_{1}, X_{2})} H_{*} (\Vietoris (\alpha); G) \, \cong\, 
\CH (X_{1}, X_{2}; G),
\]
and so the metric Vietoris homology theory employed in Morse's setting is indeed canonically isomorphic to the \v{C}ech homology we considered earlier on.